%% Generated by Sphinx.
\def\sphinxdocclass{report}
\documentclass[letterpaper,10pt,english]{sphinxmanual}
\ifdefined\pdfpxdimen
   \let\sphinxpxdimen\pdfpxdimen\else\newdimen\sphinxpxdimen
\fi \sphinxpxdimen=.75bp\relax
\ifdefined\pdfimageresolution
    \pdfimageresolution= \numexpr \dimexpr1in\relax/\sphinxpxdimen\relax
\fi
%% let collapsible pdf bookmarks panel have high depth per default
\PassOptionsToPackage{bookmarksdepth=5}{hyperref}

\PassOptionsToPackage{warn}{textcomp}
\usepackage[utf8]{inputenc}
\ifdefined\DeclareUnicodeCharacter
% support both utf8 and utf8x syntaxes
  \ifdefined\DeclareUnicodeCharacterAsOptional
    \def\sphinxDUC#1{\DeclareUnicodeCharacter{"#1}}
  \else
    \let\sphinxDUC\DeclareUnicodeCharacter
  \fi
  \sphinxDUC{00A0}{\nobreakspace}
  \sphinxDUC{2500}{\sphinxunichar{2500}}
  \sphinxDUC{2502}{\sphinxunichar{2502}}
  \sphinxDUC{2514}{\sphinxunichar{2514}}
  \sphinxDUC{251C}{\sphinxunichar{251C}}
  \sphinxDUC{2572}{\textbackslash}
\fi
\usepackage{cmap}
\usepackage[T1]{fontenc}
\usepackage{amsmath,amssymb,amstext}
\usepackage{babel}



\usepackage{tgtermes}
\usepackage{tgheros}
\renewcommand{\ttdefault}{txtt}



\usepackage[Bjarne]{fncychap}
\usepackage{sphinx}

\fvset{fontsize=auto}
\usepackage{geometry}


% Include hyperref last.
\usepackage{hyperref}
% Fix anchor placement for figures with captions.
\usepackage{hypcap}% it must be loaded after hyperref.
% Set up styles of URL: it should be placed after hyperref.
\urlstyle{same}


\usepackage{sphinxmessages}




\title{Stellar System Creator}
\date{Dec 14, 2021}
\release{0.0.5.1}
\author{Selewirre Iskvary}
\newcommand{\sphinxlogo}{\vbox{}}
\renewcommand{\releasename}{Release}
\makeindex
\begin{document}

\pagestyle{empty}
\sphinxmaketitle
\pagestyle{plain}
\sphinxtableofcontents
\pagestyle{normal}
\phantomsection\label{\detokenize{index::doc}}



\chapter{Introduction}
\label{\detokenize{introduction:introduction}}\label{\detokenize{introduction::doc}}
\sphinxAtStartPar
The Solar System Creator is a python package that aims to ease the creation of realistic
stellar systems in sci\sphinxhyphen{}fi settings. With minimal input, the user is able to create stars, planets,
moons, asteroid regions and other celestial bodies, with accurate physical characteristics, declare their habitability,
extract physical characteristics and visualize them.


\chapter{Quantities}
\label{\detokenize{quantities/quantities:quantities}}\label{\detokenize{quantities/quantities::doc}}
\sphinxAtStartPar
Here, we will explore the various physical quantities found in this package.


\section{Material}
\label{\detokenize{quantities/material/material:material}}\label{\detokenize{quantities/material/material::doc}}

\subsection{Mass}
\label{\detokenize{quantities/material/mass:mass}}\label{\detokenize{quantities/material/mass::doc}}
\sphinxAtStartPar
Mass is the quantity of mater in a physical body.
In the context of this package, mass determines most of other physical characteristics.

\sphinxAtStartPar
Suggested (approximate) masses:
\begin{enumerate}
\sphinxsetlistlabels{\arabic}{enumi}{enumii}{}{.}%
\item {} 
\sphinxAtStartPar
For rocky planets: up to around 5 earth masses (Me)

\item {} 
\sphinxAtStartPar
For ice\sphinxhyphen{}giants: between 5 and 100 earth masses

\item {} 
\sphinxAtStartPar
For gas\sphinxhyphen{}giants: between 100 earth masses and 10 jupiter masses (Mj)

\item {} 
\sphinxAtStartPar
For long\sphinxhyphen{}lived, red stars: 0.081 and 0.5 solar masses (Ms)

\item {} 
\sphinxAtStartPar
For habitable stars: 0.6 to 1.4 solar masses

\item {} 
\sphinxAtStartPar
For short\sphinxhyphen{}live, big blue stars: 1.4 to 50 solar masses.

\end{enumerate}


\subsection{Density}
\label{\detokenize{quantities/material/density:density}}\label{\detokenize{quantities/material/density::doc}}

\section{Geometric}
\label{\detokenize{quantities/geometric/geometric:geometric}}\label{\detokenize{quantities/geometric/geometric::doc}}

\subsection{Radius}
\label{\detokenize{quantities/geometric/radius:radius}}\label{\detokenize{quantities/geometric/radius::doc}}
\sphinxAtStartPar
Radius is the variable that defines the size of celestial objects.
The suggested radius is determined by the \DUrole{xref,std,std-doc}{mass} of the object via various radius models.

\sphinxAtStartPar
Models used:
\begin{enumerate}
\sphinxsetlistlabels{\arabic}{enumi}{enumii}{}{.}%
\item {} 
\sphinxAtStartPar
For planetary models, see \sphinxurl{https://arxiv.org/pdf/0707.2895.pdf}.

\item {} 
\sphinxAtStartPar
For hot gas\sphinxhyphen{}giant models, see \sphinxurl{https://arxiv.org/pdf/1804.03075.pdf}.

\item {} 
\sphinxAtStartPar
For stellar models, see \sphinxurl{https://academic.oup.com/mnras/article/479/4/5491/5056185}.

\end{enumerate}


\subsection{Circumference}
\label{\detokenize{quantities/geometric/circumference:circumference}}\label{\detokenize{quantities/geometric/circumference::doc}}

\subsection{Surface Area}
\label{\detokenize{quantities/geometric/surface_area:surface-area}}\label{\detokenize{quantities/geometric/surface_area::doc}}\phantomsection\label{\detokenize{quantities/geometric/surface_area:id1}}

\subsection{Volume}
\label{\detokenize{quantities/geometric/volume:volume}}\label{\detokenize{quantities/geometric/volume::doc}}

\section{Surface}
\label{\detokenize{quantities/surface/surface:surface}}\label{\detokenize{quantities/surface/surface::doc}}

\subsection{Emission}
\label{\detokenize{quantities/surface/emission/emission:emission}}\label{\detokenize{quantities/surface/emission/emission::doc}}

\subsubsection{Albedo}
\label{\detokenize{quantities/surface/emission/albedo:albedo}}\label{\detokenize{quantities/surface/emission/albedo::doc}}

\subsubsection{Emissivity}
\label{\detokenize{quantities/surface/emission/emissivity:emissivity}}\label{\detokenize{quantities/surface/emission/emissivity::doc}}

\subsubsection{Heat Distribution}
\label{\detokenize{quantities/surface/emission/heat_distribution:heat-distribution}}\label{\detokenize{quantities/surface/emission/heat_distribution::doc}}

\subsubsection{Normalized Greenhouse}
\label{\detokenize{quantities/surface/emission/normalized_greenhouse:normalized-greenhouse}}\label{\detokenize{quantities/surface/emission/normalized_greenhouse::doc}}

\subsubsection{Incident Flux}
\label{\detokenize{quantities/surface/emission/incident_flux:incident-flux}}\label{\detokenize{quantities/surface/emission/incident_flux::doc}}

\subsubsection{Luminosity}
\label{\detokenize{quantities/surface/emission/luminosity:luminosity}}\label{\detokenize{quantities/surface/emission/luminosity::doc}}

\subsubsection{Temperature}
\label{\detokenize{quantities/surface/emission/temperature:temperature}}\label{\detokenize{quantities/surface/emission/temperature::doc}}

\subsection{Gravity}
\label{\detokenize{quantities/surface/gravity/gravity:gravity}}\label{\detokenize{quantities/surface/gravity/gravity::doc}}

\subsubsection{Surface Gravity}
\label{\detokenize{quantities/surface/gravity/surface_gravity:surface-gravity}}\label{\detokenize{quantities/surface/gravity/surface_gravity::doc}}

\subsubsection{Escape Velocity}
\label{\detokenize{quantities/surface/gravity/escape_velocity:escape-velocity}}\label{\detokenize{quantities/surface/gravity/escape_velocity::doc}}

\subsection{Internal Heating}
\label{\detokenize{quantities/surface/internal_heating/internal_heating:internal-heating}}\label{\detokenize{quantities/surface/internal_heating/internal_heating::doc}}

\subsubsection{Tectonic Activity}
\label{\detokenize{quantities/surface/internal_heating/tectonic_activity:tectonic-activity}}\label{\detokenize{quantities/surface/internal_heating/tectonic_activity::doc}}

\subsubsection{Primordial Heating}
\label{\detokenize{quantities/surface/internal_heating/primordial_heating:primordial-heating}}\label{\detokenize{quantities/surface/internal_heating/primordial_heating::doc}}

\subsubsection{Radiogenic Heating}
\label{\detokenize{quantities/surface/internal_heating/radiogenic_heating:radiogenic-heating}}\label{\detokenize{quantities/surface/internal_heating/radiogenic_heating::doc}}

\subsubsection{Tidal Heating}
\label{\detokenize{quantities/surface/internal_heating/tidal_heating:tidal-heating}}\label{\detokenize{quantities/surface/internal_heating/tidal_heating::doc}}

\subsection{Induced Tide}
\label{\detokenize{quantities/surface/induced_tide:induced-tide}}\label{\detokenize{quantities/surface/induced_tide::doc}}

\section{Rotational}
\label{\detokenize{quantities/rotational/rotational:rotational}}\label{\detokenize{quantities/rotational/rotational::doc}}

\subsection{Spin Period}
\label{\detokenize{quantities/rotational/spin_period:spin-period}}\label{\detokenize{quantities/rotational/spin_period::doc}}

\subsection{Day Period}
\label{\detokenize{quantities/rotational/day_period:day-period}}\label{\detokenize{quantities/rotational/day_period::doc}}

\subsection{Axial Tilt}
\label{\detokenize{quantities/rotational/axial_tilt:axial-tilt}}\label{\detokenize{quantities/rotational/axial_tilt::doc}}

\chapter{Indices and tables}
\label{\detokenize{index:indices-and-tables}}\begin{itemize}
\item {} 
\sphinxAtStartPar
\DUrole{xref,std,std-ref}{genindex}

\item {} 
\sphinxAtStartPar
\DUrole{xref,std,std-ref}{modindex}

\item {} 
\sphinxAtStartPar
\DUrole{xref,std,std-ref}{search}

\end{itemize}



\renewcommand{\indexname}{Index}
\printindex
\end{document}
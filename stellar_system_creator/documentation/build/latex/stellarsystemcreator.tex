%% Generated by Sphinx.
\def\sphinxdocclass{report}
\documentclass[letterpaper,10pt,english]{sphinxmanual}
\ifdefined\pdfpxdimen
   \let\sphinxpxdimen\pdfpxdimen\else\newdimen\sphinxpxdimen
\fi \sphinxpxdimen=.75bp\relax
\ifdefined\pdfimageresolution
    \pdfimageresolution= \numexpr \dimexpr1in\relax/\sphinxpxdimen\relax
\fi
%% let collapsible pdf bookmarks panel have high depth per default
\PassOptionsToPackage{bookmarksdepth=5}{hyperref}

\PassOptionsToPackage{warn}{textcomp}
\usepackage[utf8]{inputenc}
\ifdefined\DeclareUnicodeCharacter
% support both utf8 and utf8x syntaxes
  \ifdefined\DeclareUnicodeCharacterAsOptional
    \def\sphinxDUC#1{\DeclareUnicodeCharacter{"#1}}
  \else
    \let\sphinxDUC\DeclareUnicodeCharacter
  \fi
  \sphinxDUC{00A0}{\nobreakspace}
  \sphinxDUC{2500}{\sphinxunichar{2500}}
  \sphinxDUC{2502}{\sphinxunichar{2502}}
  \sphinxDUC{2514}{\sphinxunichar{2514}}
  \sphinxDUC{251C}{\sphinxunichar{251C}}
  \sphinxDUC{2572}{\textbackslash}
\fi
\usepackage{cmap}
\usepackage[T1]{fontenc}
\usepackage{amsmath,amssymb,amstext}
\usepackage{babel}



\usepackage{tgtermes}
\usepackage{tgheros}
\renewcommand{\ttdefault}{txtt}



\usepackage[Bjarne]{fncychap}
\usepackage{sphinx}

\fvset{fontsize=auto}
\usepackage{geometry}


% Include hyperref last.
\usepackage{hyperref}
% Fix anchor placement for figures with captions.
\usepackage{hypcap}% it must be loaded after hyperref.
% Set up styles of URL: it should be placed after hyperref.
\urlstyle{same}


\usepackage{sphinxmessages}




\title{Stellar System Creator}
\date{Dec 19, 2021}
\release{0.0.5.1}
\author{Selewirre Iskvary}
\newcommand{\sphinxlogo}{\vbox{}}
\renewcommand{\releasename}{Release}
\makeindex
\begin{document}

\pagestyle{empty}
\sphinxmaketitle
\pagestyle{plain}
\sphinxtableofcontents
\pagestyle{normal}
\phantomsection\label{\detokenize{index::doc}}



\chapter{Introduction}
\label{\detokenize{introduction:introduction}}\label{\detokenize{introduction::doc}}
\sphinxAtStartPar
The Solar System Creator is a python package that aims to ease the creation of realistic
stellar systems in sci\sphinxhyphen{}fi settings. With minimal input, the user is able to create stars, planets,
moons, asteroid regions and other celestial bodies, with accurate physical characteristics, declare their habitability,
extract physical characteristics and visualize them.


\chapter{Quantities}
\label{\detokenize{quantities/quantities:quantities}}\label{\detokenize{quantities/quantities::doc}}
\sphinxAtStartPar
Here, we will explore the various physical quantities found in this package.


\section{Material}
\label{\detokenize{quantities/material/material:material}}\label{\detokenize{quantities/material/material::doc}}

\subsection{Mass}
\label{\detokenize{quantities/material/mass:mass}}\label{\detokenize{quantities/material/mass::doc}}\phantomsection\label{\detokenize{quantities/material/mass:id1}}
\sphinxAtStartPar
Mass is the quantity of mater in a physical body.
In the context of this package, mass determines most of other physical characteristics,
like {\hyperref[\detokenize{quantities/geometric/radius:id1}]{\sphinxcrossref{\DUrole{std,std-ref}{radius}}}}, {\hyperref[\detokenize{quantities/surface/emission/luminosity:id1}]{\sphinxcrossref{\DUrole{std,std-ref}{luminosity}}}}, {\hyperref[\detokenize{quantities/rotational/spin_period:id1}]{\sphinxcrossref{\DUrole{std,std-ref}{spin period}}}} and
{\hyperref[\detokenize{quantities/life/lifetime:id1}]{\sphinxcrossref{\DUrole{std,std-ref}{lifetime}}}}.

\sphinxAtStartPar
Suggested (approximate) masses:
\begin{enumerate}
\sphinxsetlistlabels{\arabic}{enumi}{enumii}{}{.}%
\item {} 
\sphinxAtStartPar
For mon\sphinxhyphen{}like satellites, less than 0.05 earth masses (Me)

\item {} 
\sphinxAtStartPar
For rocky planets: up to around 5 earth masses

\item {} 
\sphinxAtStartPar
For ice\sphinxhyphen{}giants: between 5 and 100 earth masses

\item {} 
\sphinxAtStartPar
For gas\sphinxhyphen{}giants: between 100 earth masses and 10 jupiter masses (Mj)

\item {} 
\sphinxAtStartPar
For long\sphinxhyphen{}lived, red stars: 0.081 and 0.5 solar masses (Ms)

\item {} 
\sphinxAtStartPar
For habitable stars: 0.6 to 1.4 solar masses

\item {} 
\sphinxAtStartPar
For short\sphinxhyphen{}live, big blue stars: 1.4 to 50 solar masses.

\end{enumerate}


\subsection{Density}
\label{\detokenize{quantities/material/density:density}}\label{\detokenize{quantities/material/density::doc}}\phantomsection\label{\detokenize{quantities/material/density:id1}}
\sphinxAtStartPar
Density (\(\rho = \frac{M}{V}\)) is the {\hyperref[\detokenize{quantities/material/mass:id1}]{\sphinxcrossref{\DUrole{std,std-ref}{mass}}}} per unit {\hyperref[\detokenize{quantities/geometric/volume:id1}]{\sphinxcrossref{\DUrole{std,std-ref}{volume}}}} of an a substance (or celestial object).
Usual densities in the solar system are between 0.5 and 7 grams/cm\textasciicircum{}3.


\subsection{Composition Type}
\label{\detokenize{quantities/material/composition_type:composition-type}}\label{\detokenize{quantities/material/composition_type::doc}}\phantomsection\label{\detokenize{quantities/material/composition_type:id1}}
\sphinxAtStartPar
The composition type of planets, planetoids, asteroids etc. is
what the approximate composition of a celestial object will be.
There are two types of iron worlds, two types of rocky worlds,
four types of water worlds, one type of ice\sphinxhyphen{} and one of gas\sphinxhyphen{}giants.

\sphinxAtStartPar
Even though the composition is not accurate by itself, I find
that the {\hyperref[\detokenize{quantities/material/density:id1}]{\sphinxcrossref{\DUrole{std,std-ref}{density}}}} and {\hyperref[\detokenize{quantities/geometric/radius:id1}]{\sphinxcrossref{\DUrole{std,std-ref}{radius}}}} of asteroids and moons that are not rocky,
can be generally approximated by the four different water worlds.
Same for small gassy worlds (like Pluto).


\subsection{Chemical Composition}
\label{\detokenize{quantities/material/chemical_composition:chemical-composition}}\label{\detokenize{quantities/material/chemical_composition::doc}}\phantomsection\label{\detokenize{quantities/material/chemical_composition:id1}}
\sphinxAtStartPar
Chemical composition is the ratio of different chemical compounds that constitute a substance.
There are 3 main substances that are portraited in this package.
That does not mean that there can not be other, it is just what the planetary {\hyperref[\detokenize{quantities/geometric/radius:id1}]{\sphinxcrossref{\DUrole{std,std-ref}{radius}}}} models represent.

\sphinxAtStartPar
The main chemical compounds are iron (Fe), rock (MgSiO3), water (H2O),
helium (He), Hydrogen (H2) and methane (CH4).


\section{Geometric}
\label{\detokenize{quantities/geometric/geometric:geometric}}\label{\detokenize{quantities/geometric/geometric::doc}}

\subsection{Radius}
\label{\detokenize{quantities/geometric/radius:radius}}\label{\detokenize{quantities/geometric/radius::doc}}\phantomsection\label{\detokenize{quantities/geometric/radius:id1}}
\sphinxAtStartPar
Radius is the variable that defines the size of celestial objects.
The radius determines the {\hyperref[\detokenize{quantities/geometric/circumference:id1}]{\sphinxcrossref{\DUrole{std,std-ref}{circumference}}}}, {\hyperref[\detokenize{quantities/geometric/surface_area:id1}]{\sphinxcrossref{\DUrole{std,std-ref}{surface area}}}}, {\hyperref[\detokenize{quantities/geometric/volume:id1}]{\sphinxcrossref{\DUrole{std,std-ref}{volume}}}} and {\hyperref[\detokenize{quantities/material/density:id1}]{\sphinxcrossref{\DUrole{std,std-ref}{density}}}}.
among other characteristics.
The suggested radius is determined by the {\hyperref[\detokenize{quantities/material/mass:id1}]{\sphinxcrossref{\DUrole{std,std-ref}{mass}}}} of the
object via various radius models. Use values \(\pm 8\) \% around
the suggested value.

\sphinxAtStartPar
Models used:
\begin{enumerate}
\sphinxsetlistlabels{\arabic}{enumi}{enumii}{}{.}%
\item {} 
\sphinxAtStartPar
For planetary models, see \sphinxurl{https://arxiv.org/pdf/0707.2895.pdf}.

\item {} 
\sphinxAtStartPar
For hot gas\sphinxhyphen{}giant models, see \sphinxurl{https://arxiv.org/pdf/1804.03075.pdf}.

\item {} 
\sphinxAtStartPar
For star models, see \sphinxurl{https://academic.oup.com/mnras/article/479/4/5491/5056185}.

\end{enumerate}


\subsection{Circumference}
\label{\detokenize{quantities/geometric/circumference:circumference}}\label{\detokenize{quantities/geometric/circumference::doc}}\phantomsection\label{\detokenize{quantities/geometric/circumference:id1}}
\sphinxAtStartPar
The circumference is determined by the radius \(C = 2 \pi r\).


\subsection{Surface Area}
\label{\detokenize{quantities/geometric/surface_area:surface-area}}\label{\detokenize{quantities/geometric/surface_area::doc}}\phantomsection\label{\detokenize{quantities/geometric/surface_area:id1}}
\sphinxAtStartPar
The surface area is determined by the radius \(A = 4 \pi r^2\).


\subsection{Volume}
\label{\detokenize{quantities/geometric/volume:volume}}\label{\detokenize{quantities/geometric/volume::doc}}\phantomsection\label{\detokenize{quantities/geometric/volume:id1}}
\sphinxAtStartPar
The volume is determined by the radius \(V = \frac{4\pi}{3}r^3\).


\section{Rotational}
\label{\detokenize{quantities/rotational/rotational:rotational}}\label{\detokenize{quantities/rotational/rotational::doc}}

\subsection{Spin Period}
\label{\detokenize{quantities/rotational/spin_period:spin-period}}\label{\detokenize{quantities/rotational/spin_period::doc}}\phantomsection\label{\detokenize{quantities/rotational/spin_period:id1}}
\sphinxAtStartPar
The spin period is the amount of time it takes for a celestial body to
rotate around itself compared to the distant stars.

\sphinxAtStartPar
Planetary spin period is determined by the {\hyperref[\detokenize{quantities/material/mass:id1}]{\sphinxcrossref{\DUrole{std,std-ref}{mass}}}} and {\hyperref[\detokenize{quantities/geometric/radius:id1}]{\sphinxcrossref{\DUrole{std,std-ref}{radius}}}} of the celestial body.
The more massive the body, the faster it rotates.
If there are satellites around the planet large enough
(e.g. earth\sphinxhyphen{}moon), there is a substantial transfer of
angular momentum between the two bodies, making the planet
slow down (earth spin period would have been around 16 hr).


\subsection{Day Period}
\label{\detokenize{quantities/rotational/day_period:day-period}}\label{\detokenize{quantities/rotational/day_period::doc}}\phantomsection\label{\detokenize{quantities/rotational/day_period:id1}}
\sphinxAtStartPar
The day period of a child body is determined by the {\hyperref[\detokenize{quantities/rotational/spin_period:id1}]{\sphinxcrossref{\DUrole{std,std-ref}{spin period}}}}
and the {\hyperref[\detokenize{quantities/orbital/orbital_period:id1}]{\sphinxcrossref{\DUrole{std,std-ref}{orbital period}}}} around the parent body.


\subsection{Axial Tilt}
\label{\detokenize{quantities/rotational/axial_tilt:axial-tilt}}\label{\detokenize{quantities/rotational/axial_tilt::doc}}\phantomsection\label{\detokenize{quantities/rotational/axial_tilt:id1}}
\sphinxAtStartPar
The axial tilt of a child body, also known as obliquity, is
the angle between an object’s rotational axis and its orbital axis.

\sphinxAtStartPar
As of now, it is cosmetic and does not determine any other characteristics.


\section{Life}
\label{\detokenize{quantities/life/life:life}}\label{\detokenize{quantities/life/life::doc}}

\subsection{Age}
\label{\detokenize{quantities/life/age:age}}\label{\detokenize{quantities/life/age::doc}}\phantomsection\label{\detokenize{quantities/life/age:id1}}
\sphinxAtStartPar
The suggested age of a star is set to be half of its {\hyperref[\detokenize{quantities/life/lifetime:id1}]{\sphinxcrossref{\DUrole{std,std-ref}{lifetime}}}}.
The suggested age of any other object is determined by it’s
parent age.


\subsection{Lifetime}
\label{\detokenize{quantities/life/lifetime:lifetime}}\label{\detokenize{quantities/life/lifetime::doc}}\phantomsection\label{\detokenize{quantities/life/lifetime:id1}}
\sphinxAtStartPar
The lifetime of stars is determined by its mass and its luminosity (\(T=\frac{M}{L}\)).

\sphinxAtStartPar
The lifetime of each other body is determined by the lifetime of the parent
minus a hundred million years, which is roughly the amount of time it takes
for planets to form around stars (or satellites to be captured). It is by
no means binding.


\section{Surface}
\label{\detokenize{quantities/surface/surface:surface}}\label{\detokenize{quantities/surface/surface::doc}}

\subsection{Emission}
\label{\detokenize{quantities/surface/emission/emission:emission}}\label{\detokenize{quantities/surface/emission/emission::doc}}

\subsubsection{Albedo}
\label{\detokenize{quantities/surface/emission/albedo:albedo}}\label{\detokenize{quantities/surface/emission/albedo::doc}}

\subsubsection{Emissivity}
\label{\detokenize{quantities/surface/emission/emissivity:emissivity}}\label{\detokenize{quantities/surface/emission/emissivity::doc}}

\subsubsection{Heat Distribution}
\label{\detokenize{quantities/surface/emission/heat_distribution:heat-distribution}}\label{\detokenize{quantities/surface/emission/heat_distribution::doc}}

\subsubsection{Normalized Greenhouse}
\label{\detokenize{quantities/surface/emission/normalized_greenhouse:normalized-greenhouse}}\label{\detokenize{quantities/surface/emission/normalized_greenhouse::doc}}

\subsubsection{Incident Flux}
\label{\detokenize{quantities/surface/emission/incident_flux:incident-flux}}\label{\detokenize{quantities/surface/emission/incident_flux::doc}}

\subsubsection{Temperature}
\label{\detokenize{quantities/surface/emission/temperature:temperature}}\label{\detokenize{quantities/surface/emission/temperature::doc}}

\subsubsection{Luminosity}
\label{\detokenize{quantities/surface/emission/luminosity:luminosity}}\label{\detokenize{quantities/surface/emission/luminosity::doc}}\phantomsection\label{\detokenize{quantities/surface/emission/luminosity:id1}}

\subsubsection{Peak Wavelength}
\label{\detokenize{quantities/surface/emission/peak_wavelength:peak-wavelength}}\label{\detokenize{quantities/surface/emission/peak_wavelength::doc}}

\subsection{Gravity}
\label{\detokenize{quantities/surface/gravity/gravity:gravity}}\label{\detokenize{quantities/surface/gravity/gravity::doc}}

\subsubsection{Surface Gravity}
\label{\detokenize{quantities/surface/gravity/surface_gravity:surface-gravity}}\label{\detokenize{quantities/surface/gravity/surface_gravity::doc}}

\subsubsection{Escape Velocity}
\label{\detokenize{quantities/surface/gravity/escape_velocity:escape-velocity}}\label{\detokenize{quantities/surface/gravity/escape_velocity::doc}}

\subsection{Internal Heating}
\label{\detokenize{quantities/surface/internal_heating/internal_heating:internal-heating}}\label{\detokenize{quantities/surface/internal_heating/internal_heating::doc}}

\subsubsection{Tectonic Activity}
\label{\detokenize{quantities/surface/internal_heating/tectonic_activity:tectonic-activity}}\label{\detokenize{quantities/surface/internal_heating/tectonic_activity::doc}}

\subsubsection{Primordial Heating}
\label{\detokenize{quantities/surface/internal_heating/primordial_heating:primordial-heating}}\label{\detokenize{quantities/surface/internal_heating/primordial_heating::doc}}

\subsubsection{Radiogenic Heating}
\label{\detokenize{quantities/surface/internal_heating/radiogenic_heating:radiogenic-heating}}\label{\detokenize{quantities/surface/internal_heating/radiogenic_heating::doc}}

\subsubsection{Tidal Heating}
\label{\detokenize{quantities/surface/internal_heating/tidal_heating:tidal-heating}}\label{\detokenize{quantities/surface/internal_heating/tidal_heating::doc}}

\subsection{Induced Tide}
\label{\detokenize{quantities/surface/induced_tide:induced-tide}}\label{\detokenize{quantities/surface/induced_tide::doc}}

\subsection{Angular Diameter}
\label{\detokenize{quantities/surface/angular_diameter:angular-diameter}}\label{\detokenize{quantities/surface/angular_diameter::doc}}

\section{Orbital}
\label{\detokenize{quantities/orbital/orbital:orbital}}\label{\detokenize{quantities/orbital/orbital::doc}}

\subsection{Eccentricity}
\label{\detokenize{quantities/orbital/eccentricity:eccentricity}}\label{\detokenize{quantities/orbital/eccentricity::doc}}\phantomsection\label{\detokenize{quantities/orbital/eccentricity:id1}}
\sphinxAtStartPar
Eccentricity \(e\) determines how elliptic the orbit of a child around a parent body is.
\(e = 0\) means that the orbit is circular, and \(e = 1\) means
that the orbit resembles a line (not a stable orbit).


\subsection{Semi\sphinxhyphen{}Major Axis}
\label{\detokenize{quantities/orbital/semi_major_axis:semi-major-axis}}\label{\detokenize{quantities/orbital/semi_major_axis::doc}}\phantomsection\label{\detokenize{quantities/orbital/semi_major_axis:id1}}
\sphinxAtStartPar
Semi\sphinxhyphen{}major axis \(a\) is the mean distance between a child and a parent body.


\subsection{Semi\sphinxhyphen{}Minor Axis}
\label{\detokenize{quantities/orbital/semi_minor_axis:semi-minor-axis}}\label{\detokenize{quantities/orbital/semi_minor_axis::doc}}\phantomsection\label{\detokenize{quantities/orbital/semi_minor_axis:id1}}
\sphinxAtStartPar
Semi\sphinxhyphen{}minor axis \(b\) is determined by the {\hyperref[\detokenize{quantities/orbital/semi_major_axis:id1}]{\sphinxcrossref{\DUrole{std,std-ref}{semi\sphinxhyphen{}major axis}}}} \(a\) and the
{\hyperref[\detokenize{quantities/orbital/eccentricity:id1}]{\sphinxcrossref{\DUrole{std,std-ref}{eccentricity}}}} \(e\): \(b = a \sqrt{1 - e^2}\).


\subsection{Apoapsis}
\label{\detokenize{quantities/orbital/apoapsis:apoapsis}}\label{\detokenize{quantities/orbital/apoapsis::doc}}\phantomsection\label{\detokenize{quantities/orbital/apoapsis:id1}}
\sphinxAtStartPar
Apoapsis (\(a (1 + e)\)) is the furthest distance between a child and a parent body.
It is determined by the {\hyperref[\detokenize{quantities/orbital/semi_major_axis:id1}]{\sphinxcrossref{\DUrole{std,std-ref}{semi\sphinxhyphen{}major axis}}}} \(a\) and the
{\hyperref[\detokenize{quantities/orbital/eccentricity:id1}]{\sphinxcrossref{\DUrole{std,std-ref}{eccentricity}}}} \(e\).


\subsection{Periapsis}
\label{\detokenize{quantities/orbital/periapsis:periapsis}}\label{\detokenize{quantities/orbital/periapsis::doc}}\phantomsection\label{\detokenize{quantities/orbital/periapsis:id1}}
\sphinxAtStartPar
Periapsis (\(a (1 - e)\)) is the nearest distance between a child and a parent body.
It is determined by the {\hyperref[\detokenize{quantities/orbital/semi_major_axis:id1}]{\sphinxcrossref{\DUrole{std,std-ref}{semi\sphinxhyphen{}major axis}}}} \(a\) and the
{\hyperref[\detokenize{quantities/orbital/eccentricity:id1}]{\sphinxcrossref{\DUrole{std,std-ref}{eccentricity}}}} \(e\).


\subsection{Lagrange Position}
\label{\detokenize{quantities/orbital/lagrange_position:lagrange-position}}\label{\detokenize{quantities/orbital/lagrange_position::doc}}\phantomsection\label{\detokenize{quantities/orbital/lagrange_position:id1}}
\sphinxAtStartPar
Lagrange positions (L\#) are (semi\sphinxhyphen{})stable positions between two orbiting objects.
Trojans are objects in the L4 and L5 positions, in front and behind the child
object (denoted as +1 and \sphinxhyphen{}1 respectively) that is orbiting a parent object.


\subsection{Contact}
\label{\detokenize{quantities/orbital/contact:contact}}\label{\detokenize{quantities/orbital/contact::doc}}\phantomsection\label{\detokenize{quantities/orbital/contact:id1}}
\sphinxAtStartPar
In this context, contact between two binary stars happens if the radius of at least
one of the two stars resides outside the roche lobe while the stars are in {\hyperref[\detokenize{quantities/orbital/periapsis:id1}]{\sphinxcrossref{\DUrole{std,std-ref}{periapsis}}}}.


\subsection{Roche Lobe}
\label{\detokenize{quantities/orbital/roche_lobe:roche-lobe}}\label{\detokenize{quantities/orbital/roche_lobe::doc}}\phantomsection\label{\detokenize{quantities/orbital/roche_lobe:id1}}
\sphinxAtStartPar
The Roche lobe is the region around a star in a binary system within which orbiting material is gravitationally bound
to that star.


\subsection{Orbital Period}
\label{\detokenize{quantities/orbital/orbital_period:orbital-period}}\label{\detokenize{quantities/orbital/orbital_period::doc}}\phantomsection\label{\detokenize{quantities/orbital/orbital_period:id1}}
\sphinxAtStartPar
Orbital period (\(\sqrt{\frac{a^3}{M_{tot}}}\)) is the time it takes for a child body to orbit around their parent.
It is given by the {\hyperref[\detokenize{quantities/orbital/semi_major_axis:id1}]{\sphinxcrossref{\DUrole{std,std-ref}{semi\sphinxhyphen{}major axis}}}} \(a\), and the total mass \(M_{tot}\) of child
and parent objects.


\subsection{Orbital Velocity}
\label{\detokenize{quantities/orbital/orbital_velocity:orbital-velocity}}\label{\detokenize{quantities/orbital/orbital_velocity::doc}}\phantomsection\label{\detokenize{quantities/orbital/orbital_velocity:id1}}
\sphinxAtStartPar
Orbital velocity (\(\sqrt{\frac{M_{tot}}{a}}\)) is the mean speed at which the child body
travels around the parent body. It is given by the {\hyperref[\detokenize{quantities/orbital/semi_major_axis:id1}]{\sphinxcrossref{\DUrole{std,std-ref}{semi\sphinxhyphen{}major axis}}}} \(a\), and the
total mass \(M_{tot}\) of child and parent objects.


\subsection{Orbit Type}
\label{\detokenize{quantities/orbital/orbit_type:orbit-type}}\label{\detokenize{quantities/orbital/orbit_type::doc}}\phantomsection\label{\detokenize{quantities/orbital/orbit_type:id1}}
\sphinxAtStartPar
Orbital type of a child object can be either prograde or retrograde \sphinxhyphen{} orbiting
along or against the rotation of the parent.


\subsection{Orbit Type Factor}
\label{\detokenize{quantities/orbital/orbit_type_factor:orbit-type-factor}}\label{\detokenize{quantities/orbital/orbit_type_factor::doc}}\phantomsection\label{\detokenize{quantities/orbital/orbit_type_factor:id1}}
\sphinxAtStartPar
Orbit type factor determines the {\hyperref[\detokenize{quantities/orbital/semi_major_axis_maximum_limit:id1}]{\sphinxcrossref{\DUrole{std,std-ref}{semi\sphinxhyphen{}major axis maximum limit}}}}.
It depends on the {\hyperref[\detokenize{quantities/orbital/orbit_type:id1}]{\sphinxcrossref{\DUrole{std,std-ref}{orbit type}}}} and is determined by the parent {\hyperref[\detokenize{quantities/orbital/eccentricity:id1}]{\sphinxcrossref{\DUrole{std,std-ref}{eccentricity}}}} \(e_p\)
and the child {\hyperref[\detokenize{quantities/orbital/eccentricity:id1}]{\sphinxcrossref{\DUrole{std,std-ref}{eccentricity}}}} \(e_c\).
\begin{enumerate}
\sphinxsetlistlabels{\arabic}{enumi}{enumii}{}{.}%
\item {} 
\sphinxAtStartPar
For a prograde orbit, it is given by \(0.4895 (1 - 1.0305 e_p - 0.2738 e_c)\).

\item {} 
\sphinxAtStartPar
For a retrograde orbit, it is given by \(0.9309 (1 - 1.0764 e_p - 0.9812 e_c)\).

\end{enumerate}


\subsection{Orbital Stability}
\label{\detokenize{quantities/orbital/orbital_stability:orbital-stability}}\label{\detokenize{quantities/orbital/orbital_stability::doc}}\phantomsection\label{\detokenize{quantities/orbital/orbital_stability:id1}}
\sphinxAtStartPar
Orbital stability demonstrates if the orbit of the child object around the parent object
does not exceed any limits. More specifically, for a stable orbit we must have:
\begin{enumerate}
\sphinxsetlistlabels{\arabic}{enumi}{enumii}{}{.}%
\item {} 
\sphinxAtStartPar
{\hyperref[\detokenize{quantities/orbital/periapsis:id1}]{\sphinxcrossref{\DUrole{std,std-ref}{Periapsis}}}} \textgreater{} {\hyperref[\detokenize{quantities/children_orbit_limits/roche_limit:id1}]{\sphinxcrossref{\DUrole{std,std-ref}{roche limit}}}}.

\item {} 
\sphinxAtStartPar
{\hyperref[\detokenize{quantities/orbital/semi_major_axis:id1}]{\sphinxcrossref{\DUrole{std,std-ref}{Semi\sphinxhyphen{}major axis}}}} \textgreater{} {\hyperref[\detokenize{quantities/children_orbit_limits/p_type_critical_orbit:p-type-critical-orbit}]{\sphinxcrossref{\DUrole{std,std-ref}{p\sphinxhyphen{}type critical orbit}}}}.

\item {} 
\sphinxAtStartPar
{\hyperref[\detokenize{quantities/orbital/apoapsis:id1}]{\sphinxcrossref{\DUrole{std,std-ref}{Apoapsis}}}} \textless{} {\hyperref[\detokenize{quantities/orbital/semi_major_axis_maximum_limit:id1}]{\sphinxcrossref{\DUrole{std,std-ref}{semi\sphinxhyphen{}major axis maximum limit}}}}.

\item {} 
\sphinxAtStartPar
Optional: {\hyperref[\detokenize{quantities/orbital/semi_major_axis:id1}]{\sphinxcrossref{\DUrole{std,std-ref}{Semi\sphinxhyphen{}major axis}}}} \textgreater{} {\hyperref[\detokenize{quantities/children_orbit_limits/inner_orbit_limit:id1}]{\sphinxcrossref{\DUrole{std,std-ref}{parent inner orbit limit}}}}.

\item {} 
\sphinxAtStartPar
For rock worlds: {\hyperref[\detokenize{quantities/orbital/semi_major_axis:id1}]{\sphinxcrossref{\DUrole{std,std-ref}{Semi\sphinxhyphen{}major axis}}}} \textgreater{} {\hyperref[\detokenize{quantities/children_orbit_limits/outer_rock_formation_limit:id1}]{\sphinxcrossref{\DUrole{std,std-ref}{parent outer rock formation limit}}}}.

\end{enumerate}


\subsection{Inclination}
\label{\detokenize{quantities/orbital/inclination:inclination}}\label{\detokenize{quantities/orbital/inclination::doc}}\phantomsection\label{\detokenize{quantities/orbital/inclination:id1}}
\sphinxAtStartPar
Orbital inclination is the tilt of a child object’s orbit around a celestial body.
It varies between 0 and 180. Between 0 and 90, the orbit is prograde.
Between 90 and 180, the orbit is retrograde.

\sphinxAtStartPar
However, as of now, it is cosmetic and does not determine any other characteristics,
including the {\hyperref[\detokenize{quantities/orbital/orbit_type:id1}]{\sphinxcrossref{\DUrole{std,std-ref}{orbit type}}}}.


\subsection{Argument of Periapsis}
\label{\detokenize{quantities/orbital/argument_of_periapsis:argument-of-periapsis}}\label{\detokenize{quantities/orbital/argument_of_periapsis::doc}}\phantomsection\label{\detokenize{quantities/orbital/argument_of_periapsis:id1}}
\sphinxAtStartPar
Check out \sphinxurl{https://en.wikipedia.org/wiki/Argument\_of\_periapsis}.

\sphinxAtStartPar
The argument of periapsis, as of now, it is cosmetic and does not determine any other characteristics.


\subsection{Longitude of the Ascending Node}
\label{\detokenize{quantities/orbital/longitude_of_the_ascending_node:longitude-of-the-ascending-node}}\label{\detokenize{quantities/orbital/longitude_of_the_ascending_node::doc}}\phantomsection\label{\detokenize{quantities/orbital/longitude_of_the_ascending_node:id1}}
\sphinxAtStartPar
Check out \sphinxurl{https://en.wikipedia.org/wiki/Longitude\_of\_the\_ascending\_node}.

\sphinxAtStartPar
The longitude of the ascending node, as of now, it is cosmetic and does
not determine any other characteristics.


\subsection{Semi\sphinxhyphen{}Major Axis Minimum Limit}
\label{\detokenize{quantities/orbital/semi_major_axis_minimum_limit:semi-major-axis-minimum-limit}}\label{\detokenize{quantities/orbital/semi_major_axis_minimum_limit::doc}}\phantomsection\label{\detokenize{quantities/orbital/semi_major_axis_minimum_limit:id1}}
\sphinxAtStartPar
The semi\sphinxhyphen{}major axis minimum limit is the {\hyperref[\detokenize{quantities/children_orbit_limits/roche_limit:id1}]{\sphinxcrossref{\DUrole{std,std-ref}{roche limit}}}} of the parent for the specific child
{\hyperref[\detokenize{quantities/material/density:id1}]{\sphinxcrossref{\DUrole{std,std-ref}{density}}}}, or the {\hyperref[\detokenize{quantities/children_orbit_limits/p_type_critical_orbit:p-type-critical-orbit}]{\sphinxcrossref{\DUrole{std,std-ref}{p\sphinxhyphen{}type critical limit}}}} in binary systems.


\subsection{Semi\sphinxhyphen{}Major Axis Maximum Limit}
\label{\detokenize{quantities/orbital/semi_major_axis_maximum_limit:semi-major-axis-maximum-limit}}\label{\detokenize{quantities/orbital/semi_major_axis_maximum_limit::doc}}\phantomsection\label{\detokenize{quantities/orbital/semi_major_axis_maximum_limit:id1}}
\sphinxAtStartPar
The semi\sphinxhyphen{}major axis maximum limit is the {\hyperref[\detokenize{quantities/children_orbit_limits/hill_sphere:id1}]{\sphinxcrossref{\DUrole{std,std-ref}{hill\_sphere}}}} multiplied by the
{\hyperref[\detokenize{quantities/orbital/orbit_type_factor:id1}]{\sphinxcrossref{\DUrole{std,std-ref}{orbit type factor}}}}.


\section{Children Orbit Limits}
\label{\detokenize{quantities/children_orbit_limits/children_orbit_limits:children-orbit-limits}}\label{\detokenize{quantities/children_orbit_limits/children_orbit_limits::doc}}\phantomsection\label{\detokenize{quantities/children_orbit_limits/children_orbit_limits:id1}}

\subsection{Tidal Locking Radius}
\label{\detokenize{quantities/children_orbit_limits/tidal_locking_radius:tidal-locking-radius}}\label{\detokenize{quantities/children_orbit_limits/tidal_locking_radius::doc}}

\subsection{Roche Limit}
\label{\detokenize{quantities/children_orbit_limits/roche_limit:roche-limit}}\label{\detokenize{quantities/children_orbit_limits/roche_limit::doc}}\phantomsection\label{\detokenize{quantities/children_orbit_limits/roche_limit:id1}}

\subsection{Dense Roche Limit}
\label{\detokenize{quantities/children_orbit_limits/dense_roche_limit:dense-roche-limit}}\label{\detokenize{quantities/children_orbit_limits/dense_roche_limit::doc}}

\subsection{P\sphinxhyphen{}type binary Critical Orbit}
\label{\detokenize{quantities/children_orbit_limits/p_type_critical_orbit:p-type-binary-critical-orbit}}\label{\detokenize{quantities/children_orbit_limits/p_type_critical_orbit::doc}}\phantomsection\label{\detokenize{quantities/children_orbit_limits/p_type_critical_orbit:p-type-critical-orbit}}

\subsection{Inner Orbit Limit}
\label{\detokenize{quantities/children_orbit_limits/inner_orbit_limit:inner-orbit-limit}}\label{\detokenize{quantities/children_orbit_limits/inner_orbit_limit::doc}}\phantomsection\label{\detokenize{quantities/children_orbit_limits/inner_orbit_limit:id1}}

\subsection{Hill Sphere}
\label{\detokenize{quantities/children_orbit_limits/hill_sphere:hill-sphere}}\label{\detokenize{quantities/children_orbit_limits/hill_sphere::doc}}\phantomsection\label{\detokenize{quantities/children_orbit_limits/hill_sphere:id1}}
\sphinxAtStartPar
hill sphere or roche lobe


\subsection{S\sphinxhyphen{}type binary Critical Orbit}
\label{\detokenize{quantities/children_orbit_limits/s_type_critical_orbit:s-type-binary-critical-orbit}}\label{\detokenize{quantities/children_orbit_limits/s_type_critical_orbit::doc}}

\subsection{Outer Orbit Limit}
\label{\detokenize{quantities/children_orbit_limits/outer_orbit_limit:outer-orbit-limit}}\label{\detokenize{quantities/children_orbit_limits/outer_orbit_limit::doc}}

\subsection{Inner Rock Formation Limit}
\label{\detokenize{quantities/children_orbit_limits/inner_rock_formation_limit:inner-rock-formation-limit}}\label{\detokenize{quantities/children_orbit_limits/inner_rock_formation_limit::doc}}

\subsection{Outer Rock Formation Limit}
\label{\detokenize{quantities/children_orbit_limits/outer_rock_formation_limit:outer-rock-formation-limit}}\label{\detokenize{quantities/children_orbit_limits/outer_rock_formation_limit::doc}}\phantomsection\label{\detokenize{quantities/children_orbit_limits/outer_rock_formation_limit:id1}}

\subsection{Inner Water Frost Limit}
\label{\detokenize{quantities/children_orbit_limits/inner_water_frost_limit:inner-water-frost-limit}}\label{\detokenize{quantities/children_orbit_limits/inner_water_frost_limit::doc}}

\subsection{Sol\sphinxhyphen{}Equivalent Water Frost Limit}
\label{\detokenize{quantities/children_orbit_limits/sol_equivalent_water_frost_limit:sol-equivalent-water-frost-limit}}\label{\detokenize{quantities/children_orbit_limits/sol_equivalent_water_frost_limit::doc}}

\subsection{Outer Water Frost Limit}
\label{\detokenize{quantities/children_orbit_limits/outer_water_frost_limit:outer-water-frost-limit}}\label{\detokenize{quantities/children_orbit_limits/outer_water_frost_limit::doc}}

\section{Insolation Models}
\label{\detokenize{quantities/insolation_models/insolation_models:insolation-models}}\label{\detokenize{quantities/insolation_models/insolation_models::doc}}

\subsection{Kopparapu}
\label{\detokenize{quantities/insolation_models/kopparapu/kopparapu:kopparapu}}\label{\detokenize{quantities/insolation_models/kopparapu/kopparapu::doc}}

\subsection{Selsis}
\label{\detokenize{quantities/insolation_models/selsis/selsis:selsis}}\label{\detokenize{quantities/insolation_models/selsis/selsis::doc}}

\section{Habitability}
\label{\detokenize{quantities/habitability/habitability:habitability}}\label{\detokenize{quantities/habitability/habitability::doc}}

\chapter{Indices and tables}
\label{\detokenize{index:indices-and-tables}}\begin{itemize}
\item {} 
\sphinxAtStartPar
\DUrole{xref,std,std-ref}{genindex}

\item {} 
\sphinxAtStartPar
\DUrole{xref,std,std-ref}{modindex}

\item {} 
\sphinxAtStartPar
\DUrole{xref,std,std-ref}{search}

\end{itemize}



\renewcommand{\indexname}{Index}
\printindex
\end{document}
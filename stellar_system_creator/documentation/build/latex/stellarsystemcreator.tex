%% Generated by Sphinx.
\def\sphinxdocclass{report}
\documentclass[letterpaper,10pt,english]{sphinxmanual}
\ifdefined\pdfpxdimen
   \let\sphinxpxdimen\pdfpxdimen\else\newdimen\sphinxpxdimen
\fi \sphinxpxdimen=.75bp\relax
\ifdefined\pdfimageresolution
    \pdfimageresolution= \numexpr \dimexpr1in\relax/\sphinxpxdimen\relax
\fi
%% let collapsible pdf bookmarks panel have high depth per default
\PassOptionsToPackage{bookmarksdepth=5}{hyperref}

\PassOptionsToPackage{warn}{textcomp}
\usepackage[utf8]{inputenc}
\ifdefined\DeclareUnicodeCharacter
% support both utf8 and utf8x syntaxes
  \ifdefined\DeclareUnicodeCharacterAsOptional
    \def\sphinxDUC#1{\DeclareUnicodeCharacter{"#1}}
  \else
    \let\sphinxDUC\DeclareUnicodeCharacter
  \fi
  \sphinxDUC{00A0}{\nobreakspace}
  \sphinxDUC{2500}{\sphinxunichar{2500}}
  \sphinxDUC{2502}{\sphinxunichar{2502}}
  \sphinxDUC{2514}{\sphinxunichar{2514}}
  \sphinxDUC{251C}{\sphinxunichar{251C}}
  \sphinxDUC{2572}{\textbackslash}
\fi
\usepackage{cmap}
\usepackage[T1]{fontenc}
\usepackage{amsmath,amssymb,amstext}
\usepackage{babel}



\usepackage{tgtermes}
\usepackage{tgheros}
\renewcommand{\ttdefault}{txtt}



\usepackage[Bjarne]{fncychap}
\usepackage{sphinx}

\fvset{fontsize=auto}
\usepackage{geometry}


% Include hyperref last.
\usepackage{hyperref}
% Fix anchor placement for figures with captions.
\usepackage{hypcap}% it must be loaded after hyperref.
% Set up styles of URL: it should be placed after hyperref.
\urlstyle{same}


\usepackage{sphinxmessages}




\title{Stellar System Creator}
\date{Jan 03, 2022}
\release{0.1.1.0}
\author{Selewirre Iskvary}
\newcommand{\sphinxlogo}{\vbox{}}
\renewcommand{\releasename}{Release}
\makeindex
\begin{document}

\pagestyle{empty}
\sphinxmaketitle
\pagestyle{plain}
\sphinxtableofcontents
\pagestyle{normal}
\phantomsection\label{\detokenize{index::doc}}



\chapter{Introduction}
\label{\detokenize{introduction:introduction}}\label{\detokenize{introduction::doc}}
\sphinxAtStartPar
The Solar System Creator is a python package that aims to ease the creation of realistic
stellar systems in sci\sphinxhyphen{}fi settings. With minimal input, the user is able to create stars, planets,
moons, asteroid regions and other celestial bodies, with accurate physical characteristics, declare their habitability,
extract physical characteristics and visualize them.


\chapter{Celestial Bodies}
\label{\detokenize{celestial_bodies/celestial_bodies:celestial-bodies}}\label{\detokenize{celestial_bodies/celestial_bodies::doc}}\phantomsection\label{\detokenize{celestial_bodies/celestial_bodies:id1}}
\sphinxAtStartPar
Celestial bodies are different types of objects that constitute a stellar system.
They can be very massive, or very light. They can be puffy or very dense.
They can be bright or dim.


\section{Star}
\label{\detokenize{celestial_bodies/star:star}}\label{\detokenize{celestial_bodies/star::doc}}\phantomsection\label{\detokenize{celestial_bodies/star:id1}}
\sphinxAtStartPar
A star is a celestial object that is massive enough to sustain nuclear fission in it’s core.
In this package we will only work with main sequence stars.
Their {\hyperref[\detokenize{quantities/material/mass:id1}]{\sphinxcrossref{\DUrole{std,std-ref}{masses}}}} vary between \(0.08 \, {\rm M_s}\) and \(150 \, {\rm M_s}\).

\sphinxAtStartPar
The more massive the star, the bigger it’s {\hyperref[\detokenize{quantities/geometric/radius:id1}]{\sphinxcrossref{\DUrole{std,std-ref}{size}}}}, the more {\hyperref[\detokenize{quantities/surface/emission/luminosity:id1}]{\sphinxcrossref{\DUrole{std,std-ref}{luminous}}}} it is,
and the smallest its {\hyperref[\detokenize{quantities/life/lifetime:id1}]{\sphinxcrossref{\DUrole{std,std-ref}{lifetime}}}} will be.
Ideal for {\hyperref[\detokenize{quantities/habitability/habitability:id1}]{\sphinxcrossref{\DUrole{std,std-ref}{life}}}} stars, are stars with {\hyperref[\detokenize{quantities/material/mass:id1}]{\sphinxcrossref{\DUrole{std,std-ref}{mass}}}} between \(0.6 \, {\rm M_s}\)
and \(1.4 \, {\rm M_s}\).

\sphinxAtStartPar
Stars can form {\hyperref[\detokenize{celestial_systems/binary_system:id1}]{\sphinxcrossref{\DUrole{std,std-ref}{stellar binaries}}}} and host {\hyperref[\detokenize{celestial_bodies/planet:id1}]{\sphinxcrossref{\DUrole{std,std-ref}{planets}}}}
and {\hyperref[\detokenize{celestial_bodies/asteroid_belt:id1}]{\sphinxcrossref{\DUrole{std,std-ref}{asteroid belts}}}}, either together or separately from their companion star (or both).


\section{Planet}
\label{\detokenize{celestial_bodies/planet:planet}}\label{\detokenize{celestial_bodies/planet::doc}}\phantomsection\label{\detokenize{celestial_bodies/planet:id1}}
\sphinxAtStartPar
A planet is a celestial object that is {\hyperref[\detokenize{quantities/orbital/orbital:id1}]{\sphinxcrossref{\DUrole{std,std-ref}{orbiting}}}}
around a {\hyperref[\detokenize{celestial_bodies/star:id1}]{\sphinxcrossref{\DUrole{std,std-ref}{star}}}} or a {\hyperref[\detokenize{celestial_systems/binary_system:id1}]{\sphinxcrossref{\DUrole{std,std-ref}{binary stellar system}}}}
and is massive enough to be rounded due to its own gravity, but not massive
enough to sustain nuclear fission in it’s core.

\sphinxAtStartPar
In this package we will work with planets of a few distinct, but representative
{\hyperref[\detokenize{quantities/material/composition_type:id1}]{\sphinxcrossref{\DUrole{std,std-ref}{composition types}}}}, that fall under four categories:
iron worlds, rocky worlds, water worlds, and ice/gas giants. Rocky and water
worlds are ideal for life, either on the land,
in the surface/underground ocean, or both.

\sphinxAtStartPar
Planets can host {\hyperref[\detokenize{celestial_bodies/satellite:id1}]{\sphinxcrossref{\DUrole{std,std-ref}{satellites}}}}, {\hyperref[\detokenize{celestial_bodies/trojan:id1}]{\sphinxcrossref{\DUrole{std,std-ref}{trojans}}}},
and {\hyperref[\detokenize{celestial_bodies/trojan_satellite:id1}]{\sphinxcrossref{\DUrole{std,std-ref}{trojan satellites}}}}, and orbit around {\hyperref[\detokenize{celestial_bodies/star:id1}]{\sphinxcrossref{\DUrole{std,std-ref}{stars}}}}
or {\hyperref[\detokenize{celestial_systems/binary_system:id1}]{\sphinxcrossref{\DUrole{std,std-ref}{stellar binaries}}}}.


\section{Asteroid Belt}
\label{\detokenize{celestial_bodies/asteroid_belt:asteroid-belt}}\label{\detokenize{celestial_bodies/asteroid_belt::doc}}\phantomsection\label{\detokenize{celestial_bodies/asteroid_belt:id1}}
\sphinxAtStartPar
An asteroid belt (or circumstellar disc) is is a torus,
pancake or ring\sphinxhyphen{}shaped accumulation of matter composed of
gas, dust, planetesimals, asteroids, or collision fragments in {\hyperref[\detokenize{quantities/orbital/orbital:id1}]{\sphinxcrossref{\DUrole{std,std-ref}{orbit}}}} around a {\hyperref[\detokenize{celestial_bodies/star:id1}]{\sphinxcrossref{\DUrole{std,std-ref}{star}}}}.


\section{Satellite}
\label{\detokenize{celestial_bodies/satellite:satellite}}\label{\detokenize{celestial_bodies/satellite::doc}}\phantomsection\label{\detokenize{celestial_bodies/satellite:id1}}
\sphinxAtStartPar
A satellite is a celestial object that is {\hyperref[\detokenize{quantities/orbital/orbital:id1}]{\sphinxcrossref{\DUrole{std,std-ref}{orbiting}}}}
around a {\hyperref[\detokenize{celestial_bodies/planet:id1}]{\sphinxcrossref{\DUrole{std,std-ref}{planet}}}}. It’s {\hyperref[\detokenize{quantities/material/composition_type:id1}]{\sphinxcrossref{\DUrole{std,std-ref}{composition}}}}
varies between a rocky world and water world.
There is no a minimum {\hyperref[\detokenize{quantities/material/mass:id1}]{\sphinxcrossref{\DUrole{std,std-ref}{mass}}}} requirement.

\sphinxAtStartPar
The more {\hyperref[\detokenize{quantities/material/mass:id1}]{\sphinxcrossref{\DUrole{std,std-ref}{massive}}}} the parent {\hyperref[\detokenize{celestial_bodies/planet:id1}]{\sphinxcrossref{\DUrole{std,std-ref}{planet}}}},
and the bigger the {\hyperref[\detokenize{quantities/orbital/semi_major_axis:id1}]{\sphinxcrossref{\DUrole{std,std-ref}{distance}}}} between child\sphinxhyphen{}parent,
the more massive a satellite can be.
If a satellite is massive enough, it can resemble a planet and can
potentially sustain {\hyperref[\detokenize{quantities/habitability/habitability:id1}]{\sphinxcrossref{\DUrole{std,std-ref}{life}}}}.


\section{Trojan}
\label{\detokenize{celestial_bodies/trojan:trojan}}\label{\detokenize{celestial_bodies/trojan::doc}}\phantomsection\label{\detokenize{celestial_bodies/trojan:id1}}
\sphinxAtStartPar
Trojans are multiple small celestial objects that are {\hyperref[\detokenize{quantities/orbital/orbital:id1}]{\sphinxcrossref{\DUrole{std,std-ref}{orbiting}}}} around a {\hyperref[\detokenize{celestial_bodies/star:id1}]{\sphinxcrossref{\DUrole{std,std-ref}{star}}}}
while locked in the L4 or L5 {\hyperref[\detokenize{quantities/orbital/lagrange_position:id1}]{\sphinxcrossref{\DUrole{std,std-ref}{Lagrange point}}}} of a {\hyperref[\detokenize{celestial_bodies/planet:id1}]{\sphinxcrossref{\DUrole{std,std-ref}{planet}}}}.
Their {\hyperref[\detokenize{quantities/material/composition_type:id1}]{\sphinxcrossref{\DUrole{std,std-ref}{composition}}}}
varies between a rocky world and water world.
There is no a minimum {\hyperref[\detokenize{quantities/material/mass:id1}]{\sphinxcrossref{\DUrole{std,std-ref}{mass}}}} requirement.

\sphinxAtStartPar
The more {\hyperref[\detokenize{quantities/material/mass:id1}]{\sphinxcrossref{\DUrole{std,std-ref}{massive}}}} the parent {\hyperref[\detokenize{celestial_bodies/planet:id1}]{\sphinxcrossref{\DUrole{std,std-ref}{planet}}}},
the higher the total {\hyperref[\detokenize{quantities/material/mass:id1}]{\sphinxcrossref{\DUrole{std,std-ref}{massive}}}} trojan can be.


\section{Trojan Satellite}
\label{\detokenize{celestial_bodies/trojan_satellite:trojan-satellite}}\label{\detokenize{celestial_bodies/trojan_satellite::doc}}\phantomsection\label{\detokenize{celestial_bodies/trojan_satellite:id1}}
\sphinxAtStartPar
A trojan satellite is a celestial object that is {\hyperref[\detokenize{quantities/orbital/orbital:id1}]{\sphinxcrossref{\DUrole{std,std-ref}{orbiting}}}} around a {\hyperref[\detokenize{celestial_bodies/star:id1}]{\sphinxcrossref{\DUrole{std,std-ref}{star}}}}
while locked in the L4 or L5 {\hyperref[\detokenize{quantities/orbital/lagrange_position:id1}]{\sphinxcrossref{\DUrole{std,std-ref}{Lagrange point}}}} of a {\hyperref[\detokenize{celestial_bodies/planet:id1}]{\sphinxcrossref{\DUrole{std,std-ref}{planet}}}}.
Its {\hyperref[\detokenize{quantities/material/composition_type:id1}]{\sphinxcrossref{\DUrole{std,std-ref}{composition}}}}
varies between a rocky world and water world.
There is no a minimum {\hyperref[\detokenize{quantities/material/mass:id1}]{\sphinxcrossref{\DUrole{std,std-ref}{mass}}}} requirement.

\sphinxAtStartPar
The more {\hyperref[\detokenize{quantities/material/mass:id1}]{\sphinxcrossref{\DUrole{std,std-ref}{massive}}}} the parent {\hyperref[\detokenize{celestial_bodies/planet:id1}]{\sphinxcrossref{\DUrole{std,std-ref}{planet}}}},
the more massive a trojan satellite can be.
If a trojan satellite is massive enough, it can resemble a planet and can
potentially sustain {\hyperref[\detokenize{quantities/habitability/habitability:id1}]{\sphinxcrossref{\DUrole{std,std-ref}{life}}}}.


\section{Ring}
\label{\detokenize{celestial_bodies/ring:ring}}\label{\detokenize{celestial_bodies/ring::doc}}\phantomsection\label{\detokenize{celestial_bodies/ring:id1}}
\sphinxAtStartPar
An ring is is a torus,
pancake or ring\sphinxhyphen{}shaped accumulation of matter composed of
gas, dust, asteroids, or collision fragments in {\hyperref[\detokenize{quantities/orbital/orbital:id1}]{\sphinxcrossref{\DUrole{std,std-ref}{orbit}}}} around a {\hyperref[\detokenize{celestial_bodies/planet:id1}]{\sphinxcrossref{\DUrole{std,std-ref}{planet}}}}.

\sphinxAtStartPar
It is not yet implemented in this package. Soon though!


\chapter{Celestial Systems}
\label{\detokenize{celestial_systems/celestial_systems:celestial-systems}}\label{\detokenize{celestial_systems/celestial_systems::doc}}\phantomsection\label{\detokenize{celestial_systems/celestial_systems:id1}}
\sphinxAtStartPar
Celestial systems are systems that include one or more celestial bodies,
and can be part of other celestial systems.


\section{Binary System}
\label{\detokenize{celestial_systems/binary_system:binary-system}}\label{\detokenize{celestial_systems/binary_system::doc}}\phantomsection\label{\detokenize{celestial_systems/binary_system:id1}}
\sphinxAtStartPar
A binary system is a system of two {\hyperref[\detokenize{celestial_bodies/celestial_bodies:id1}]{\sphinxcrossref{\DUrole{std,std-ref}{celestial objects}}}} that orbit around a common center.

\sphinxAtStartPar
In this package, we only implement stellar binary systems.
Planetary binary systems are coming soon though!

\sphinxAtStartPar
There are two main types of binary systems:
\begin{enumerate}
\sphinxsetlistlabels{\arabic}{enumi}{enumii}{}{.}%
\item {} 
\sphinxAtStartPar
P\sphinxhyphen{}type or close binaries. Their distance is small enough so that other bodies can orbit around both of them

\item {} 
\sphinxAtStartPar
S\sphinxhyphen{}type or wide binaries. Their distance is big enough so that other bodies can orbit each object individually.

\end{enumerate}

\sphinxAtStartPar
In some cases a P\sphinxhyphen{}type system can also be an S\sphinxhyphen{}type system, meaning that objects orbit
a) around both hosts, b) around one host.

\sphinxAtStartPar
Binary systems can also also be part of other binary systems. It is wise
to not put too many binaries within binaries, since the orbits become highly unstable.


\section{Planetary System}
\label{\detokenize{celestial_systems/planetary_system:planetary-system}}\label{\detokenize{celestial_systems/planetary_system::doc}}\phantomsection\label{\detokenize{celestial_systems/planetary_system:id1}}
\sphinxAtStartPar
A planetary system consists of a parent/host {\hyperref[\detokenize{celestial_bodies/planet:id1}]{\sphinxcrossref{\DUrole{std,std-ref}{planet}}}} or
a {\hyperref[\detokenize{celestial_systems/binary_system:id1}]{\sphinxcrossref{\DUrole{std,std-ref}{planetary binary}}}} and a number of
{\hyperref[\detokenize{celestial_bodies/satellite:id1}]{\sphinxcrossref{\DUrole{std,std-ref}{satellites}}}} and {\hyperref[\detokenize{celestial_bodies/ring:id1}]{\sphinxcrossref{\DUrole{std,std-ref}{rings}}}} as children.
They can also host  {\hyperref[\detokenize{celestial_bodies/trojan:id1}]{\sphinxcrossref{\DUrole{std,std-ref}{trojans}}}} and {\hyperref[\detokenize{celestial_bodies/trojan_satellite:id1}]{\sphinxcrossref{\DUrole{std,std-ref}{trojan satellites}}}}
on their {\hyperref[\detokenize{quantities/orbital/lagrange_position:id1}]{\sphinxcrossref{\DUrole{std,std-ref}{Lagrange positions}}}} L4 and L5.


\section{Stellar System}
\label{\detokenize{celestial_systems/stellar_system:stellar-system}}\label{\detokenize{celestial_systems/stellar_system::doc}}\phantomsection\label{\detokenize{celestial_systems/stellar_system:id1}}
\sphinxAtStartPar
A stellar system consists of a parent/host {\hyperref[\detokenize{celestial_bodies/star:id1}]{\sphinxcrossref{\DUrole{std,std-ref}{star}}}} or
a {\hyperref[\detokenize{celestial_systems/binary_system:id1}]{\sphinxcrossref{\DUrole{std,std-ref}{stellar binary}}}} and a small number of
{\hyperref[\detokenize{celestial_systems/planetary_system:id1}]{\sphinxcrossref{\DUrole{std,std-ref}{planetary systems}}}} and {\hyperref[\detokenize{celestial_bodies/asteroid_belt:id1}]{\sphinxcrossref{\DUrole{std,std-ref}{asteroid belts}}}} as children.


\section{Multi\sphinxhyphen{}Stellar System}
\label{\detokenize{celestial_systems/multi_stellar_system:multi-stellar-system}}\label{\detokenize{celestial_systems/multi_stellar_system::doc}}\phantomsection\label{\detokenize{celestial_systems/multi_stellar_system:id1}}
\sphinxAtStartPar
A multi\sphinxhyphen{}stellar system consists of a parent/host {\hyperref[\detokenize{celestial_systems/binary_system:id1}]{\sphinxcrossref{\DUrole{std,std-ref}{S\sphinxhyphen{}type stellar binary}}}}
and two individual {\hyperref[\detokenize{celestial_systems/stellar_system:id1}]{\sphinxcrossref{\DUrole{std,std-ref}{stellar systems}}}}.


\chapter{Quantities}
\label{\detokenize{quantities/quantities:quantities}}\label{\detokenize{quantities/quantities::doc}}\phantomsection\label{\detokenize{quantities/quantities:id1}}
\sphinxAtStartPar
Here, we will explore the various physical quantities found in this package.


\section{Material}
\label{\detokenize{quantities/material/material:material}}\label{\detokenize{quantities/material/material::doc}}

\subsection{Mass}
\label{\detokenize{quantities/material/mass:mass}}\label{\detokenize{quantities/material/mass::doc}}\phantomsection\label{\detokenize{quantities/material/mass:id1}}
\sphinxAtStartPar
Mass is the quantity of mater in a physical body.
In the context of this package, mass determines most of other physical characteristics,
like {\hyperref[\detokenize{quantities/geometric/radius:id1}]{\sphinxcrossref{\DUrole{std,std-ref}{radius}}}}, {\hyperref[\detokenize{quantities/surface/emission/luminosity:id1}]{\sphinxcrossref{\DUrole{std,std-ref}{luminosity}}}}, {\hyperref[\detokenize{quantities/rotational/spin_period:id1}]{\sphinxcrossref{\DUrole{std,std-ref}{spin period}}}} and
{\hyperref[\detokenize{quantities/life/lifetime:id1}]{\sphinxcrossref{\DUrole{std,std-ref}{lifetime}}}}.

\sphinxAtStartPar
Suggested (approximate) masses:
\begin{enumerate}
\sphinxsetlistlabels{\arabic}{enumi}{enumii}{}{.}%
\item {} 
\sphinxAtStartPar
For moon\sphinxhyphen{}like satellites, less than 0.05 earth masses (Me)

\item {} 
\sphinxAtStartPar
For rocky planets: up to around 5 earth masses

\item {} 
\sphinxAtStartPar
For ice\sphinxhyphen{}giants: between 5 and 100 earth masses

\item {} 
\sphinxAtStartPar
For gas\sphinxhyphen{}giants: between 100 earth masses and 10 jupiter masses (Mj)

\item {} 
\sphinxAtStartPar
For long\sphinxhyphen{}lived, red stars: 0.081 and 0.5 solar masses (Ms)

\item {} 
\sphinxAtStartPar
For habitable stars: 0.6 to 1.4 solar masses

\item {} 
\sphinxAtStartPar
For short\sphinxhyphen{}live, big blue stars: 1.4 to 50 solar masses.

\end{enumerate}


\subsection{Density}
\label{\detokenize{quantities/material/density:density}}\label{\detokenize{quantities/material/density::doc}}\phantomsection\label{\detokenize{quantities/material/density:id1}}
\sphinxAtStartPar
Density (\(\rho = \frac{M}{V}\)) is the {\hyperref[\detokenize{quantities/material/mass:id1}]{\sphinxcrossref{\DUrole{std,std-ref}{mass}}}} per unit {\hyperref[\detokenize{quantities/geometric/volume:id1}]{\sphinxcrossref{\DUrole{std,std-ref}{volume}}}} of an a substance (or celestial object).
Usual densities in the solar system are between 0.5 and 7 grams/cm\textasciicircum{}3.


\subsection{Composition Type}
\label{\detokenize{quantities/material/composition_type:composition-type}}\label{\detokenize{quantities/material/composition_type::doc}}\phantomsection\label{\detokenize{quantities/material/composition_type:id1}}
\sphinxAtStartPar
The composition type of planets, planetoids, asteroids etc. is
what the approximate composition of a celestial object will be.
There are two types of iron worlds, two types of rocky worlds,
four types of water worlds, one type of ice\sphinxhyphen{} and one of gas\sphinxhyphen{}giants.
\begin{enumerate}
\sphinxsetlistlabels{\arabic}{enumi}{enumii}{}{.}%
\item {} 
\sphinxAtStartPar
Iron worlds (e.g. Mercury) are {\hyperref[\detokenize{quantities/geometric/radius:id1}]{\sphinxcrossref{\DUrole{std,std-ref}{small}}}}, very {\hyperref[\detokenize{quantities/material/density:id1}]{\sphinxcrossref{\DUrole{std,std-ref}{dense}}}} and can sustain limited to no {\hyperref[\detokenize{quantities/surface/internal_heating/tectonic_activity:id1}]{\sphinxcrossref{\DUrole{std,std-ref}{tectonic activity}}}}, which is essential to {\hyperref[\detokenize{quantities/habitability/habitability:id1}]{\sphinxcrossref{\DUrole{std,std-ref}{life}}}}.

\item {} 
\sphinxAtStartPar
Rocky worlds (e.g. Venus, Earth, Mars) are on average a bit {\hyperref[\detokenize{quantities/geometric/radius:id1}]{\sphinxcrossref{\DUrole{std,std-ref}{bigger}}}}, {\hyperref[\detokenize{quantities/material/density:id1}]{\sphinxcrossref{\DUrole{std,std-ref}{dense}}}} and can sustain substantial {\hyperref[\detokenize{quantities/surface/internal_heating/tectonic_activity:id1}]{\sphinxcrossref{\DUrole{std,std-ref}{tectonic activity}}}}.

\item {} 
\sphinxAtStartPar
Water worlds (e.g. \sphinxhref{https://en.wikipedia.org/wiki/Gliese\_1214\_b}{Gliese 1214b}) are of similar {\hyperref[\detokenize{quantities/geometric/radius:id1}]{\sphinxcrossref{\DUrole{std,std-ref}{size}}}} or a bit bigger compared to rocky worlds, {\hyperref[\detokenize{quantities/material/density:id1}]{\sphinxcrossref{\DUrole{std,std-ref}{less dense}}}}, and depending on the amount of rock in their mantle, they may sustain {\hyperref[\detokenize{quantities/surface/internal_heating/tectonic_activity:id1}]{\sphinxcrossref{\DUrole{std,std-ref}{tectonic activity}}}}.

\item {} 
\sphinxAtStartPar
Ice giants (e.g. Neptune, Uranus) are {\hyperref[\detokenize{quantities/geometric/radius:id1}]{\sphinxcrossref{\DUrole{std,std-ref}{bigger}}}}, {\hyperref[\detokenize{quantities/material/density:id1}]{\sphinxcrossref{\DUrole{std,std-ref}{puffy}}}} and can not sustain any {\hyperref[\detokenize{quantities/surface/internal_heating/tectonic_activity:id1}]{\sphinxcrossref{\DUrole{std,std-ref}{tectonic activity}}}}.

\item {} 
\sphinxAtStartPar
Gas giants (e.g. Saturn, Jupiter) are the {\hyperref[\detokenize{quantities/geometric/radius:id1}]{\sphinxcrossref{\DUrole{std,std-ref}{biggest}}}} of planets, very {\hyperref[\detokenize{quantities/material/density:id1}]{\sphinxcrossref{\DUrole{std,std-ref}{puffy}}}} and can not sustain any {\hyperref[\detokenize{quantities/surface/internal_heating/tectonic_activity:id1}]{\sphinxcrossref{\DUrole{std,std-ref}{tectonic activity}}}}.

\end{enumerate}

\sphinxAtStartPar
I find that the {\hyperref[\detokenize{quantities/material/density:id1}]{\sphinxcrossref{\DUrole{std,std-ref}{density}}}} and {\hyperref[\detokenize{quantities/geometric/radius:id1}]{\sphinxcrossref{\DUrole{std,std-ref}{radius}}}}
of asteroids and moons that are not rocky,
can be generally approximated by the four different water worlds,
even though the {\hyperref[\detokenize{quantities/material/chemical_composition:id1}]{\sphinxcrossref{\DUrole{std,std-ref}{chemical composition}}}}
is not accurate by itself.
Same for small gassy worlds (like Pluto).


\subsection{Chemical Composition}
\label{\detokenize{quantities/material/chemical_composition:chemical-composition}}\label{\detokenize{quantities/material/chemical_composition::doc}}\phantomsection\label{\detokenize{quantities/material/chemical_composition:id1}}
\sphinxAtStartPar
Chemical composition is the ratio of different chemical compounds that constitute a substance.
There are 3 main substances that are portraited in this package.
That does not mean that there can not be other, it is just what the planetary {\hyperref[\detokenize{quantities/geometric/radius:id1}]{\sphinxcrossref{\DUrole{std,std-ref}{radius}}}} models represent.

\sphinxAtStartPar
The main chemical compounds are iron (Fe), rock (MgSiO3), water (H2O),
helium (He), Hydrogen (H2) and methane (CH4).


\section{Geometric}
\label{\detokenize{quantities/geometric/geometric:geometric}}\label{\detokenize{quantities/geometric/geometric::doc}}

\subsection{Radius}
\label{\detokenize{quantities/geometric/radius:radius}}\label{\detokenize{quantities/geometric/radius::doc}}\phantomsection\label{\detokenize{quantities/geometric/radius:id1}}
\sphinxAtStartPar
Radius is the variable that defines the size of celestial objects.
The radius determines the {\hyperref[\detokenize{quantities/geometric/circumference:id1}]{\sphinxcrossref{\DUrole{std,std-ref}{circumference}}}}, {\hyperref[\detokenize{quantities/geometric/surface_area:id1}]{\sphinxcrossref{\DUrole{std,std-ref}{surface area}}}}, {\hyperref[\detokenize{quantities/geometric/volume:id1}]{\sphinxcrossref{\DUrole{std,std-ref}{volume}}}} and {\hyperref[\detokenize{quantities/material/density:id1}]{\sphinxcrossref{\DUrole{std,std-ref}{density}}}}.
among other characteristics.
The suggested radius is determined by the {\hyperref[\detokenize{quantities/material/mass:id1}]{\sphinxcrossref{\DUrole{std,std-ref}{mass}}}} of the
object via various radius models. Use values \(\pm 8\) \% around
the suggested value.

\sphinxAtStartPar
Models used:
\begin{enumerate}
\sphinxsetlistlabels{\arabic}{enumi}{enumii}{}{.}%
\item {} 
\sphinxAtStartPar
For planetary models, see \sphinxurl{https://arxiv.org/pdf/0707.2895.pdf}.

\item {} 
\sphinxAtStartPar
For hot gas\sphinxhyphen{}giant models, see \sphinxurl{https://arxiv.org/pdf/1804.03075.pdf}.

\item {} 
\sphinxAtStartPar
For star models, see \sphinxurl{https://academic.oup.com/mnras/article/479/4/5491/5056185}.

\end{enumerate}


\subsection{Circumference}
\label{\detokenize{quantities/geometric/circumference:circumference}}\label{\detokenize{quantities/geometric/circumference::doc}}\phantomsection\label{\detokenize{quantities/geometric/circumference:id1}}
\sphinxAtStartPar
The circumference is determined by the radius \(C = 2 \pi r\).


\subsection{Surface Area}
\label{\detokenize{quantities/geometric/surface_area:surface-area}}\label{\detokenize{quantities/geometric/surface_area::doc}}\phantomsection\label{\detokenize{quantities/geometric/surface_area:id1}}
\sphinxAtStartPar
The surface area is determined by the radius \(A = 4 \pi r^2\).


\subsection{Volume}
\label{\detokenize{quantities/geometric/volume:volume}}\label{\detokenize{quantities/geometric/volume::doc}}\phantomsection\label{\detokenize{quantities/geometric/volume:id1}}
\sphinxAtStartPar
The volume is determined by the radius \(V = \frac{4\pi}{3}r^3\).


\section{Rotational}
\label{\detokenize{quantities/rotational/rotational:rotational}}\label{\detokenize{quantities/rotational/rotational::doc}}

\subsection{Spin Period}
\label{\detokenize{quantities/rotational/spin_period:spin-period}}\label{\detokenize{quantities/rotational/spin_period::doc}}\phantomsection\label{\detokenize{quantities/rotational/spin_period:id1}}
\sphinxAtStartPar
The spin period is the amount of time it takes for a celestial body to
rotate around itself compared to the distant stars.

\sphinxAtStartPar
Planetary spin period is determined by the {\hyperref[\detokenize{quantities/material/mass:id1}]{\sphinxcrossref{\DUrole{std,std-ref}{mass}}}} and {\hyperref[\detokenize{quantities/geometric/radius:id1}]{\sphinxcrossref{\DUrole{std,std-ref}{radius}}}} of the celestial body.
The more massive the body, the faster it rotates.
If there are satellites around the planet large enough
(e.g. earth\sphinxhyphen{}moon), there is a substantial transfer of
angular momentum between the two bodies, making the planet
slow down (earth spin period would have been around 16 hr).


\subsection{Day Period}
\label{\detokenize{quantities/rotational/day_period:day-period}}\label{\detokenize{quantities/rotational/day_period::doc}}\phantomsection\label{\detokenize{quantities/rotational/day_period:id1}}
\sphinxAtStartPar
The day period of a child body is determined by the {\hyperref[\detokenize{quantities/rotational/spin_period:id1}]{\sphinxcrossref{\DUrole{std,std-ref}{spin period}}}}
and the {\hyperref[\detokenize{quantities/orbital/orbital_period:id1}]{\sphinxcrossref{\DUrole{std,std-ref}{orbital period}}}} around the parent body.


\subsection{Axial Tilt}
\label{\detokenize{quantities/rotational/axial_tilt:axial-tilt}}\label{\detokenize{quantities/rotational/axial_tilt::doc}}\phantomsection\label{\detokenize{quantities/rotational/axial_tilt:id1}}
\sphinxAtStartPar
The axial tilt of a child body, also known as obliquity, is
the angle between an object’s rotational axis and its orbital axis.

\sphinxAtStartPar
As of now, it is cosmetic and does not determine any other characteristics.


\section{Life}
\label{\detokenize{quantities/life/life:life}}\label{\detokenize{quantities/life/life::doc}}

\subsection{Age}
\label{\detokenize{quantities/life/age:age}}\label{\detokenize{quantities/life/age::doc}}\phantomsection\label{\detokenize{quantities/life/age:id1}}
\sphinxAtStartPar
The suggested age of a star is set to be half of its {\hyperref[\detokenize{quantities/life/lifetime:id1}]{\sphinxcrossref{\DUrole{std,std-ref}{lifetime}}}}.
The suggested age of any other object is determined by it’s
parent age.


\subsection{Lifetime}
\label{\detokenize{quantities/life/lifetime:lifetime}}\label{\detokenize{quantities/life/lifetime::doc}}\phantomsection\label{\detokenize{quantities/life/lifetime:id1}}
\sphinxAtStartPar
The lifetime of stars is determined by its mass and its luminosity (\(T=\frac{M}{L} \cdot 10\) billion years).

\sphinxAtStartPar
The lifetime of each other body is determined by the lifetime of the parent
minus a hundred million years, which is roughly the amount of time it takes
for planets to form around stars (or satellites to be captured). It is by
no means binding.


\section{Surface}
\label{\detokenize{quantities/surface/surface:surface}}\label{\detokenize{quantities/surface/surface::doc}}\phantomsection\label{\detokenize{quantities/surface/surface:id1}}

\subsection{Emission}
\label{\detokenize{quantities/surface/emission/emission:emission}}\label{\detokenize{quantities/surface/emission/emission::doc}}\phantomsection\label{\detokenize{quantities/surface/emission/emission:id1}}

\subsubsection{Albedo}
\label{\detokenize{quantities/surface/emission/albedo:albedo}}\label{\detokenize{quantities/surface/emission/albedo::doc}}\phantomsection\label{\detokenize{quantities/surface/emission/albedo:id1}}
\sphinxAtStartPar
(Bond) albedo \(A\) is the measure of reflection of incident radiation of an object.
A value of \(A = 0\) means the object absorbs all incident radiation.
A value of \(A = 1\) means the object reflects all incident radiation.

\sphinxAtStartPar
Albedo values for planets in our solar system:
\begin{enumerate}
\sphinxsetlistlabels{\arabic}{enumi}{enumii}{}{.}%
\item {} 
\sphinxAtStartPar
Mercury, Mars \textasciitilde{}0.14.

\item {} 
\sphinxAtStartPar
Earth, Uranus, Neptune \textasciitilde{} 0.3.

\item {} 
\sphinxAtStartPar
Jupiter, Saturn \textasciitilde{} 0.5.

\item {} 
\sphinxAtStartPar
Venus \textasciitilde{} 0.75.

\end{enumerate}

\sphinxAtStartPar
More information and example values on: \sphinxurl{https://en.wikipedia.org/wiki/Albedo}.


\subsubsection{Emissivity}
\label{\detokenize{quantities/surface/emission/emissivity:emissivity}}\label{\detokenize{quantities/surface/emission/emissivity::doc}}\phantomsection\label{\detokenize{quantities/surface/emission/emissivity:id1}}
\sphinxAtStartPar
Emissivity \(\epsilon\) is the measure of how much of the overall radiation on the surface of an object
is emitted outside. Most planets have an emissivity close to 1 (the maximum value).
Emissivity of 0 (minimum value) means that no radiation is emitted.


\subsubsection{Heat Distribution}
\label{\detokenize{quantities/surface/emission/heat_distribution:heat-distribution}}\label{\detokenize{quantities/surface/emission/heat_distribution::doc}}\phantomsection\label{\detokenize{quantities/surface/emission/heat_distribution:id1}}
\sphinxAtStartPar
Heat distribution \(\beta\) is the measure of how well a planet distributes heat,
and can take values from 0 to 1. It highly depends on the rotation speed of the planet.

\sphinxAtStartPar
The faster the rotation, the closer the number is to 1 (usually 1 for all non\sphinxhyphen{}tidally locked planets).
The heat distribution is 0.5 for a planet that is {\hyperref[\detokenize{quantities/children_orbit_limits/tidal_locking_radius:id1}]{\sphinxcrossref{\DUrole{std,std-ref}{tidally locked}}}} to its parent.


\subsubsection{Normalized Greenhouse}
\label{\detokenize{quantities/surface/emission/normalized_greenhouse:normalized-greenhouse}}\label{\detokenize{quantities/surface/emission/normalized_greenhouse::doc}}\phantomsection\label{\detokenize{quantities/surface/emission/normalized_greenhouse:id1}}
\sphinxAtStartPar
The normalized greenhouse effect \(g\) is a measure of the greenhouse effect in the atmosphere of a planet.
It takes values between 0 and 1. The closer to 1, the higher the temperature of the planet will be.
When you get close to 1 (e.g. 0.99), adding a new 9 at the end (e.g. 0.999) will dramatically increase the effect.

\sphinxAtStartPar
The normalized greenhouse effect on earth is approximately 0.34.


\subsubsection{Incident Flux}
\label{\detokenize{quantities/surface/emission/incident_flux:incident-flux}}\label{\detokenize{quantities/surface/emission/incident_flux::doc}}\phantomsection\label{\detokenize{quantities/surface/emission/incident_flux:id1}}
\sphinxAtStartPar
Incident flux is the incoming radiation flux from all major sources.
The total incident flux is added from the parent, the parent of the parent
etc.

\sphinxAtStartPar
The incident flux \(S\) from a single source of {\hyperref[\detokenize{quantities/surface/emission/luminosity:id1}]{\sphinxcrossref{\DUrole{std,std-ref}{luminosity}}}} \(L\)
at effective distance :math:’r\_\{rm eff\}’ is given by the equation: \(\frac{L}{r_{\rm eff}^2}\).

\sphinxAtStartPar
For a child orbiting at {\hyperref[\detokenize{quantities/orbital/semi_major_axis:id1}]{\sphinxcrossref{\DUrole{std,std-ref}{mean distance}}}} \(a\)
and {\hyperref[\detokenize{quantities/orbital/eccentricity:id1}]{\sphinxcrossref{\DUrole{std,std-ref}{eccentricity}}}} \(e\),
there are two types of effective distance we care about.
The first is the one that is related to the average incoming flux and is given by:
\(r_F = a \left(1 - e^2\right)^{1/4}\).

\sphinxAtStartPar
The second is the one that is related to the average surface temperature and is given by:
\(r_T \approx a (1 + \frac{1}{8} e^2 + \frac{21}{512} e^4)\).

\sphinxAtStartPar
For more information: \sphinxurl{https://arxiv.org/pdf/1702.07314.pdf} (eq. 3, 17, 19).


\subsubsection{Temperature}
\label{\detokenize{quantities/surface/emission/temperature:temperature}}\label{\detokenize{quantities/surface/emission/temperature::doc}}\phantomsection\label{\detokenize{quantities/surface/emission/temperature:id1}}
\sphinxAtStartPar
The surface temperature of an object can be defined differently for objects
that produce significant radiation (stars) and ones that are mostly heated
by another stellar body (planets, satellites).

\sphinxAtStartPar
For a star of {\hyperref[\detokenize{quantities/surface/emission/luminosity:id1}]{\sphinxcrossref{\DUrole{std,std-ref}{luminosity}}}} \(L\) in solar luminosities
and {\hyperref[\detokenize{quantities/geometric/radius:id1}]{\sphinxcrossref{\DUrole{std,std-ref}{radius}}}} \(R\) in solar radii, the surface temperature
is given by the ideal body equation: \(\frac{L}{R^2} \cdot 5778\) K.

\sphinxAtStartPar
For a planet of {\hyperref[\detokenize{quantities/surface/emission/albedo:id1}]{\sphinxcrossref{\DUrole{std,std-ref}{bond albedo}}}} \(A\), {\hyperref[\detokenize{quantities/surface/emission/emissivity:id1}]{\sphinxcrossref{\DUrole{std,std-ref}{emissivity}}}} \(\epsilon\),
{\hyperref[\detokenize{quantities/surface/emission/heat_distribution:id1}]{\sphinxcrossref{\DUrole{std,std-ref}{heat distribution}}}} \(\beta\),
{\hyperref[\detokenize{quantities/surface/emission/normalized_greenhouse:id1}]{\sphinxcrossref{\DUrole{std,std-ref}{normalized greenhouse}}}} \(g\) and
{\hyperref[\detokenize{quantities/surface/emission/incident_flux:id1}]{\sphinxcrossref{\DUrole{std,std-ref}{incident flux}}}} \(S\),
the surface temperature is given by the equation:
\(\left(\frac{(1 - A) S}{\beta \epsilon (1 - g)}\right)^{1/4} \cdot 278.5\) K.

\sphinxAtStartPar
For more information of the planetary surface temperature:
\sphinxurl{https://arxiv.org/pdf/1702.07314.pdf} (eq. 6 and 16)


\subsubsection{Luminosity}
\label{\detokenize{quantities/surface/emission/luminosity:luminosity}}\label{\detokenize{quantities/surface/emission/luminosity::doc}}\phantomsection\label{\detokenize{quantities/surface/emission/luminosity:id1}}
\sphinxAtStartPar
Luminosity is the measure of the emitted radiation from an object.
It is determined by the {\hyperref[\detokenize{quantities/material/mass:id1}]{\sphinxcrossref{\DUrole{std,std-ref}{mass}}}} of the object, and the
{\hyperref[\detokenize{quantities/surface/emission/incident_flux:id1}]{\sphinxcrossref{\DUrole{std,std-ref}{incident flux}}}} if it is relatively higher than the emitted flux
of radiation.

\sphinxAtStartPar
For star of {\hyperref[\detokenize{quantities/material/mass:id1}]{\sphinxcrossref{\DUrole{std,std-ref}{mass}}}} \(M\) in the main sequence, the luminosity
is given by various mass\sphinxhyphen{}luminosity relationships as in \sphinxurl{https://en.wikipedia.org/wiki/Mass-luminosity\_relation}.
Use values \(\pm 8\) \% around the suggested value.

\sphinxAtStartPar
For a planet of {\hyperref[\detokenize{quantities/surface/emission/temperature:id1}]{\sphinxcrossref{\DUrole{std,std-ref}{surface temperature}}}} \(T\),
{\hyperref[\detokenize{quantities/surface/emission/incident_flux:id1}]{\sphinxcrossref{\DUrole{std,std-ref}{temperature\sphinxhyphen{}based incident flux}}}} \(S_T\),
{\hyperref[\detokenize{quantities/surface/emission/albedo:id1}]{\sphinxcrossref{\DUrole{std,std-ref}{albedo}}}} \(A\) and {\hyperref[\detokenize{quantities/geometric/surface_area:id1}]{\sphinxcrossref{\DUrole{std,std-ref}{surface area}}}} \(A_S\),
the luminosity is determined by:
\(\sigma A_S T ^ 4 + A S_T A_S / 4\),
where \(\sigma\) is the Stefan\sphinxhyphen{}Boltzmann constant.

\sphinxAtStartPar
For more information on planetary luminosity:
\begin{enumerate}
\sphinxsetlistlabels{\arabic}{enumi}{enumii}{}{.}%
\item {} 
\sphinxAtStartPar
\sphinxurl{https://www.acs.org/content/acs/en/climatescience/atmosphericwarming/singlelayermodel.html}

\item {} 
\sphinxAtStartPar
\sphinxurl{https://en.wikipedia.org/wiki/Planetary\_equilibrium\_temperature}

\end{enumerate}


\subsubsection{Peak Wavelength}
\label{\detokenize{quantities/surface/emission/peak_wavelength:peak-wavelength}}\label{\detokenize{quantities/surface/emission/peak_wavelength::doc}}\phantomsection\label{\detokenize{quantities/surface/emission/peak_wavelength:id1}}
\sphinxAtStartPar
The {\hyperref[\detokenize{quantities/surface/emission/temperature:id1}]{\sphinxcrossref{\DUrole{std,std-ref}{surface temperature}}}} \(T\) of an object,
if it is an (close to) ideal black body object can be associated with
a profile of emitted wavelengths, with a single maximum.
For our sun, the maximum of the emission spectrum is around 500 nm (green).

\sphinxAtStartPar
In general, the peak wavelength it is given by: \(\lambda = \frac{2.898 \cdot 10 ^ 6 \,{\rm nm\, K}}{T}\).


\subsection{Gravity}
\label{\detokenize{quantities/surface/gravity/gravity:gravity}}\label{\detokenize{quantities/surface/gravity/gravity::doc}}\phantomsection\label{\detokenize{quantities/surface/gravity/gravity:id1}}

\subsubsection{Surface Gravity}
\label{\detokenize{quantities/surface/gravity/surface_gravity:surface-gravity}}\label{\detokenize{quantities/surface/gravity/surface_gravity::doc}}\phantomsection\label{\detokenize{quantities/surface/gravity/surface_gravity:id1}}
\sphinxAtStartPar
Surface gravity \(g\) is the gravitational acceleration experienced at
the surface of an object.

\sphinxAtStartPar
Earth’s surface gravity is approximately 9.81 \({\rm m/s^2}\).


\subsubsection{Escape Velocity}
\label{\detokenize{quantities/surface/gravity/escape_velocity:escape-velocity}}\label{\detokenize{quantities/surface/gravity/escape_velocity::doc}}\phantomsection\label{\detokenize{quantities/surface/gravity/escape_velocity:id1}}
\sphinxAtStartPar
Escape velocity \(v_{\rm esc}\) is the initial speed a small object needs to escape the gravitational
pull of the celestial object it is bound to.

\sphinxAtStartPar
Earth’s escape velocity is approximately 11.19 km/s.


\subsection{Internal Heating}
\label{\detokenize{quantities/surface/internal_heating/internal_heating:internal-heating}}\label{\detokenize{quantities/surface/internal_heating/internal_heating::doc}}\phantomsection\label{\detokenize{quantities/surface/internal_heating/internal_heating:id1}}
\sphinxAtStartPar
Internal heating are a variety of processes with which an object generates heat.


\subsubsection{Tectonic Activity}
\label{\detokenize{quantities/surface/internal_heating/tectonic_activity:tectonic-activity}}\label{\detokenize{quantities/surface/internal_heating/tectonic_activity::doc}}\phantomsection\label{\detokenize{quantities/surface/internal_heating/tectonic_activity:id1}}
\sphinxAtStartPar
The tectonic activity label is determined by amount of total internal heating \(Q_{\rm tot}\).
Since each internal heating process is not estimated very accurately, take the
label as a suggestion, rather than an absolute truth, and optimize however you see fit.
Plate tectonics are only present on sold planets, so any planets with substantial gaseous
components will not have any plate tectonics.

\sphinxAtStartPar
The labels we use are:
\begin{enumerate}
\sphinxsetlistlabels{\arabic}{enumi}{enumii}{}{.}%
\item {} 
\sphinxAtStartPar
‘Not applicable’, if the composition is Icegiant or Gasgiant.

\item {} 
\sphinxAtStartPar
‘Unknown’, if \(Q_{\rm tot} = {\rm nan}\).

\item {} 
\sphinxAtStartPar
‘Stagnant’, if \(Q_{\rm tot} < 0.01 \, {\rm W/m^2}\).

\item {} 
\sphinxAtStartPar
‘Low’, if \(Q_{\rm tot} < 0.04 \, {\rm W/m^2}\).

\item {} 
\sphinxAtStartPar
‘Medium Low’, if \(Q_{\rm tot} < 0.07 \, {\rm W/m^2}\).

\item {} 
\sphinxAtStartPar
‘Medium’, if \(Q_{\rm tot} < 0.15 \, {\rm W/m^2}\).

\item {} 
\sphinxAtStartPar
‘Medium High’, if \(Q_{\rm tot} < 0.2 \, {\rm W/m^2}\).

\item {} 
\sphinxAtStartPar
‘High’, if \(Q_{\rm tot} < 0.35 \, {\rm W/m^2}\).

\item {} 
\sphinxAtStartPar
‘Extreme’, if \(Q_{\rm tot} \ge 0.35 \, {\rm W/m^2}\).

\end{enumerate}


\subsubsection{Primordial Heating}
\label{\detokenize{quantities/surface/internal_heating/primordial_heating:primordial-heating}}\label{\detokenize{quantities/surface/internal_heating/primordial_heating::doc}}\phantomsection\label{\detokenize{quantities/surface/internal_heating/primordial_heating:id1}}
\sphinxAtStartPar
Primordial heating \(Q_{\rm prim}\) of a planet is heating that originates from the initial formation heating.
It is stored in the planet as an internal heating source that slowly (or quickly)
escapes, depending on the planet materials.

\sphinxAtStartPar
For a planet of {\hyperref[\detokenize{quantities/material/mass:id1}]{\sphinxcrossref{\DUrole{std,std-ref}{mass}}}} \(M\), {\hyperref[\detokenize{quantities/geometric/surface_area:id1}]{\sphinxcrossref{\DUrole{std,std-ref}{surface area}}}} \(S\),
{\hyperref[\detokenize{quantities/life/age:id1}]{\sphinxcrossref{\DUrole{std,std-ref}{age}}}} \(T\), and initial heating \(H_o\), the primordial heating is given by:
\(Q_{\rm prim} = H_o {\rm e}^{- \lambda T} M / S\), where \(\lambda\) is a decay constant.

\sphinxAtStartPar
For this package we use a constant \(H_o = 1.2 \cdot 10^{-11}\) Watts/kg and
\(\lambda = \frac{0.3391}{\rm billion\,years}\).

\sphinxAtStartPar
More information on:
\sphinxurl{https://www.sciencedirect.com/science/article/abs/pii/S003206331300161X} Eq. 2.1 \(\cdot M / S\)


\subsubsection{Radiogenic Heating}
\label{\detokenize{quantities/surface/internal_heating/radiogenic_heating:radiogenic-heating}}\label{\detokenize{quantities/surface/internal_heating/radiogenic_heating::doc}}\phantomsection\label{\detokenize{quantities/surface/internal_heating/radiogenic_heating:id1}}
\sphinxAtStartPar
Radiogenic heating \(Q_{\rm rad}\) is the heating produced by slowly radiative isotopes in a planets mantle.
Since we only care for the mantle, we take into account the planetary composition
(we assume that only the rocky part of the planet is contributing to radiogenic heating).
This model assumes the same percentages of radioactive isotopes as earth, although these may vary
from planet to planet and stellar system to stellar system, depending on the age of the galaxy
and local isotope abundances.

\sphinxAtStartPar
For a radiogenic isotope of number \(k\) with heating production \(H_k\), initial abundance \(n_k (0)\),
lifetime \(\tau_k\), the heating heating produced at a certain time ({\hyperref[\detokenize{quantities/life/age:id1}]{\sphinxcrossref{\DUrole{std,std-ref}{age}}}}) \(t\)
is given by: \(H_k(t) = H_k n_k (0) {\rm e}^{- t / \tau}\).

\sphinxAtStartPar
For a planet of {\hyperref[\detokenize{quantities/material/mass:id1}]{\sphinxcrossref{\DUrole{std,std-ref}{mass}}}} \(M\), {\hyperref[\detokenize{quantities/geometric/surface_area:id1}]{\sphinxcrossref{\DUrole{std,std-ref}{surface area}}}} \(S\),
{\hyperref[\detokenize{quantities/life/age:id1}]{\sphinxcrossref{\DUrole{std,std-ref}{age}}}} \(T\), and rocky mantle percentage \(p_{\rm rocky}\) (see {\hyperref[\detokenize{quantities/material/chemical_composition:id1}]{\sphinxcrossref{\DUrole{std,std-ref}{chemical composition}}}}
and {\hyperref[\detokenize{quantities/material/composition_type:id1}]{\sphinxcrossref{\DUrole{std,std-ref}{composition type}}}}), the total heating is given by:
\(Q_{\rm rad} = \sum_k \left(H_k(T)\right) p_{\rm rocky} M / S\).

\sphinxAtStartPar
More information on:
\sphinxurl{https://www.sciencedirect.com/science/article/abs/pii/S0019103514004473?via\%3Dihub\#b0415\%20}.
See table 1 for heat production and half\sphinxhyphen{}times, and table 2 for relative abundance.


\subsubsection{Tidal Heating}
\label{\detokenize{quantities/surface/internal_heating/tidal_heating:tidal-heating}}\label{\detokenize{quantities/surface/internal_heating/tidal_heating::doc}}\phantomsection\label{\detokenize{quantities/surface/internal_heating/tidal_heating:id1}}
\sphinxAtStartPar
When an child object of {\hyperref[\detokenize{quantities/geometric/radius:id1}]{\sphinxcrossref{\DUrole{std,std-ref}{radius}}}} \(r\) rotates around a parent object
of {\hyperref[\detokenize{quantities/material/mass:id1}]{\sphinxcrossref{\DUrole{std,std-ref}{mass}}}} \(M\) in an non\sphinxhyphen{}circular (eccentric) orbit, at a
{\hyperref[\detokenize{quantities/orbital/semi_major_axis:id1}]{\sphinxcrossref{\DUrole{std,std-ref}{mean distance}}}} \(a\) and {\hyperref[\detokenize{quantities/orbital/eccentricity:id1}]{\sphinxcrossref{\DUrole{std,std-ref}{eccentricity}}}} \(e\),
the shape of the child object can periodically change due to the variation of gravitational pull
it feels from the parent object (it wobbles). This motion results in internal heating.

\sphinxAtStartPar
Tidal heating is proportional to: \(Q_{\rm tidal} \propto M^{2.5} r^5 e^2 a^{-7.5}\)

\sphinxAtStartPar
More information on:
\begin{enumerate}
\sphinxsetlistlabels{\arabic}{enumi}{enumii}{}{.}%
\item {} 
\sphinxAtStartPar
\sphinxurl{https://academic.oup.com/mnras/article/391/1/237/1121115}

\item {} 
\sphinxAtStartPar
\sphinxurl{https://www.liebertpub.com/doi/10.1089/ast.2015.1325} Eq.2

\item {} 
\sphinxAtStartPar
\sphinxurl{https://iopscience.iop.org/article/10.1088/0004-637X/789/1/30/pdf} table 2, pg. 22

\end{enumerate}


\subsection{Induced Tide}
\label{\detokenize{quantities/surface/induced_tide:induced-tide}}\label{\detokenize{quantities/surface/induced_tide::doc}}
\sphinxAtStartPar
Induced tides \(h_{\rm tides}\) are the height differences that occur due to tidal forces
on a planet’s massive ocean’s water level. The values provided are very
crued and are only meant as a suggestion to the user. Tide height depends
on many local parameters, as explained in \sphinxhref{https://youtu.be/wsuejZRqus4}{artifexian’s video}.

\sphinxAtStartPar
For an object of {\hyperref[\detokenize{quantities/material/mass:id1}]{\sphinxcrossref{\DUrole{std,std-ref}{mass}}}} \(m\) and {\hyperref[\detokenize{quantities/geometric/radius:id1}]{\sphinxcrossref{\DUrole{std,std-ref}{radius}}}} \(r\)
that interacts with a companion object of {\hyperref[\detokenize{quantities/material/mass:id1}]{\sphinxcrossref{\DUrole{std,std-ref}{mass}}}} \(m_c\)
from a {\hyperref[\detokenize{quantities/orbital/semi_major_axis:id1}]{\sphinxcrossref{\DUrole{std,std-ref}{mean distance}}}} \(a\)
with orbital {\hyperref[\detokenize{quantities/orbital/eccentricity:id1}]{\sphinxcrossref{\DUrole{std,std-ref}{eccentricity}}}} \(e\),
will experience a maximum tide height of:
\(h_{\rm tides} = 3 \frac{m_c}{m} \left(\frac{r}{a (1 - e)}\right)^3 \frac{r}{2}\).

\sphinxAtStartPar
More information on the model used can be found on
\sphinxurl{https://www.cambridge.org/resources/0521846560/7708\_Tidal\%20distortion.pdf} (Eq. 20)
where mean distance is replaced by the periapsis (which yields the maximum observed tides).
A mistake must be noted: the example below eq. 20 needs \(m_1\) to be the earth mass
and \(m_2\) the moon mass.


\subsection{Angular Diameter}
\label{\detokenize{quantities/surface/angular_diameter:angular-diameter}}\label{\detokenize{quantities/surface/angular_diameter::doc}}\phantomsection\label{\detokenize{quantities/surface/angular_diameter:id1}}
\sphinxAtStartPar
Angular diameter \(\delta\) is the size of a celestial body in the sky. The angular diameter of the sun and the moon are
similar on earth, and approximately \textasciitilde{} 0.5 degrees

\sphinxAtStartPar
The angular diameter of a body with {\hyperref[\detokenize{quantities/geometric/radius:id1}]{\sphinxcrossref{\DUrole{std,std-ref}{radius}}}} \(r\) at a distance \(D\)
is given by: \(\delta = 2 \arcsin(r/D)\)

\sphinxAtStartPar
More information on: \sphinxurl{https://en.wikipedia.org/wiki/Angular\_diameter}


\section{Orbital}
\label{\detokenize{quantities/orbital/orbital:orbital}}\label{\detokenize{quantities/orbital/orbital::doc}}\phantomsection\label{\detokenize{quantities/orbital/orbital:id1}}
\sphinxAtStartPar
An orbit is a curved trajectory of an object. This trajectory can either be circular, or elliptical.


\subsection{Eccentricity}
\label{\detokenize{quantities/orbital/eccentricity:eccentricity}}\label{\detokenize{quantities/orbital/eccentricity::doc}}\phantomsection\label{\detokenize{quantities/orbital/eccentricity:id1}}
\sphinxAtStartPar
Eccentricity \(e\) determines how elliptic the orbit of a child around a parent body is.
\(e = 0\) means that the orbit is circular, and \(e = 1\) means
that the orbit resembles a line (not a stable orbit).

\sphinxAtStartPar
The suggested eccentricity is \(e = 0\) for single {\hyperref[\detokenize{celestial_bodies/star:id1}]{\sphinxcrossref{\DUrole{std,std-ref}{star}}}} systems.
For {\hyperref[\detokenize{celestial_systems/binary_system:id1}]{\sphinxcrossref{\DUrole{std,std-ref}{binary systems}}}}, the suggested eccentricity depends on the
eccentricity of the binary \(e_c\),
the {\hyperref[\detokenize{quantities/orbital/semi_major_axis:id1}]{\sphinxcrossref{\DUrole{std,std-ref}{semi\sphinxhyphen{}major axis}}}} \(a_c\) of the child, the
{\hyperref[\detokenize{quantities/orbital/semi_major_axis:id1}]{\sphinxcrossref{\DUrole{std,std-ref}{mean distance}}}} \(a_b\) between the two binary system objects,
and the secondary to total {\hyperref[\detokenize{quantities/material/mass:id1}]{\sphinxcrossref{\DUrole{std,std-ref}{mass}}}} ratio \(\mu\) of the binary. Small
variations of the eccentricity (e.g. \(\pm 0.05\)) are suggested for a more realistic
{\hyperref[\detokenize{quantities/orbital/orbital:id1}]{\sphinxcrossref{\DUrole{std,std-ref}{orbit}}}}.

\sphinxAtStartPar
For S\sphinxhyphen{}type {\hyperref[\detokenize{celestial_systems/binary_system:id1}]{\sphinxcrossref{\DUrole{std,std-ref}{binary systems}}}}, the suggested eccentricity is given by:
\(e_S = \frac{5}{4} \frac{a_c}{a_b} \frac{e_b}{1-e_b^2}\).

\sphinxAtStartPar
For P\sphinxhyphen{}type {\hyperref[\detokenize{celestial_systems/binary_system:id1}]{\sphinxcrossref{\DUrole{std,std-ref}{binary systems}}}}, the suggested eccentricity is given by:
\(e_P = \frac{5}{4} \frac{a_b}{a_c} (1-2\mu) \frac{4 e_b + 3e_b^3}{4 + 6e_b^2}\).


\subsection{Semi\sphinxhyphen{}Major Axis}
\label{\detokenize{quantities/orbital/semi_major_axis:semi-major-axis}}\label{\detokenize{quantities/orbital/semi_major_axis::doc}}\phantomsection\label{\detokenize{quantities/orbital/semi_major_axis:id1}}
\sphinxAtStartPar
Semi\sphinxhyphen{}major axis \(a\) is the mean distance between a child and a parent body.

\sphinxAtStartPar
When placing two objects in the same {\hyperref[\detokenize{celestial_systems/celestial_systems:id1}]{\sphinxcrossref{\DUrole{std,std-ref}{celestial system}}}}
with semi\sphinxhyphen{}major axes \(a_1\) and \(a_2\),
try to space them out in such a way that \(a_2 = (1.4 \, {\rm to} \, 2) \cdot a1\).
Maintain this distance ratio for all new objects.
Also, make sure to stay within the {\hyperref[\detokenize{quantities/orbital/semi_major_axis_minimum_limit:id1}]{\sphinxcrossref{\DUrole{std,std-ref}{minimum}}}}
and {\hyperref[\detokenize{quantities/orbital/semi_major_axis_maximum_limit:id1}]{\sphinxcrossref{\DUrole{std,std-ref}{maximum}}}} limits since
the program will not stop the user from exceeding the limits.


\subsection{Semi\sphinxhyphen{}Minor Axis}
\label{\detokenize{quantities/orbital/semi_minor_axis:semi-minor-axis}}\label{\detokenize{quantities/orbital/semi_minor_axis::doc}}\phantomsection\label{\detokenize{quantities/orbital/semi_minor_axis:id1}}
\sphinxAtStartPar
Semi\sphinxhyphen{}minor axis \(b\) is determined by the {\hyperref[\detokenize{quantities/orbital/semi_major_axis:id1}]{\sphinxcrossref{\DUrole{std,std-ref}{semi\sphinxhyphen{}major axis}}}} \(a\) and the
{\hyperref[\detokenize{quantities/orbital/eccentricity:id1}]{\sphinxcrossref{\DUrole{std,std-ref}{eccentricity}}}} \(e\): \(b = a \sqrt{1 - e^2}\).


\subsection{Apoapsis}
\label{\detokenize{quantities/orbital/apoapsis:apoapsis}}\label{\detokenize{quantities/orbital/apoapsis::doc}}\phantomsection\label{\detokenize{quantities/orbital/apoapsis:id1}}
\sphinxAtStartPar
Apoapsis (\(a (1 + e)\)) is the furthest distance between a child and a parent body.
It is determined by the {\hyperref[\detokenize{quantities/orbital/semi_major_axis:id1}]{\sphinxcrossref{\DUrole{std,std-ref}{semi\sphinxhyphen{}major axis}}}} \(a\) and the
{\hyperref[\detokenize{quantities/orbital/eccentricity:id1}]{\sphinxcrossref{\DUrole{std,std-ref}{eccentricity}}}} \(e\).


\subsection{Periapsis}
\label{\detokenize{quantities/orbital/periapsis:periapsis}}\label{\detokenize{quantities/orbital/periapsis::doc}}\phantomsection\label{\detokenize{quantities/orbital/periapsis:id1}}
\sphinxAtStartPar
Periapsis (\(a (1 - e)\)) is the nearest distance between a child and a parent body.
It is determined by the {\hyperref[\detokenize{quantities/orbital/semi_major_axis:id1}]{\sphinxcrossref{\DUrole{std,std-ref}{semi\sphinxhyphen{}major axis}}}} \(a\) and the
{\hyperref[\detokenize{quantities/orbital/eccentricity:id1}]{\sphinxcrossref{\DUrole{std,std-ref}{eccentricity}}}} \(e\).


\subsection{Lagrange Position}
\label{\detokenize{quantities/orbital/lagrange_position:lagrange-position}}\label{\detokenize{quantities/orbital/lagrange_position::doc}}\phantomsection\label{\detokenize{quantities/orbital/lagrange_position:id1}}
\sphinxAtStartPar
Lagrange positions/points (L\#) are the equilibrium points between
two {\hyperref[\detokenize{quantities/material/mass:id1}]{\sphinxcrossref{\DUrole{std,std-ref}{masses}}}} (grandparent \(m_1\) and parent \(m_2\)) that can
potentially host an object of small mass \(m_3\). There are
5 such points, three (L1, L2, L3) of which are unstable and two that are semi\sphinxhyphen{}stable (L4, L5).
Points L4 and L5 can host from multiple asteroid\sphinxhyphen{}type bodies to a small planet, depending on \(m_2\).

\sphinxAtStartPar
The rule of thumb is that the total {\hyperref[\detokenize{quantities/material/mass:id1}]{\sphinxcrossref{\DUrole{std,std-ref}{mass}}}} of objects in these points
must be smaller than \(0.0018878 \cdot m_2\) or smaller than the Gascheu’s limit.
It must be noted that, since the three bodies are never just by themselves,
there are other orbital instabilities introduced that
further degrade the stability of the Lagrange points.

\sphinxAtStartPar
More information about the limits on:
\begin{enumerate}
\sphinxsetlistlabels{\arabic}{enumi}{enumii}{}{.}%
\item {} 
\sphinxAtStartPar
\sphinxurl{https://hal.archives-ouvertes.fr/hal-00552502/document} (For a discussion on Gascheu’s limit)

\item {} 
\sphinxAtStartPar
\sphinxurl{https://www.aanda.org/articles/aa/abs/2007/07/aa6582-06/aa6582-06.html} (For simple \(0.0018878 \cdot m_2\) limit)

\end{enumerate}


\subsection{Extent}
\label{\detokenize{quantities/orbital/extent:extent}}\label{\detokenize{quantities/orbital/extent::doc}}\phantomsection\label{\detokenize{quantities/orbital/extent:id1}}
\sphinxAtStartPar
Extent is the distance around the {\hyperref[\detokenize{quantities/orbital/semi_major_axis:id1}]{\sphinxcrossref{\DUrole{std,std-ref}{semi\sphinxhyphen{}major axis}}}}
for which an {\hyperref[\detokenize{celestial_bodies/asteroid_belt:id1}]{\sphinxcrossref{\DUrole{std,std-ref}{asteroid belt}}}} or {\hyperref[\detokenize{celestial_bodies/trojan:id1}]{\sphinxcrossref{\DUrole{std,std-ref}{trojan asteroids}}}} extends to.

\sphinxAtStartPar
The default value is 1/8 of the {\hyperref[\detokenize{quantities/orbital/semi_major_axis:id1}]{\sphinxcrossref{\DUrole{std,std-ref}{semi\sphinxhyphen{}major axis}}}}.


\subsection{Contact}
\label{\detokenize{quantities/orbital/contact:contact}}\label{\detokenize{quantities/orbital/contact::doc}}\phantomsection\label{\detokenize{quantities/orbital/contact:id1}}
\sphinxAtStartPar
In this context, contact between two binary stars happens if the radius of at least
one of the two stars resides outside the {\hyperref[\detokenize{quantities/orbital/roche_lobe:id1}]{\sphinxcrossref{\DUrole{std,std-ref}{Roche lobe}}}} while the stars
are in {\hyperref[\detokenize{quantities/orbital/periapsis:id1}]{\sphinxcrossref{\DUrole{std,std-ref}{periapsis}}}}.


\subsection{Roche Lobe}
\label{\detokenize{quantities/orbital/roche_lobe:roche-lobe}}\label{\detokenize{quantities/orbital/roche_lobe::doc}}\phantomsection\label{\detokenize{quantities/orbital/roche_lobe:id1}}
\sphinxAtStartPar
The Roche lobe is the region around a star in a binary system within which orbiting material is gravitationally bound
to that star.


\subsection{Orbital Period}
\label{\detokenize{quantities/orbital/orbital_period:orbital-period}}\label{\detokenize{quantities/orbital/orbital_period::doc}}\phantomsection\label{\detokenize{quantities/orbital/orbital_period:id1}}
\sphinxAtStartPar
Orbital period (\(\sqrt{\frac{a^3}{M_{tot}}}\)) is the time it takes for a child body to orbit around their parent.
It is given by the {\hyperref[\detokenize{quantities/orbital/semi_major_axis:id1}]{\sphinxcrossref{\DUrole{std,std-ref}{semi\sphinxhyphen{}major axis}}}} \(a\), and the total mass \(M_{tot}\) of child
and parent objects.


\subsection{Orbital Velocity}
\label{\detokenize{quantities/orbital/orbital_velocity:orbital-velocity}}\label{\detokenize{quantities/orbital/orbital_velocity::doc}}\phantomsection\label{\detokenize{quantities/orbital/orbital_velocity:id1}}
\sphinxAtStartPar
Orbital velocity (\(\sqrt{\frac{M_{tot}}{a}}\)) is the mean speed at which the child body
travels around the parent body. It is given by the {\hyperref[\detokenize{quantities/orbital/semi_major_axis:id1}]{\sphinxcrossref{\DUrole{std,std-ref}{semi\sphinxhyphen{}major axis}}}} \(a\), and the
total mass \(M_{tot}\) of child and parent objects.


\subsection{Orbit Type}
\label{\detokenize{quantities/orbital/orbit_type:orbit-type}}\label{\detokenize{quantities/orbital/orbit_type::doc}}\phantomsection\label{\detokenize{quantities/orbital/orbit_type:id1}}
\sphinxAtStartPar
Orbital type of a child object can be either prograde or retrograde \sphinxhyphen{} orbiting
along or against the rotation of the parent.


\subsection{Orbit Type Factor}
\label{\detokenize{quantities/orbital/orbit_type_factor:orbit-type-factor}}\label{\detokenize{quantities/orbital/orbit_type_factor::doc}}\phantomsection\label{\detokenize{quantities/orbital/orbit_type_factor:id1}}
\sphinxAtStartPar
Orbit type factor determines the {\hyperref[\detokenize{quantities/orbital/semi_major_axis_maximum_limit:id1}]{\sphinxcrossref{\DUrole{std,std-ref}{semi\sphinxhyphen{}major axis maximum limit}}}}.
It depends on the {\hyperref[\detokenize{quantities/orbital/orbit_type:id1}]{\sphinxcrossref{\DUrole{std,std-ref}{orbit type}}}} and is determined by the parent {\hyperref[\detokenize{quantities/orbital/eccentricity:id1}]{\sphinxcrossref{\DUrole{std,std-ref}{eccentricity}}}} \(e_p\)
and the child {\hyperref[\detokenize{quantities/orbital/eccentricity:id1}]{\sphinxcrossref{\DUrole{std,std-ref}{eccentricity}}}} \(e_c\).
\begin{enumerate}
\sphinxsetlistlabels{\arabic}{enumi}{enumii}{}{.}%
\item {} 
\sphinxAtStartPar
For a prograde orbit, it is given by \(0.4895 (1 - 1.0305 e_p - 0.2738 e_c)\).

\item {} 
\sphinxAtStartPar
For a retrograde orbit, it is given by \(0.9309 (1 - 1.0764 e_p - 0.9812 e_c)\).

\end{enumerate}


\subsection{Orbital Stability}
\label{\detokenize{quantities/orbital/orbital_stability:orbital-stability}}\label{\detokenize{quantities/orbital/orbital_stability::doc}}\phantomsection\label{\detokenize{quantities/orbital/orbital_stability:id1}}
\sphinxAtStartPar
Orbital stability demonstrates if the orbit of the child object around the parent object
does not exceed any limits. More specifically, for a stable orbit we must have:
\begin{enumerate}
\sphinxsetlistlabels{\arabic}{enumi}{enumii}{}{.}%
\item {} 
\sphinxAtStartPar
{\hyperref[\detokenize{quantities/orbital/periapsis:id1}]{\sphinxcrossref{\DUrole{std,std-ref}{Periapsis}}}} \textgreater{} {\hyperref[\detokenize{quantities/children_orbit_limits/roche_limit:id1}]{\sphinxcrossref{\DUrole{std,std-ref}{roche limit}}}}.

\item {} 
\sphinxAtStartPar
{\hyperref[\detokenize{quantities/orbital/semi_major_axis:id1}]{\sphinxcrossref{\DUrole{std,std-ref}{Semi\sphinxhyphen{}major axis}}}} \textgreater{} {\hyperref[\detokenize{quantities/children_orbit_limits/p_type_critical_orbit:p-type-critical-orbit}]{\sphinxcrossref{\DUrole{std,std-ref}{p\sphinxhyphen{}type critical orbit}}}}.

\item {} 
\sphinxAtStartPar
{\hyperref[\detokenize{quantities/orbital/semi_major_axis:id1}]{\sphinxcrossref{\DUrole{std,std-ref}{Semi\sphinxhyphen{}major axis}}}} \textless{} {\hyperref[\detokenize{quantities/orbital/semi_major_axis_maximum_limit:id1}]{\sphinxcrossref{\DUrole{std,std-ref}{semi\sphinxhyphen{}major axis maximum limit}}}}.

\item {} 
\sphinxAtStartPar
Optional: {\hyperref[\detokenize{quantities/orbital/semi_major_axis:id1}]{\sphinxcrossref{\DUrole{std,std-ref}{Semi\sphinxhyphen{}major axis}}}} \textgreater{} {\hyperref[\detokenize{quantities/children_orbit_limits/inner_orbit_limit:id1}]{\sphinxcrossref{\DUrole{std,std-ref}{parent inner orbit limit}}}}.

\item {} 
\sphinxAtStartPar
For rock worlds: {\hyperref[\detokenize{quantities/orbital/semi_major_axis:id1}]{\sphinxcrossref{\DUrole{std,std-ref}{Semi\sphinxhyphen{}major axis}}}} \textgreater{} {\hyperref[\detokenize{quantities/children_orbit_limits/outer_rock_formation_limit:id1}]{\sphinxcrossref{\DUrole{std,std-ref}{parent outer rock formation limit}}}}.

\end{enumerate}


\subsection{Inclination}
\label{\detokenize{quantities/orbital/inclination:inclination}}\label{\detokenize{quantities/orbital/inclination::doc}}\phantomsection\label{\detokenize{quantities/orbital/inclination:id1}}
\sphinxAtStartPar
Orbital inclination is the tilt of a child object’s orbit around a celestial body.
It varies between 0 and 180. Between 0 and 90, the orbit is prograde.
Between 90 and 180, the orbit is retrograde.

\sphinxAtStartPar
However, as of now, it is cosmetic and does not determine any other characteristics,
including the {\hyperref[\detokenize{quantities/orbital/orbit_type:id1}]{\sphinxcrossref{\DUrole{std,std-ref}{orbit type}}}}.


\subsection{Argument of Periapsis}
\label{\detokenize{quantities/orbital/argument_of_periapsis:argument-of-periapsis}}\label{\detokenize{quantities/orbital/argument_of_periapsis::doc}}\phantomsection\label{\detokenize{quantities/orbital/argument_of_periapsis:id1}}
\sphinxAtStartPar
Check out \sphinxurl{https://en.wikipedia.org/wiki/Argument\_of\_periapsis}.

\sphinxAtStartPar
The argument of periapsis, as of now, it is cosmetic and does not determine any other characteristics.


\subsection{Longitude of the Ascending Node}
\label{\detokenize{quantities/orbital/longitude_of_the_ascending_node:longitude-of-the-ascending-node}}\label{\detokenize{quantities/orbital/longitude_of_the_ascending_node::doc}}\phantomsection\label{\detokenize{quantities/orbital/longitude_of_the_ascending_node:id1}}
\sphinxAtStartPar
Check out \sphinxurl{https://en.wikipedia.org/wiki/Longitude\_of\_the\_ascending\_node}.

\sphinxAtStartPar
The longitude of the ascending node, as of now, it is cosmetic and does
not determine any other characteristics.


\subsection{Semi\sphinxhyphen{}Major Axis Minimum Limit}
\label{\detokenize{quantities/orbital/semi_major_axis_minimum_limit:semi-major-axis-minimum-limit}}\label{\detokenize{quantities/orbital/semi_major_axis_minimum_limit::doc}}\phantomsection\label{\detokenize{quantities/orbital/semi_major_axis_minimum_limit:id1}}
\sphinxAtStartPar
The semi\sphinxhyphen{}major axis minimum limit is the {\hyperref[\detokenize{quantities/children_orbit_limits/roche_limit:id1}]{\sphinxcrossref{\DUrole{std,std-ref}{roche limit}}}} of the parent for the specific child
{\hyperref[\detokenize{quantities/material/density:id1}]{\sphinxcrossref{\DUrole{std,std-ref}{density}}}}, or the {\hyperref[\detokenize{quantities/children_orbit_limits/p_type_critical_orbit:p-type-critical-orbit}]{\sphinxcrossref{\DUrole{std,std-ref}{p\sphinxhyphen{}type critical limit}}}} in binary systems.


\subsection{Semi\sphinxhyphen{}Major Axis Maximum Limit}
\label{\detokenize{quantities/orbital/semi_major_axis_maximum_limit:semi-major-axis-maximum-limit}}\label{\detokenize{quantities/orbital/semi_major_axis_maximum_limit::doc}}\phantomsection\label{\detokenize{quantities/orbital/semi_major_axis_maximum_limit:id1}}
\sphinxAtStartPar
The semi\sphinxhyphen{}major axis maximum limit is the {\hyperref[\detokenize{quantities/children_orbit_limits/hill_sphere:id1}]{\sphinxcrossref{\DUrole{std,std-ref}{hill\_sphere}}}} multiplied by the
{\hyperref[\detokenize{quantities/orbital/orbit_type_factor:id1}]{\sphinxcrossref{\DUrole{std,std-ref}{orbit type factor}}}}.


\section{Children Orbit Limits}
\label{\detokenize{quantities/children_orbit_limits/children_orbit_limits:children-orbit-limits}}\label{\detokenize{quantities/children_orbit_limits/children_orbit_limits::doc}}\phantomsection\label{\detokenize{quantities/children_orbit_limits/children_orbit_limits:id1}}

\subsection{Tidal Locking Radius}
\label{\detokenize{quantities/children_orbit_limits/tidal_locking_radius:tidal-locking-radius}}\label{\detokenize{quantities/children_orbit_limits/tidal_locking_radius::doc}}\phantomsection\label{\detokenize{quantities/children_orbit_limits/tidal_locking_radius:id1}}
\sphinxAtStartPar
The tidal locking radius is the furthest distance a child would be tidally locked
to its parent. Tidal locking means that an orbiting object around a parent turns
around itself at the same amount of time it turns around its parent.
One side of the planet is always faces the parent, while the other side
does not.

\sphinxAtStartPar
The tidal locking radius is larger the more {\hyperref[\detokenize{quantities/material/mass:id1}]{\sphinxcrossref{\DUrole{std,std-ref}{massive}}}} the parent is,
and the more {\hyperref[\detokenize{quantities/life/age:id1}]{\sphinxcrossref{\DUrole{std,std-ref}{old}}}} it is. Many parameters of the child object
determine the tidal locking radius, but they are hard to estimate,
so treat this this number as a suggestion rather than a hard limit.

\sphinxAtStartPar
More info on:
\begin{enumerate}
\sphinxsetlistlabels{\arabic}{enumi}{enumii}{}{.}%
\item {} 
\sphinxAtStartPar
\sphinxurl{https://en.wikipedia.org/wiki/Tidal\_locking\#Timescale},

\item {} 
\sphinxAtStartPar
\sphinxurl{https://physics.stackexchange.com/questions/12541/tidal-lock-radius-in-habitable-zones}

\item {} 
\sphinxAtStartPar
\sphinxurl{https://www.sciencedirect.com/science/article/abs/pii/S0019103583710109} (eq. 10 in CGS units)

\end{enumerate}


\subsection{Roche Limit}
\label{\detokenize{quantities/children_orbit_limits/roche_limit:roche-limit}}\label{\detokenize{quantities/children_orbit_limits/roche_limit::doc}}\phantomsection\label{\detokenize{quantities/children_orbit_limits/roche_limit:id1}}
\sphinxAtStartPar
Roche limit is the minimum distance \(d_{Roche}\) a child object of {\hyperref[\detokenize{quantities/material/mass:id1}]{\sphinxcrossref{\DUrole{std,std-ref}{mass}}}} \(m\),
{\hyperref[\detokenize{quantities/geometric/radius:id1}]{\sphinxcrossref{\DUrole{std,std-ref}{radius}}}} \(r\) and {\hyperref[\detokenize{quantities/material/density:id1}]{\sphinxcrossref{\DUrole{std,std-ref}{density}}}} \(\rho_m\) can orbit a parent object of
{\hyperref[\detokenize{quantities/material/mass:id1}]{\sphinxcrossref{\DUrole{std,std-ref}{mass}}}} \(M\), {\hyperref[\detokenize{quantities/geometric/radius:id1}]{\sphinxcrossref{\DUrole{std,std-ref}{radius}}}} \(R\) and {\hyperref[\detokenize{quantities/material/density:id1}]{\sphinxcrossref{\DUrole{std,std-ref}{density}}}} \(\rho_M\)
before the child object breaks apart into pieces. An additional condition is that
the child object must only be held together by its own gravity.

\sphinxAtStartPar
It is given the equation: \(d_{Roche} = 2.44 r \left(\frac{M}{m}\right)^{1/3}\) or
\(d_{Roche} = 2.44 R \left(\frac{\rho_M}{\rho_m}\right)^{1/3}\).


\subsection{Dense Roche Limit}
\label{\detokenize{quantities/children_orbit_limits/dense_roche_limit:dense-roche-limit}}\label{\detokenize{quantities/children_orbit_limits/dense_roche_limit::doc}}\phantomsection\label{\detokenize{quantities/children_orbit_limits/dense_roche_limit:id1}}
\sphinxAtStartPar
Dense Roche limit is the {\hyperref[\detokenize{quantities/children_orbit_limits/roche_limit:id1}]{\sphinxcrossref{\DUrole{std,std-ref}{Roche limit}}}} of a child of unknown {\hyperref[\detokenize{quantities/geometric/radius:id1}]{\sphinxcrossref{\DUrole{std,std-ref}{radius}}}}
and a parent object of {\hyperref[\detokenize{quantities/geometric/radius:id1}]{\sphinxcrossref{\DUrole{std,std-ref}{radius}}}} \(R\) {\hyperref[\detokenize{quantities/material/density:id1}]{\sphinxcrossref{\DUrole{std,std-ref}{density}}}} 10 times bigger than the child {\hyperref[\detokenize{quantities/material/density:id1}]{\sphinxcrossref{\DUrole{std,std-ref}{density}}}}.

\sphinxAtStartPar
It is approximately: \(5.2568 R\).


\subsection{P\sphinxhyphen{}type binary Critical Orbit}
\label{\detokenize{quantities/children_orbit_limits/p_type_critical_orbit:p-type-binary-critical-orbit}}\label{\detokenize{quantities/children_orbit_limits/p_type_critical_orbit::doc}}\phantomsection\label{\detokenize{quantities/children_orbit_limits/p_type_critical_orbit:p-type-critical-orbit}}
\sphinxAtStartPar
In close binary systems (p\sphinxhyphen{}type binaries) of two stars with {\hyperref[\detokenize{quantities/material/mass:id1}]{\sphinxcrossref{\DUrole{std,std-ref}{masses}}}}
\(m_1\) and \(m_2\) with \(m_1 > m_2\), {\hyperref[\detokenize{quantities/orbital/semi_major_axis:id1}]{\sphinxcrossref{\DUrole{std,std-ref}{mean distance}}}} \(d\)
and {\hyperref[\detokenize{quantities/orbital/eccentricity:id1}]{\sphinxcrossref{\DUrole{std,std-ref}{eccentricity}}}} \(e\), the critical orbit limit
is the \sphinxstyleemphasis{minimum} distance at which an orbit stays stable.

\sphinxAtStartPar
It is estimated by: \(d(1.6 + 5.1 e - 2.22 e^2 + 4.12 \mu - 5.09 \mu^2 + 4.61 e^2  \mu^2 - 4.27 e \mu)\),
where \(\mu = \frac{m_2}{m_1 + m_2}\).

\sphinxAtStartPar
More info on: \sphinxurl{https://arxiv.org/pdf/2108.07815.pdf} (eq. 3).


\subsection{Rough Inner Orbit Limit}
\label{\detokenize{quantities/children_orbit_limits/rough_inner_orbit_limit:rough-inner-orbit-limit}}\label{\detokenize{quantities/children_orbit_limits/rough_inner_orbit_limit::doc}}
\sphinxAtStartPar
The rough inner limit is an orbit limit given in \sphinxhref{https://youtu.be/wsuejZRqus4}{artifexian’s video}.
For a {\hyperref[\detokenize{celestial_bodies/star:id1}]{\sphinxcrossref{\DUrole{std,std-ref}{star}}}} similar to our sun, it is similar to the {\hyperref[\detokenize{quantities/children_orbit_limits/dense_roche_limit:id1}]{\sphinxcrossref{\DUrole{std,std-ref}{dense Roche limit}}}}
and the {\hyperref[\detokenize{quantities/children_orbit_limits/outer_rock_formation_limit:id1}]{\sphinxcrossref{\DUrole{std,std-ref}{rock formaiton line}}}}.


\subsection{Inner Orbit Limit}
\label{\detokenize{quantities/children_orbit_limits/inner_orbit_limit:inner-orbit-limit}}\label{\detokenize{quantities/children_orbit_limits/inner_orbit_limit::doc}}\phantomsection\label{\detokenize{quantities/children_orbit_limits/inner_orbit_limit:id1}}
\sphinxAtStartPar
The inner orbit limit of a parent for its children is the highest of
{\hyperref[\detokenize{quantities/children_orbit_limits/dense_roche_limit:id1}]{\sphinxcrossref{\DUrole{std,std-ref}{dense Roche limit}}}} and {\hyperref[\detokenize{quantities/children_orbit_limits/p_type_critical_orbit:p-type-critical-orbit}]{\sphinxcrossref{\DUrole{std,std-ref}{P\sphinxhyphen{}type critical orbit}}}}.
It is provided to assist the user determine the worst possible inner orbit limit.


\subsection{Hill Sphere}
\label{\detokenize{quantities/children_orbit_limits/hill_sphere:hill-sphere}}\label{\detokenize{quantities/children_orbit_limits/hill_sphere::doc}}\phantomsection\label{\detokenize{quantities/children_orbit_limits/hill_sphere:id1}}
\sphinxAtStartPar
Hill sphere is the maximum distance (\(r_H\)) an object (child) can affect even smaller body (grandchild),
because of the presence of a bigger body around (parent).
For example the moon is within the hill sphere of earth, and the hill sphere of earth is determined
by earth’s {\hyperref[\detokenize{quantities/material/mass:id1}]{\sphinxcrossref{\DUrole{std,std-ref}{mass}}}} \(m\) {\hyperref[\detokenize{quantities/orbital/semi_major_axis:id1}]{\sphinxcrossref{\DUrole{std,std-ref}{semi\sphinxhyphen{}major axis}}}} \(a\),
and {\hyperref[\detokenize{quantities/orbital/eccentricity:id1}]{\sphinxcrossref{\DUrole{std,std-ref}{eccentricity}}}} \(e\), and the sun’s {\hyperref[\detokenize{quantities/material/mass:id1}]{\sphinxcrossref{\DUrole{std,std-ref}{mass}}}} \(M\).

\sphinxAtStartPar
It is given by the equation: \(r_H = a (1 - e) \left(\frac{m}{3 M}\right)^{1/3}\).

\sphinxAtStartPar
If we looking at a single object that is part of a binary system, a more accurate determination
of the Hill sphere is given by the {\hyperref[\detokenize{quantities/orbital/roche_lobe:id1}]{\sphinxcrossref{\DUrole{std,std-ref}{Roche lobe}}}}.


\subsection{Roche Lobe}
\label{\detokenize{quantities/children_orbit_limits/roche_lobe:roche-lobe}}\label{\detokenize{quantities/children_orbit_limits/roche_lobe::doc}}\phantomsection\label{\detokenize{quantities/children_orbit_limits/roche_lobe:id1}}
\sphinxAtStartPar
Roche lobe (or sphere) (not to be confused with {\hyperref[\detokenize{quantities/children_orbit_limits/roche_limit:id1}]{\sphinxcrossref{\DUrole{std,std-ref}{Roche limit}}}}) is the region
around a star in a binary system within which orbiting material is gravitationally bound to that star.

\sphinxAtStartPar
For an object of {\hyperref[\detokenize{quantities/material/mass:id1}]{\sphinxcrossref{\DUrole{std,std-ref}{mass}}}} \(m\) in a binary system, and a companion object of
{\hyperref[\detokenize{quantities/material/mass:id1}]{\sphinxcrossref{\DUrole{std,std-ref}{mass}}}} \(m_c\), with a binary  {\hyperref[\detokenize{quantities/orbital/semi_major_axis:id1}]{\sphinxcrossref{\DUrole{std,std-ref}{mean distance}}}} \(d\)
and {\hyperref[\detokenize{quantities/orbital/eccentricity:id1}]{\sphinxcrossref{\DUrole{std,std-ref}{eccentricity}}}} \(e\), the roche lobe is approximated by:
\(d (1 - e) \frac{0.49 \mu^{2/3}}{0.6\mu^{2/3} + \ln{\left(1 + \mu^{1/3}\right)}}\), where
\(\mu = \frac{m}{m_c}\).


\subsection{S\sphinxhyphen{}type binary Critical Orbit}
\label{\detokenize{quantities/children_orbit_limits/s_type_critical_orbit:s-type-binary-critical-orbit}}\label{\detokenize{quantities/children_orbit_limits/s_type_critical_orbit::doc}}\phantomsection\label{\detokenize{quantities/children_orbit_limits/s_type_critical_orbit:s-type-critical-orbit}}
\sphinxAtStartPar
In wide binary systems (S\sphinxhyphen{}type binaries) of two objects with {\hyperref[\detokenize{quantities/material/mass:id1}]{\sphinxcrossref{\DUrole{std,std-ref}{masses}}}}
\(m_1\) and \(m_2\) with \(m_1 > m_2\), {\hyperref[\detokenize{quantities/orbital/semi_major_axis:id1}]{\sphinxcrossref{\DUrole{std,std-ref}{mean distance}}}} \(d\)
and {\hyperref[\detokenize{quantities/orbital/eccentricity:id1}]{\sphinxcrossref{\DUrole{std,std-ref}{eccentricity}}}} \(e\), the critical orbit limit is the \sphinxstyleemphasis{maximum}
distance at which an orbit stays stable around a single object.
This distance is different for each object and changes due to their mass difference.

\sphinxAtStartPar
For the primary star of {\hyperref[\detokenize{quantities/material/mass:id1}]{\sphinxcrossref{\DUrole{std,std-ref}{mass}}}} \(m_1\), \(\mu\) is determined
by the companions to the binary’s mass ratio \(\mu = \frac{m_2}{m_1 + m_2}\).

\sphinxAtStartPar
For the secondary star of {\hyperref[\detokenize{quantities/material/mass:id1}]{\sphinxcrossref{\DUrole{std,std-ref}{mass}}}} \(m_2\), \(\mu\) is determined
by the companions to the binary’s mass ratio \(\mu = \frac{m_1}{m_1 + m_2}\).

\sphinxAtStartPar
The critical distance is estimated by: \(d (0.464 - 0.38 \mu - 0.631 e + 0.586 \mu e + 0.15 e^2 - 0.198 \mu e^2)\).

\sphinxAtStartPar
More info on: \sphinxurl{https://arxiv.org/pdf/2108.07815.pdf} (eq. 1).


\subsection{Rough Outer Orbit Limit}
\label{\detokenize{quantities/children_orbit_limits/rough_outer_orbit_limit:rough-outer-orbit-limit}}\label{\detokenize{quantities/children_orbit_limits/rough_outer_orbit_limit::doc}}
\sphinxAtStartPar
The rough inner limit is an orbit limit given in \sphinxhref{https://youtu.be/wsuejZRqus4}{artifexian’s video}.
For a {\hyperref[\detokenize{celestial_bodies/star:id1}]{\sphinxcrossref{\DUrole{std,std-ref}{star}}}} similar to our sun, it is similar to the {\hyperref[\detokenize{quantities/orbital/semi_major_axis:id1}]{\sphinxcrossref{\DUrole{std,std-ref}{semi\sphinxhyphen{}major axis}}}}
of pluto.


\subsection{Outer Orbit Limit}
\label{\detokenize{quantities/children_orbit_limits/outer_orbit_limit:outer-orbit-limit}}\label{\detokenize{quantities/children_orbit_limits/outer_orbit_limit::doc}}\phantomsection\label{\detokenize{quantities/children_orbit_limits/outer_orbit_limit:id1}}
\sphinxAtStartPar
The outer orbit limit of a parent for its children is the lowest of
{\hyperref[\detokenize{quantities/children_orbit_limits/hill_sphere:id1}]{\sphinxcrossref{\DUrole{std,std-ref}{Hill Sphere}}}} and {\hyperref[\detokenize{quantities/children_orbit_limits/s_type_critical_orbit:s-type-critical-orbit}]{\sphinxcrossref{\DUrole{std,std-ref}{S\sphinxhyphen{}type critical orbit}}}}.
It is provided to assist the user determine the worst possible outer orbit limit.


\subsection{Inner Rock Formation Limit}
\label{\detokenize{quantities/children_orbit_limits/inner_rock_formation_limit:inner-rock-formation-limit}}\label{\detokenize{quantities/children_orbit_limits/inner_rock_formation_limit::doc}}\phantomsection\label{\detokenize{quantities/children_orbit_limits/inner_rock_formation_limit:id1}}
\sphinxAtStartPar
We use {\hyperref[\detokenize{quantities/insolation_models/selsis/selsis:selsis-insolation-model}]{\sphinxcrossref{\DUrole{std,std-ref}{Selsis insolation model}}}} since it allows for easy, Solar system comparisons.
Rock line is the distance at which iron and rock can form clusters, planetesimals and eventually planets.
Since the rock line is determined by when rock and iron are more or less solid, I decided to use
the boiling point of a fast rotating iron ball
({\hyperref[\detokenize{quantities/surface/emission/heat_distribution:id1}]{\sphinxcrossref{\DUrole{std,std-ref}{heating distribution}}}} \(\beta\) 1, {\hyperref[\detokenize{quantities/surface/emission/albedo:id1}]{\sphinxcrossref{\DUrole{std,std-ref}{albedo}}}} \(A\) 0.15)
@ \(T = 2870\) K.
for the optimistic inner rock line limit, giving a value of \(\approx 0.087\) A.U for our sun.

\sphinxAtStartPar
Distance estimation: \(\left(\frac{T}{T_{\rm eff}}\right) ^ 2 \sqrt{\frac{1-A}{\beta}}\),
with \(T_{\rm eff} = 278.5\) K (\sphinxurl{https://arxiv.org/pdf/1702.07314.pdf} eq. 6 with \(L = 1\)).


\subsection{Outer Rock Formation Limit}
\label{\detokenize{quantities/children_orbit_limits/outer_rock_formation_limit:outer-rock-formation-limit}}\label{\detokenize{quantities/children_orbit_limits/outer_rock_formation_limit::doc}}\phantomsection\label{\detokenize{quantities/children_orbit_limits/outer_rock_formation_limit:id1}}
\sphinxAtStartPar
We use {\hyperref[\detokenize{quantities/insolation_models/selsis/selsis:selsis-insolation-model}]{\sphinxcrossref{\DUrole{std,std-ref}{Selsis insolation model}}}} since it allows for easy, Solar system comparisons.
Rock line is the distance at which iron and rock can form clusters, planetesimals and eventually planets.
Since the rock line is determined by when rock and iron are more or less solid, I decided to use
the melting point of a slow rotating ({\hyperref[\detokenize{quantities/children_orbit_limits/tidal_locking_radius:id1}]{\sphinxcrossref{\DUrole{std,std-ref}{tidally locked}}}}) rock ball
({\hyperref[\detokenize{quantities/surface/emission/heat_distribution:id1}]{\sphinxcrossref{\DUrole{std,std-ref}{heating distribution}}}} \(\beta\) 0.5, {\hyperref[\detokenize{quantities/surface/emission/albedo:id1}]{\sphinxcrossref{\DUrole{std,std-ref}{albedo}}}} \(A\) 0.85)
@ \(T = 600\) K to find the equivalent solar system distance
and multiply by 5/3.1 (similar to the {\hyperref[\detokenize{quantities/children_orbit_limits/outer_water_frost_limit:id1}]{\sphinxcrossref{\DUrole{std,std-ref}{early solar system water frost line}}}})
(lowest temperature from \sphinxurl{http://hyperphysics.phy-astr.gsu.edu/hbase/Geophys/meltrock.html})
for the optimistic outer rock line limit, giving a value of \(\approx 0.281\) A.U for our sun.

\sphinxAtStartPar
Distance estimation: \(\frac{5}{3.1}\left(\frac{T}{T_{\rm eff}}\right) ^ 2 \sqrt{\frac{1-A}{\beta}}\),
with \(T_{\rm eff} = 278.5\) K (\sphinxurl{https://arxiv.org/pdf/1702.07314.pdf} eq. 6 with \(L = 1\)).


\subsection{Inner Water Frost Limit}
\label{\detokenize{quantities/children_orbit_limits/inner_water_frost_limit:inner-water-frost-limit}}\label{\detokenize{quantities/children_orbit_limits/inner_water_frost_limit::doc}}\phantomsection\label{\detokenize{quantities/children_orbit_limits/inner_water_frost_limit:id1}}
\sphinxAtStartPar
We use {\hyperref[\detokenize{quantities/insolation_models/selsis/selsis:selsis-insolation-model}]{\sphinxcrossref{\DUrole{std,std-ref}{Selsis insolation model}}}} since it allows for easy, Solar system comparisons.
As shown on the wikipedia page \sphinxurl{https://en.wikipedia.org/wiki/Frost\_line\_(astrophysics}), there are different
frost lines for different compounds. Water is important and seems to determine the line between gas planets and
rocky planets.

\sphinxAtStartPar
It must be noted that gas giants can migrate to inner orbits after their creation,
so it is not impossible to find one where venus or mercury would have been. It would however
probably destabilize any other planets in its path.

\sphinxAtStartPar
The inner limit is taken from wiki’s suggestion for big sized bodies at \textasciitilde{}1.94 AU (\textasciitilde{}200 K).


\subsection{Sol\sphinxhyphen{}Equivalent Water Frost Limit}
\label{\detokenize{quantities/children_orbit_limits/sol_equivalent_water_frost_limit:sol-equivalent-water-frost-limit}}\label{\detokenize{quantities/children_orbit_limits/sol_equivalent_water_frost_limit::doc}}\phantomsection\label{\detokenize{quantities/children_orbit_limits/sol_equivalent_water_frost_limit:id1}}
\sphinxAtStartPar
We use {\hyperref[\detokenize{quantities/insolation_models/selsis/selsis:selsis-insolation-model}]{\sphinxcrossref{\DUrole{std,std-ref}{Selsis insolation model}}}} since it allows for easy, Solar system comparisons.
As shown on the wikipedia page \sphinxurl{https://en.wikipedia.org/wiki/Frost\_line\_(astrophysics}), there are different
frost lines for different compounds. Water is important and seems to determine the line between gas planets and
rocky planets.

\sphinxAtStartPar
It must be noted that gas giants can migrate to inner orbits after their creation,
so it is not impossible to find one where venus or mercury would have been. It would however
probably destabilize any other planets in its path.

\sphinxAtStartPar
The Sol equivalent limit is from the average of the newest finds \textasciitilde{} 3.1 AU (\textasciitilde{} 158.2 K).


\subsection{Outer Water Frost Limit}
\label{\detokenize{quantities/children_orbit_limits/outer_water_frost_limit:outer-water-frost-limit}}\label{\detokenize{quantities/children_orbit_limits/outer_water_frost_limit::doc}}\phantomsection\label{\detokenize{quantities/children_orbit_limits/outer_water_frost_limit:id1}}
\sphinxAtStartPar
We use {\hyperref[\detokenize{quantities/insolation_models/selsis/selsis:selsis-insolation-model}]{\sphinxcrossref{\DUrole{std,std-ref}{Selsis insolation model}}}} since it allows for easy, Solar system comparisons.
As shown on the wikipedia page \sphinxurl{https://en.wikipedia.org/wiki/Frost\_line\_(astrophysics}), there are different
frost lines for different compounds. Water is important and seems to determine the line between gas planets and
rocky planets.

\sphinxAtStartPar
It must be noted that gas giants can migrate to inner orbits after their creation,
so it is not impossible to find one where venus or mercury would have been. It would however
probably destabilize any other planets in its path.

\sphinxAtStartPar
The outer limit is taken from wiki’s mention on the early\sphinxhyphen{}days frost line at 5 AU (\textasciitilde{}124.5 K).


\section{Insolation Models}
\label{\detokenize{quantities/insolation_models/insolation_models:insolation-models}}\label{\detokenize{quantities/insolation_models/insolation_models::doc}}\phantomsection\label{\detokenize{quantities/insolation_models/insolation_models:id1}}
\sphinxAtStartPar
Insolation or effective stellar flux is the effective flux that
reaches a specific {\hyperref[\detokenize{quantities/orbital/orbital:id1}]{\sphinxcrossref{\DUrole{std,std-ref}{orbital}}}} distance, called threshold (or limit).
Insolation changes with the {\hyperref[\detokenize{celestial_bodies/star:id1}]{\sphinxcrossref{\DUrole{std,std-ref}{star’s}}}} {\hyperref[\detokenize{quantities/surface/emission/temperature:id1}]{\sphinxcrossref{\DUrole{std,std-ref}{temperature}}}},
as well as the environmental conditions of the target {\hyperref[\detokenize{quantities/habitability/habitability:id1}]{\sphinxcrossref{\DUrole{std,std-ref}{habitable}}}} world.
We use insolation for a specific climate to normalize the
{\hyperref[\detokenize{quantities/surface/emission/luminosity:id1}]{\sphinxcrossref{\DUrole{std,std-ref}{luminosity}}}} of a {\hyperref[\detokenize{celestial_bodies/star:id1}]{\sphinxcrossref{\DUrole{std,std-ref}{star’s}}}},
and try to estimate the threshold at which distance from
the {\hyperref[\detokenize{celestial_bodies/star:id1}]{\sphinxcrossref{\DUrole{std,std-ref}{star’s}}}} the aforementioned environmental conditions occur.
By using extreme environmental conditions that could potentially support life,
we can determine minima and maxima for {\hyperref[\detokenize{quantities/habitability/habitable_zones/habitable_zones:id1}]{\sphinxcrossref{\DUrole{std,std-ref}{zones of habitability}}}}
around single\sphinxhyphen{} or multi\sphinxhyphen{}star {\hyperref[\detokenize{celestial_systems/celestial_systems:id1}]{\sphinxcrossref{\DUrole{std,std-ref}{systems}}}}.

\sphinxAtStartPar
There are different types of insolation models. In this package,
we are using one that was designed by {\hyperref[\detokenize{quantities/insolation_models/kopparapu/kopparapu:kopparapu-insolation-model}]{\sphinxcrossref{\DUrole{std,std-ref}{Kopparapu}}}}, and
one that was designed by {\hyperref[\detokenize{quantities/insolation_models/selsis/selsis:selsis-insolation-model}]{\sphinxcrossref{\DUrole{std,std-ref}{Selsis}}}}.

\sphinxAtStartPar
These models have multiple different thresholds, from which
we only use a handful that are representative of the {\hyperref[\detokenize{quantities/habitability/habitable_zones/habitable_zones:id1}]{\sphinxcrossref{\DUrole{std,std-ref}{habitability limits}}}}.
These are designated as \sphinxstyleemphasis{earth equivalent}, \sphinxstyleemphasis{conservative} or \sphinxstyleemphasis{relaxed} and \sphinxstyleemphasis{minimum} (inner) or \sphinxstyleemphasis{maximum} (outer).


\subsection{Kopparapu}
\label{\detokenize{quantities/insolation_models/kopparapu/kopparapu:kopparapu}}\label{\detokenize{quantities/insolation_models/kopparapu/kopparapu::doc}}\phantomsection\label{\detokenize{quantities/insolation_models/kopparapu/kopparapu:kopparapu-insolation-model}}
\sphinxAtStartPar
Kopparapu’s insolation model is an elegant model that describes seven
disparate thresholds of interest.
It’s main asset is the distinction of thresholds between different planetary
masses for the runaway greenhouse effect.

\sphinxAtStartPar
The provided thresholds are:
\begin{enumerate}
\sphinxsetlistlabels{\arabic}{enumi}{enumii}{}{.}%
\item {} 
\sphinxAtStartPar
Recent Venus: \sphinxstyleemphasis{inner}, {\hyperref[\detokenize{quantities/insolation_models/relaxed_minimum_limit:id1}]{\sphinxcrossref{\DUrole{std,std-ref}{relaxed minimum limit}}}}.

\item {} 
\sphinxAtStartPar
Runaway Greenhouse Effect, Subterran: \sphinxstyleemphasis{inner}.

\item {} 
\sphinxAtStartPar
Runaway Greenhouse Effect, Terran: \sphinxstyleemphasis{inner}, {\hyperref[\detokenize{quantities/insolation_models/conservative_minimum_limit:id1}]{\sphinxcrossref{\DUrole{std,std-ref}{conservative minimum limit}}}}.

\item {} 
\sphinxAtStartPar
Runaway Greenhouse Effect, Superterran: \sphinxstyleemphasis{inner}.

\item {} 
\sphinxAtStartPar
Moist Greenhouse Effect: \sphinxstyleemphasis{inner}, {\hyperref[\detokenize{quantities/insolation_models/earth_equivalent_limit:id1}]{\sphinxcrossref{\DUrole{std,std-ref}{earth equivalent limit}}}}.

\item {} 
\sphinxAtStartPar
Maximum Greenhouse Effect: \sphinxstyleemphasis{outer}, {\hyperref[\detokenize{quantities/insolation_models/conservative_maximum_limit:id1}]{\sphinxcrossref{\DUrole{std,std-ref}{conservative maximum limit}}}}.

\item {} 
\sphinxAtStartPar
Early Mars: \sphinxstyleemphasis{outer}, {\hyperref[\detokenize{quantities/insolation_models/relaxed_maximum_limit:id1}]{\sphinxcrossref{\DUrole{std,std-ref}{relaxed maximum limit}}}}.

\end{enumerate}

\sphinxAtStartPar
Sources: Kopparapu. et al. 2013 {[}1{]}, 2014 {[}2{]}, Wang and Cuntz 2019 {[}3{]}
\begin{enumerate}
\sphinxsetlistlabels{\arabic}{enumi}{enumii}{}{.}%
\item {} 
\sphinxAtStartPar
\sphinxurl{https://iopscience.iop.org/article/10.1088/0004-637X/765/2/131}

\item {} 
\sphinxAtStartPar
\sphinxurl{https://iopscience.iop.org/article/10.1088/2041-8205/787/2/L29}

\item {} 
\sphinxAtStartPar
\sphinxurl{https://iopscience.iop.org/article/10.3847/1538-4357/ab0377} (overview of this and other models)

\end{enumerate}


\subsection{Selsis}
\label{\detokenize{quantities/insolation_models/selsis/selsis:selsis}}\label{\detokenize{quantities/insolation_models/selsis/selsis::doc}}\phantomsection\label{\detokenize{quantities/insolation_models/selsis/selsis:selsis-insolation-model}}
\sphinxAtStartPar
Selsis’ insolation model is a simple model that describes a multitude of
disparate thresholds of interest.
It’s main assets are the distinction cloudy and non\sphinxhyphen{}cloudy greenhouse effect\sphinxhyphen{}based thresholds,
the provision of very relaxed thresholds, and the ability to make once own thresholds. This last part
is important in determining the {\hyperref[\detokenize{quantities/children_orbit_limits/inner_rock_formation_limit:id1}]{\sphinxcrossref{\DUrole{std,std-ref}{inner rock formation limit}}}},
the {\hyperref[\detokenize{quantities/children_orbit_limits/outer_rock_formation_limit:id1}]{\sphinxcrossref{\DUrole{std,std-ref}{outer rock formation limit}}}},
the {\hyperref[\detokenize{quantities/children_orbit_limits/inner_water_frost_limit:id1}]{\sphinxcrossref{\DUrole{std,std-ref}{inner water frost limit}}}},
the {\hyperref[\detokenize{quantities/children_orbit_limits/sol_equivalent_water_frost_limit:id1}]{\sphinxcrossref{\DUrole{std,std-ref}{sol\sphinxhyphen{}equivalent water frost limit}}}},
and the {\hyperref[\detokenize{quantities/children_orbit_limits/outer_water_frost_limit:id1}]{\sphinxcrossref{\DUrole{std,std-ref}{outer water frost limit}}}}.

\sphinxAtStartPar
The provided thresholds are:
\begin{enumerate}
\sphinxsetlistlabels{\arabic}{enumi}{enumii}{}{.}%
\item {} \begin{description}
\item[{Planet\sphinxhyphen{}based}] \leavevmode\begin{enumerate}
\sphinxsetlistlabels{\alph}{enumii}{enumiii}{}{.}%
\item {} 
\sphinxAtStartPar
Recent Venus: \sphinxstyleemphasis{inner}.

\item {} 
\sphinxAtStartPar
Earth Equivalent’: \sphinxstyleemphasis{inner}, {\hyperref[\detokenize{quantities/insolation_models/earth_equivalent_limit:id1}]{\sphinxcrossref{\DUrole{std,std-ref}{earth equivalent limit}}}}.

\item {} 
\sphinxAtStartPar
Early Mars: \sphinxstyleemphasis{outer}.

\end{enumerate}

\end{description}

\item {} \begin{description}
\item[{0\% Clouds}] \leavevmode\begin{enumerate}
\sphinxsetlistlabels{\alph}{enumii}{enumiii}{}{.}%
\item {} 
\sphinxAtStartPar
Runaway Greenhouse Effect, 0\% Clouds: \sphinxstyleemphasis{inner}, {\hyperref[\detokenize{quantities/insolation_models/conservative_minimum_limit:id1}]{\sphinxcrossref{\DUrole{std,std-ref}{conservative minimum limit}}}}.

\item {} 
\sphinxAtStartPar
Start of water loss, 0\% Clouds: \sphinxstyleemphasis{inner}.

\item {} 
\sphinxAtStartPar
First C02 Condensation, 0\% Clouds: \sphinxstyleemphasis{outer}.

\item {} 
\sphinxAtStartPar
Maximum Greenhouse Effect, 0\% Clouds: \sphinxstyleemphasis{outer}, {\hyperref[\detokenize{quantities/insolation_models/conservative_maximum_limit:id1}]{\sphinxcrossref{\DUrole{std,std-ref}{conservative maximum limit}}}}.

\end{enumerate}

\end{description}

\item {} \begin{description}
\item[{50\% Clouds}] \leavevmode\begin{enumerate}
\sphinxsetlistlabels{\alph}{enumii}{enumiii}{}{.}%
\item {} 
\sphinxAtStartPar
Runaway Greenhouse Effect, 50\% Clouds: \sphinxstyleemphasis{inner}.

\item {} 
\sphinxAtStartPar
Start of water loss, 50\% Clouds: \sphinxstyleemphasis{inner}.

\item {} 
\sphinxAtStartPar
Maximum Greenhouse Effect, 50\% Clouds: \sphinxstyleemphasis{outer}.

\end{enumerate}

\end{description}

\item {} \begin{description}
\item[{100\% Clouds}] \leavevmode\begin{enumerate}
\sphinxsetlistlabels{\alph}{enumii}{enumiii}{}{.}%
\item {} 
\sphinxAtStartPar
Runaway Greenhouse Effect, 100\% Clouds: \sphinxstyleemphasis{inner}, {\hyperref[\detokenize{quantities/insolation_models/relaxed_minimum_limit:id1}]{\sphinxcrossref{\DUrole{std,std-ref}{relaxed minimum limit}}}}.

\item {} 
\sphinxAtStartPar
Start of water loss, 100\% Clouds: \sphinxstyleemphasis{inner}.

\item {} 
\sphinxAtStartPar
Maximum Greenhouse Effect, 100\% Clouds: \sphinxstyleemphasis{outer}, {\hyperref[\detokenize{quantities/insolation_models/relaxed_maximum_limit:id1}]{\sphinxcrossref{\DUrole{std,std-ref}{relaxed maximum limit}}}}.

\end{enumerate}

\end{description}

\end{enumerate}

\sphinxAtStartPar
Sources: Selsis. et al. 2007 {[}1{]}, Wang and Cuntz 2019 {[}2{]}
1. \sphinxurl{https://www.aanda.org/articles/aa/pdf/2007/48}
2. \sphinxurl{https://iopscience.iop.org/article/10.3847/1538-4357/ab0377} (overview of this and others models)


\subsection{Relaxed Minimum Limit}
\label{\detokenize{quantities/insolation_models/relaxed_minimum_limit:relaxed-minimum-limit}}\label{\detokenize{quantities/insolation_models/relaxed_minimum_limit::doc}}\phantomsection\label{\detokenize{quantities/insolation_models/relaxed_minimum_limit:id1}}
\sphinxAtStartPar
Relaxed minimum limit is the smallest inner limit of an {\hyperref[\detokenize{quantities/insolation_models/insolation_models:id1}]{\sphinxcrossref{\DUrole{std,std-ref}{insolation model}}}}.


\subsection{Relaxed Maximum Limit}
\label{\detokenize{quantities/insolation_models/relaxed_maximum_limit:relaxed-maximum-limit}}\label{\detokenize{quantities/insolation_models/relaxed_maximum_limit::doc}}\phantomsection\label{\detokenize{quantities/insolation_models/relaxed_maximum_limit:id1}}
\sphinxAtStartPar
Relaxed maximum limit is the furthest outer limit of an {\hyperref[\detokenize{quantities/insolation_models/insolation_models:id1}]{\sphinxcrossref{\DUrole{std,std-ref}{insolation model}}}}.


\subsection{Conservative Minimum Limit}
\label{\detokenize{quantities/insolation_models/conservative_minimum_limit:conservative-minimum-limit}}\label{\detokenize{quantities/insolation_models/conservative_minimum_limit::doc}}\phantomsection\label{\detokenize{quantities/insolation_models/conservative_minimum_limit:id1}}
\sphinxAtStartPar
Conservative minimum limit is the furthest inner limit of an {\hyperref[\detokenize{quantities/insolation_models/insolation_models:id1}]{\sphinxcrossref{\DUrole{std,std-ref}{insolation model}}}}.


\subsection{Conservative Maximum Limit}
\label{\detokenize{quantities/insolation_models/conservative_maximum_limit:conservative-maximum-limit}}\label{\detokenize{quantities/insolation_models/conservative_maximum_limit::doc}}\phantomsection\label{\detokenize{quantities/insolation_models/conservative_maximum_limit:id1}}
\sphinxAtStartPar
Conservative maximum limit is the smallest outer limit of an {\hyperref[\detokenize{quantities/insolation_models/insolation_models:id1}]{\sphinxcrossref{\DUrole{std,std-ref}{insolation model}}}}.


\subsection{Earth\sphinxhyphen{}Equivalent Limit}
\label{\detokenize{quantities/insolation_models/earth_equivalent_limit:earth-equivalent-limit}}\label{\detokenize{quantities/insolation_models/earth_equivalent_limit::doc}}\phantomsection\label{\detokenize{quantities/insolation_models/earth_equivalent_limit:id1}}
\sphinxAtStartPar
Earth\sphinxhyphen{}equivalent limit is the limit that closely matches the distance of the earth
from the sun of an {\hyperref[\detokenize{quantities/insolation_models/insolation_models:id1}]{\sphinxcrossref{\DUrole{std,std-ref}{insolation model}}}}.


\section{Habitability}
\label{\detokenize{quantities/habitability/habitability:habitability}}\label{\detokenize{quantities/habitability/habitability::doc}}\phantomsection\label{\detokenize{quantities/habitability/habitability:id1}}
\sphinxAtStartPar
Habitability is a label given to stars or planets that satisfy certain criteria.
These criteria are different for ({\hyperref[\detokenize{celestial_systems/binary_system:id1}]{\sphinxcrossref{\DUrole{std,std-ref}{binary}}}}) {\hyperref[\detokenize{celestial_bodies/star:id1}]{\sphinxcrossref{\DUrole{std,std-ref}{stars}}}},
{\hyperref[\detokenize{celestial_bodies/planet:id1}]{\sphinxcrossref{\DUrole{std,std-ref}{planets}}}} and {\hyperref[\detokenize{celestial_bodies/satellite:id1}]{\sphinxcrossref{\DUrole{std,std-ref}{satellites}}}} and {\hyperref[\detokenize{celestial_bodies/trojan_satellite:id1}]{\sphinxcrossref{\DUrole{std,std-ref}{trojan satellites}}}}.


\subsection{Habitable Zones}
\label{\detokenize{quantities/habitability/habitable_zones/habitable_zones:habitable-zones}}\label{\detokenize{quantities/habitability/habitable_zones/habitable_zones::doc}}\phantomsection\label{\detokenize{quantities/habitability/habitable_zones/habitable_zones:id1}}
\sphinxAtStartPar
Habitable zone (HZ) is defined as an area between two orbital distances from a parent star or binary system
that life can exist. The extremes of the zone are probably not as hospitable to complex organisms
like ourselves, but you never know…

\sphinxAtStartPar
Depending on the system type, the habitable zone is estimated in different ways.
They always depend on the {\hyperref[\detokenize{quantities/surface/emission/luminosity:id1}]{\sphinxcrossref{\DUrole{std,std-ref}{luminosity}}}} and {\hyperref[\detokenize{quantities/surface/emission/temperature:id1}]{\sphinxcrossref{\DUrole{std,std-ref}{temperature}}}} of the star(s),
and often on the orbital {\hyperref[\detokenize{quantities/orbital/eccentricity:id1}]{\sphinxcrossref{\DUrole{std,std-ref}{eccentricity}}}} of the potential habitable world.
The single star habitable zone or {\hyperref[\detokenize{quantities/habitability/habitable_zones/single_star_habitable_zone:id1}]{\sphinxcrossref{\DUrole{std,std-ref}{SSHZ}}}} is the simplest type of
HZ. For binary stars, there are three different types,
the radiative habitable zone or {\hyperref[\detokenize{quantities/habitability/habitable_zones/radiative_habitable_zone:id1}]{\sphinxcrossref{\DUrole{std,std-ref}{RHZ}}}},
the permanent habitable zone or {\hyperref[\detokenize{quantities/habitability/habitable_zones/permanent_habitable_zone:id1}]{\sphinxcrossref{\DUrole{std,std-ref}{PHZ}}}},
the average habitable zone or {\hyperref[\detokenize{quantities/habitability/habitable_zones/average_habitable_zone:id1}]{\sphinxcrossref{\DUrole{std,std-ref}{AHZ}}}}.


\subsubsection{Single Star Habitable Zone}
\label{\detokenize{quantities/habitability/habitable_zones/single_star_habitable_zone:single-star-habitable-zone}}\label{\detokenize{quantities/habitability/habitable_zones/single_star_habitable_zone::doc}}\phantomsection\label{\detokenize{quantities/habitability/habitable_zones/single_star_habitable_zone:id1}}
\sphinxAtStartPar
The Single Star Habitable Zone (SSHZ) is the simplest type of habitable zone.

\sphinxAtStartPar
For a single {\hyperref[\detokenize{celestial_bodies/star:id1}]{\sphinxcrossref{\DUrole{std,std-ref}{star}}}} of {\hyperref[\detokenize{quantities/surface/emission/luminosity:id1}]{\sphinxcrossref{\DUrole{std,std-ref}{luminosity}}}} \(L\) and
{\hyperref[\detokenize{quantities/surface/emission/temperature:id1}]{\sphinxcrossref{\DUrole{std,std-ref}{temperature}}}} \(T\), the habitable thresholds \(r_x\) (where
\(x = {\rm inner, \, outer}\)) with insolation
\(S_x(T)\) are estimated by:
\(r_x = \sqrt{\frac{L}{S_x(T)}}\).


\subsubsection{Radiative Habitable Zone}
\label{\detokenize{quantities/habitability/habitable_zones/radiative_habitable_zone:radiative-habitable-zone}}\label{\detokenize{quantities/habitability/habitable_zones/radiative_habitable_zone::doc}}\phantomsection\label{\detokenize{quantities/habitability/habitable_zones/radiative_habitable_zone:id1}}
\sphinxAtStartPar
The Radiative Habitable Zone (SSHZ) is the simplest type of habitable zone in a {\hyperref[\detokenize{celestial_systems/binary_system:id1}]{\sphinxcrossref{\DUrole{std,std-ref}{binary system}}}}.
It does \sphinxstyleemphasis{not} take into account the forced and periodically changing {\hyperref[\detokenize{quantities/orbital/eccentricity:id1}]{\sphinxcrossref{\DUrole{std,std-ref}{eccentricity}}}}
of the {\hyperref[\detokenize{celestial_bodies/planet:id1}]{\sphinxcrossref{\DUrole{std,std-ref}{planets}}}} due to the asymmetry
the two {\hyperref[\detokenize{celestial_bodies/star:id1}]{\sphinxcrossref{\DUrole{std,std-ref}{stars}}}} of different {\hyperref[\detokenize{quantities/material/mass:id1}]{\sphinxcrossref{\DUrole{std,std-ref}{mass}}}} introduce to the {\hyperref[\detokenize{celestial_systems/stellar_system:id1}]{\sphinxcrossref{\DUrole{std,std-ref}{stellar system}}}}.

\sphinxAtStartPar
Let us assume a {\hyperref[\detokenize{celestial_systems/binary_system:id1}]{\sphinxcrossref{\DUrole{std,std-ref}{binary system}}}} with {\hyperref[\detokenize{celestial_bodies/star:id1}]{\sphinxcrossref{\DUrole{std,std-ref}{stars}}}} \(s = A,\, B\)
of {\hyperref[\detokenize{quantities/surface/emission/luminosity:id1}]{\sphinxcrossref{\DUrole{std,std-ref}{luminosities}}}} \(L_s\) and
{\hyperref[\detokenize{quantities/surface/emission/temperature:id1}]{\sphinxcrossref{\DUrole{std,std-ref}{temperatures}}}} \(T_s\).
They are separated by a mean {\hyperref[\detokenize{quantities/orbital/semi_major_axis:id1}]{\sphinxcrossref{\DUrole{std,std-ref}{distance}}}} \(D\)
and orbit around a common center with {\hyperref[\detokenize{quantities/orbital/eccentricity:id1}]{\sphinxcrossref{\DUrole{std,std-ref}{eccentricity}}}} \(e\).
We aim to calculate the \sphinxstyleemphasis{radiative} habitable thresholds with insolation
\(S_{s,x}(T_s)\) (where \(x = {\rm I, \, O}\)).
For convenience, we define the
normalized luminosity: \(\tilde L_{s,x} = \frac{L_s}{S_{s,x}(T_s)}\), the
{\hyperref[\detokenize{quantities/habitability/habitable_zones/single_star_habitable_zone:id1}]{\sphinxcrossref{\DUrole{std,std-ref}{SSHZ}}}} \(r_{s, x} = \sqrt{\tilde L_{s,x}}\),
the double\sphinxhyphen{}star equivalent \(r_{\rm AB, x} = \sqrt{\tilde L_{A,x} + \tilde L_{B,x}}\),
and the half mean distance \(b = D/2\).

\sphinxAtStartPar
The S\sphinxhyphen{}type RHZ for {\hyperref[\detokenize{celestial_bodies/star:id1}]{\sphinxcrossref{\DUrole{std,std-ref}{stars}}}} \(A\), namely \(r_{A, x}^{\rm S-type}\), thresholds are estimated by:
\begin{enumerate}
\sphinxsetlistlabels{\arabic}{enumi}{enumii}{}{.}%
\item {} 
\sphinxAtStartPar
Inner: \(r_{\rm A, I}^{\rm S-type} = r_{\rm A, I} \left(1 + \frac{\tilde L_{B,\rm I}}{\left(D - r_{\rm A, I}\right)^2} \right)\).

\item {} 
\sphinxAtStartPar
Outer: \(r_{\rm A, O}^{\rm S-type} = r_{\rm A, O} \left(1 + \frac{\tilde L_{B,\rm O}}{\left(D + r_{\rm A, O}\right)^2} \right)\).

\end{enumerate}

\sphinxAtStartPar
The P\sphinxhyphen{}type RHZ, namely \(r_{\rm AB, x}^{\rm P-type}\), thresholds are estimated by:
\begin{enumerate}
\sphinxsetlistlabels{\arabic}{enumi}{enumii}{}{.}%
\item {} 
\sphinxAtStartPar
Inner: \(r_{\rm AB, I}^{\rm P-type} = \sqrt{ \tilde L_{A,\rm I} \frac{r_{\rm AB, I} + b}{r_{\rm AB, I} - b} + \tilde L_{B,\rm I} \frac{r_{\rm AB, I} - b}{r_{\rm AB, I} + b} - b^2}\)

\item {} 
\sphinxAtStartPar
Outer: \(r_{\rm AB, O}^{\rm P-type} = \sqrt{ \tilde L_{A,\rm O} \frac{r_{\rm AB, O} - b}{r_{\rm AB, O} + b} + \tilde L_{B,\rm O} \frac{r_{\rm AB, O} + b}{r_{\rm AB, O} - b} - b^2}\)

\end{enumerate}


\subsubsection{Permanent Habitable Zone}
\label{\detokenize{quantities/habitability/habitable_zones/permanent_habitable_zone:permanent-habitable-zone}}\label{\detokenize{quantities/habitability/habitable_zones/permanent_habitable_zone::doc}}\phantomsection\label{\detokenize{quantities/habitability/habitable_zones/permanent_habitable_zone:id1}}
\sphinxAtStartPar
The Permanent Habitable Zone (PHZ) is a complex type of habitable zone in a {\hyperref[\detokenize{celestial_systems/binary_system:id1}]{\sphinxcrossref{\DUrole{std,std-ref}{binary system}}}}.
It takes into account the forced and periodically changing {\hyperref[\detokenize{quantities/orbital/eccentricity:id1}]{\sphinxcrossref{\DUrole{std,std-ref}{eccentricity}}}}
of the {\hyperref[\detokenize{celestial_bodies/planet:id1}]{\sphinxcrossref{\DUrole{std,std-ref}{planets}}}} due to the asymmetry
the two {\hyperref[\detokenize{celestial_bodies/star:id1}]{\sphinxcrossref{\DUrole{std,std-ref}{stars}}}} of different {\hyperref[\detokenize{quantities/material/mass:id1}]{\sphinxcrossref{\DUrole{std,std-ref}{mass}}}} introduce to the {\hyperref[\detokenize{celestial_systems/stellar_system:id1}]{\sphinxcrossref{\DUrole{std,std-ref}{stellar system}}}}.
It assumes that the climate of the potential habitable world is fast at immediately adjusting to the periodically changing
{\hyperref[\detokenize{quantities/orbital/eccentricity:id1}]{\sphinxcrossref{\DUrole{std,std-ref}{eccentricity}}}}.


\subsubsection{Average Habitable Zone}
\label{\detokenize{quantities/habitability/habitable_zones/average_habitable_zone:average-habitable-zone}}\label{\detokenize{quantities/habitability/habitable_zones/average_habitable_zone::doc}}\phantomsection\label{\detokenize{quantities/habitability/habitable_zones/average_habitable_zone:id1}}
\sphinxAtStartPar
The Average Habitable Zone (AHZ) is a complex type of habitable zone in a {\hyperref[\detokenize{celestial_systems/binary_system:id1}]{\sphinxcrossref{\DUrole{std,std-ref}{binary system}}}}.
It takes into account the forced and periodically changing {\hyperref[\detokenize{quantities/orbital/eccentricity:id1}]{\sphinxcrossref{\DUrole{std,std-ref}{eccentricity}}}}
of the {\hyperref[\detokenize{celestial_bodies/planet:id1}]{\sphinxcrossref{\DUrole{std,std-ref}{planets}}}} due to the asymmetry
the two {\hyperref[\detokenize{celestial_bodies/star:id1}]{\sphinxcrossref{\DUrole{std,std-ref}{stars}}}} of different {\hyperref[\detokenize{quantities/material/mass:id1}]{\sphinxcrossref{\DUrole{std,std-ref}{mass}}}} introduce to the {\hyperref[\detokenize{celestial_systems/stellar_system:id1}]{\sphinxcrossref{\DUrole{std,std-ref}{stellar system}}}}.
It assumes that the climate of the potential habitable world is slow at adjusting to the periodically changing
{\hyperref[\detokenize{quantities/orbital/eccentricity:id1}]{\sphinxcrossref{\DUrole{std,std-ref}{eccentricity}}}}.


\subsection{Single Star Habitability}
\label{\detokenize{quantities/habitability/single_star_habitability:single-star-habitability}}\label{\detokenize{quantities/habitability/single_star_habitability::doc}}\phantomsection\label{\detokenize{quantities/habitability/single_star_habitability:id1}}
\sphinxAtStartPar
For a single {\hyperref[\detokenize{celestial_bodies/star:id1}]{\sphinxcrossref{\DUrole{std,std-ref}{star}}}}, the habitability is determined by the following conditions:
\begin{enumerate}
\sphinxsetlistlabels{\arabic}{enumi}{enumii}{}{.}%
\item {} 
\sphinxAtStartPar
There must be a viable {\hyperref[\detokenize{quantities/habitability/habitable_zones/single_star_habitable_zone:id1}]{\sphinxcrossref{\DUrole{std,std-ref}{SSHZ}}}} given by the {\hyperref[\detokenize{quantities/insolation_models/relaxed_minimum_limit:id1}]{\sphinxcrossref{\DUrole{std,std-ref}{relaxed minimum limit}}}} and the {\hyperref[\detokenize{quantities/insolation_models/relaxed_maximum_limit:id1}]{\sphinxcrossref{\DUrole{std,std-ref}{relaxed maximum limit}}}}.

\item {} 
\sphinxAtStartPar
The {\hyperref[\detokenize{quantities/insolation_models/relaxed_maximum_limit:id1}]{\sphinxcrossref{\DUrole{std,std-ref}{relaxed maximum limit}}}} must not be smaller than the {\hyperref[\detokenize{quantities/children_orbit_limits/inner_orbit_limit:id1}]{\sphinxcrossref{\DUrole{std,std-ref}{inner orbit limit}}}}.

\item {} 
\sphinxAtStartPar
The {\hyperref[\detokenize{quantities/insolation_models/relaxed_minimum_limit:id1}]{\sphinxcrossref{\DUrole{std,std-ref}{relaxed minimum limit}}}} must not be larger than the {\hyperref[\detokenize{quantities/children_orbit_limits/outer_orbit_limit:id1}]{\sphinxcrossref{\DUrole{std,std-ref}{outer orbit limit}}}}.

\item {} 
\sphinxAtStartPar
The {\hyperref[\detokenize{quantities/life/lifetime:id1}]{\sphinxcrossref{\DUrole{std,std-ref}{lifetime}}}} or {\hyperref[\detokenize{quantities/life/age:id1}]{\sphinxcrossref{\DUrole{std,std-ref}{age}}}}  must be bigger than 1 billion years. This minimum age limit is when life is expected to start developing (see \sphinxhref{https://link.springer.com/article/10.1007/BF00160399}{here}).

\end{enumerate}

\sphinxAtStartPar
These are the most relaxed criteria, so keep in mind that habitability varies
from object to object and the type of life that can survive on them.


\subsection{Binary S\sphinxhyphen{}type Habitability}
\label{\detokenize{quantities/habitability/binary_s_type_habitability:binary-s-type-habitability}}\label{\detokenize{quantities/habitability/binary_s_type_habitability::doc}}\phantomsection\label{\detokenize{quantities/habitability/binary_s_type_habitability:id1}}
\sphinxAtStartPar
For a {\hyperref[\detokenize{celestial_bodies/star:id1}]{\sphinxcrossref{\DUrole{std,std-ref}{star}}}} in an S\sphinxhyphen{}type {\hyperref[\detokenize{celestial_systems/binary_system:id1}]{\sphinxcrossref{\DUrole{std,std-ref}{binary system}}}},
the habitability conditions are the same as the {\hyperref[\detokenize{quantities/habitability/single_star_habitability:id1}]{\sphinxcrossref{\DUrole{std,std-ref}{single star habitability conditions}}}}
but the {\hyperref[\detokenize{quantities/habitability/habitable_zones/single_star_habitable_zone:id1}]{\sphinxcrossref{\DUrole{std,std-ref}{SSHZ}}}} is replaced with {\hyperref[\detokenize{quantities/habitability/habitable_zones/average_habitable_zone:id1}]{\sphinxcrossref{\DUrole{std,std-ref}{AHZ}}}}.
For convenience, the conditions are listed here:
\begin{enumerate}
\sphinxsetlistlabels{\arabic}{enumi}{enumii}{}{.}%
\item {} 
\sphinxAtStartPar
There must be a viable {\hyperref[\detokenize{quantities/habitability/habitable_zones/average_habitable_zone:id1}]{\sphinxcrossref{\DUrole{std,std-ref}{AHZ}}}} given by the {\hyperref[\detokenize{quantities/insolation_models/relaxed_minimum_limit:id1}]{\sphinxcrossref{\DUrole{std,std-ref}{relaxed minimum limit}}}} and the {\hyperref[\detokenize{quantities/insolation_models/relaxed_maximum_limit:id1}]{\sphinxcrossref{\DUrole{std,std-ref}{relaxed maximum limit}}}}.

\item {} 
\sphinxAtStartPar
The {\hyperref[\detokenize{quantities/insolation_models/relaxed_maximum_limit:id1}]{\sphinxcrossref{\DUrole{std,std-ref}{relaxed maximum limit}}}} must not be smaller than the {\hyperref[\detokenize{quantities/children_orbit_limits/inner_orbit_limit:id1}]{\sphinxcrossref{\DUrole{std,std-ref}{inner orbit limit}}}}.

\item {} 
\sphinxAtStartPar
The {\hyperref[\detokenize{quantities/insolation_models/relaxed_minimum_limit:id1}]{\sphinxcrossref{\DUrole{std,std-ref}{relaxed minimum limit}}}} must not be larger than the {\hyperref[\detokenize{quantities/children_orbit_limits/outer_orbit_limit:id1}]{\sphinxcrossref{\DUrole{std,std-ref}{outer orbit limit}}}}.

\item {} 
\sphinxAtStartPar
The {\hyperref[\detokenize{quantities/life/lifetime:id1}]{\sphinxcrossref{\DUrole{std,std-ref}{lifetime}}}} or {\hyperref[\detokenize{quantities/life/age:id1}]{\sphinxcrossref{\DUrole{std,std-ref}{age}}}}  must be bigger than 1 billion years. This minimum age limit is when life is expected to start developing (see \sphinxhref{https://link.springer.com/article/10.1007/BF00160399}{here}).

\end{enumerate}

\sphinxAtStartPar
These are the most relaxed criteria, so keep in mind that habitability varies
from object to object and the type of life that can survive on them.


\subsection{Binary P\sphinxhyphen{}type Habitability}
\label{\detokenize{quantities/habitability/binary_p_type_habitability:binary-p-type-habitability}}\label{\detokenize{quantities/habitability/binary_p_type_habitability::doc}}\phantomsection\label{\detokenize{quantities/habitability/binary_p_type_habitability:id1}}
\sphinxAtStartPar
For a P\sphinxhyphen{}type {\hyperref[\detokenize{celestial_systems/binary_system:id1}]{\sphinxcrossref{\DUrole{std,std-ref}{stellar binary}}}}, the habitability is determined by the following conditions:
\begin{enumerate}
\sphinxsetlistlabels{\arabic}{enumi}{enumii}{}{.}%
\item {} 
\sphinxAtStartPar
There must be a viable {\hyperref[\detokenize{quantities/habitability/habitable_zones/average_habitable_zone:id1}]{\sphinxcrossref{\DUrole{std,std-ref}{P\sphinxhyphen{}Type AHZ}}}} given by the {\hyperref[\detokenize{quantities/insolation_models/relaxed_minimum_limit:id1}]{\sphinxcrossref{\DUrole{std,std-ref}{relaxed minimum limit}}}} and the {\hyperref[\detokenize{quantities/insolation_models/relaxed_maximum_limit:id1}]{\sphinxcrossref{\DUrole{std,std-ref}{relaxed maximum limit}}}}.

\item {} 
\sphinxAtStartPar
The {\hyperref[\detokenize{quantities/insolation_models/relaxed_maximum_limit:id1}]{\sphinxcrossref{\DUrole{std,std-ref}{relaxed maximum limit}}}} must not be smaller than the {\hyperref[\detokenize{quantities/children_orbit_limits/inner_orbit_limit:id1}]{\sphinxcrossref{\DUrole{std,std-ref}{inner orbit limit}}}}.

\item {} 
\sphinxAtStartPar
The {\hyperref[\detokenize{quantities/insolation_models/relaxed_minimum_limit:id1}]{\sphinxcrossref{\DUrole{std,std-ref}{relaxed minimum limit}}}} must not be larger than the {\hyperref[\detokenize{quantities/children_orbit_limits/outer_orbit_limit:id1}]{\sphinxcrossref{\DUrole{std,std-ref}{outer orbit limit}}}}.

\item {} 
\sphinxAtStartPar
The {\hyperref[\detokenize{quantities/life/lifetime:id1}]{\sphinxcrossref{\DUrole{std,std-ref}{lifetime}}}} or {\hyperref[\detokenize{quantities/life/age:id1}]{\sphinxcrossref{\DUrole{std,std-ref}{age}}}} of the primary {\hyperref[\detokenize{celestial_bodies/star:id1}]{\sphinxcrossref{\DUrole{std,std-ref}{star}}}} must be bigger than 1 billion years. This minimum age limit is when life is expected to start developing (see \sphinxhref{https://link.springer.com/article/10.1007/BF00160399}{here}).

\end{enumerate}

\sphinxAtStartPar
These are the most relaxed criteria, so keep in mind that habitability varies
from object to object and the type of life that can survive on them.


\subsection{Planet Habitability}
\label{\detokenize{quantities/habitability/planet_habitability:planet-habitability}}\label{\detokenize{quantities/habitability/planet_habitability::doc}}\phantomsection\label{\detokenize{quantities/habitability/planet_habitability:id1}}
\sphinxAtStartPar
For a {\hyperref[\detokenize{celestial_bodies/planet:id1}]{\sphinxcrossref{\DUrole{std,std-ref}{planet}}}}, the habitability is determined by the following conditions:
\begin{enumerate}
\sphinxsetlistlabels{\arabic}{enumi}{enumii}{}{.}%
\item {} 
\sphinxAtStartPar
The {\hyperref[\detokenize{quantities/material/mass:id1}]{\sphinxcrossref{\DUrole{std,std-ref}{mass}}}} must be less than \(5 \, {\rm M_e}\).

\item {} 
\sphinxAtStartPar
The {\hyperref[\detokenize{quantities/material/mass:id1}]{\sphinxcrossref{\DUrole{std,std-ref}{mass}}}} must be more than \(0.0268 \, {\rm M_e}\) for water worlds and more than \(0.1 \, {\rm M_e}\) for other {\hyperref[\detokenize{quantities/material/composition_type:id1}]{\sphinxcrossref{\DUrole{std,std-ref}{composition types}}}}.

\item {} 
\sphinxAtStartPar
The {\hyperref[\detokenize{quantities/geometric/radius:id1}]{\sphinxcrossref{\DUrole{std,std-ref}{radius}}}} must be between \(0.5 \, {\rm R_e}\) and \(1.5 \, {\rm R_e}\).

\item {} 
\sphinxAtStartPar
The {\hyperref[\detokenize{celestial_bodies/planet:id1}]{\sphinxcrossref{\DUrole{std,std-ref}{planet’s}}}} parent must have a viable relaxed {\hyperref[\detokenize{quantities/habitability/habitable_zones/habitable_zones:id1}]{\sphinxcrossref{\DUrole{std,std-ref}{HZ}}}}.

\item {} 
\sphinxAtStartPar
The {\hyperref[\detokenize{quantities/orbital/semi_major_axis:id1}]{\sphinxcrossref{\DUrole{std,std-ref}{semi\sphinxhyphen{}major axis}}}} must be within the relaxed {\hyperref[\detokenize{quantities/habitability/habitable_zones/habitable_zones:id1}]{\sphinxcrossref{\DUrole{std,std-ref}{HZ}}}} of the parent.

\item {} 
\sphinxAtStartPar
The orbit must be {\hyperref[\detokenize{quantities/orbital/orbital_stability:id1}]{\sphinxcrossref{\DUrole{std,std-ref}{stable}}}}.

\item {} 
\sphinxAtStartPar
The {\hyperref[\detokenize{quantities/surface/internal_heating/tectonic_activity:id1}]{\sphinxcrossref{\DUrole{std,std-ref}{tectonic activity}}}} must be labeled as “Medium Low”, “Medium” or “Medium High”.

\end{enumerate}


\subsection{Satellite Habitability}
\label{\detokenize{quantities/habitability/satellite_habitability:satellite-habitability}}\label{\detokenize{quantities/habitability/satellite_habitability::doc}}\phantomsection\label{\detokenize{quantities/habitability/satellite_habitability:id1}}
\sphinxAtStartPar
For a {\hyperref[\detokenize{celestial_bodies/satellite:id1}]{\sphinxcrossref{\DUrole{std,std-ref}{satellite}}}}, the habitability conditions are the same as the
{\hyperref[\detokenize{quantities/habitability/planet_habitability:id1}]{\sphinxcrossref{\DUrole{std,std-ref}{planetary habitability conditions}}}}
but the the reference to the {\hyperref[\detokenize{quantities/habitability/habitable_zones/habitable_zones:id1}]{\sphinxcrossref{\DUrole{std,std-ref}{HZ}}}} of the parent, must be changed to the grandparent.
For convenience, the conditions are listed here:
\begin{enumerate}
\sphinxsetlistlabels{\arabic}{enumi}{enumii}{}{.}%
\item {} 
\sphinxAtStartPar
The {\hyperref[\detokenize{quantities/material/mass:id1}]{\sphinxcrossref{\DUrole{std,std-ref}{mass}}}} must be less than \(5 \, {\rm M_e}\).

\item {} 
\sphinxAtStartPar
The {\hyperref[\detokenize{quantities/material/mass:id1}]{\sphinxcrossref{\DUrole{std,std-ref}{mass}}}} must be more than \(0.0268 \, {\rm M_e}\) for water worlds and more than \(0.1 \, {\rm M_e}\) for other {\hyperref[\detokenize{quantities/material/composition_type:id1}]{\sphinxcrossref{\DUrole{std,std-ref}{composition types}}}}.

\item {} 
\sphinxAtStartPar
The {\hyperref[\detokenize{quantities/geometric/radius:id1}]{\sphinxcrossref{\DUrole{std,std-ref}{radius}}}} must be between \(0.5 \, {\rm R_e}\) and \(1.5 \, {\rm R_e}\).

\item {} 
\sphinxAtStartPar
The {\hyperref[\detokenize{celestial_bodies/satellite:id1}]{\sphinxcrossref{\DUrole{std,std-ref}{satellite’s}}}} grandparent must have a viable relaxed {\hyperref[\detokenize{quantities/habitability/habitable_zones/habitable_zones:id1}]{\sphinxcrossref{\DUrole{std,std-ref}{HZ}}}}.

\item {} 
\sphinxAtStartPar
The {\hyperref[\detokenize{quantities/orbital/semi_major_axis:id1}]{\sphinxcrossref{\DUrole{std,std-ref}{semi\sphinxhyphen{}major axis}}}} of the {\hyperref[\detokenize{celestial_bodies/satellite:id1}]{\sphinxcrossref{\DUrole{std,std-ref}{satellite’s}}}} parent must be within the relaxed {\hyperref[\detokenize{quantities/habitability/habitable_zones/habitable_zones:id1}]{\sphinxcrossref{\DUrole{std,std-ref}{HZ}}}} of the {\hyperref[\detokenize{celestial_bodies/satellite:id1}]{\sphinxcrossref{\DUrole{std,std-ref}{satellite’s}}}} grandparent.

\item {} 
\sphinxAtStartPar
The orbit must be {\hyperref[\detokenize{quantities/orbital/orbital_stability:id1}]{\sphinxcrossref{\DUrole{std,std-ref}{stable}}}}.

\item {} 
\sphinxAtStartPar
The {\hyperref[\detokenize{quantities/surface/internal_heating/tectonic_activity:id1}]{\sphinxcrossref{\DUrole{std,std-ref}{tectonic activity}}}} must be labeled as “Medium Low”, “Medium” or “Medium High”.

\end{enumerate}


\subsection{Trojan Satellite Habitability}
\label{\detokenize{quantities/habitability/trojan_satellite_habitability:trojan-satellite-habitability}}\label{\detokenize{quantities/habitability/trojan_satellite_habitability::doc}}\phantomsection\label{\detokenize{quantities/habitability/trojan_satellite_habitability:id1}}
\sphinxAtStartPar
For a {\hyperref[\detokenize{celestial_bodies/trojan_satellite:id1}]{\sphinxcrossref{\DUrole{std,std-ref}{trojan satellite}}}}, the habitability conditions are the same as the
{\hyperref[\detokenize{quantities/habitability/planet_habitability:id1}]{\sphinxcrossref{\DUrole{std,std-ref}{planetary habitability conditions}}}}
but the the reference to the {\hyperref[\detokenize{quantities/habitability/habitable_zones/habitable_zones:id1}]{\sphinxcrossref{\DUrole{std,std-ref}{HZ}}}} of the parent, must be changed to the grandparent.
For convenience, the conditions are listed here:
\begin{enumerate}
\sphinxsetlistlabels{\arabic}{enumi}{enumii}{}{.}%
\item {} 
\sphinxAtStartPar
The {\hyperref[\detokenize{quantities/material/mass:id1}]{\sphinxcrossref{\DUrole{std,std-ref}{mass}}}} must be less than \(5 \, {\rm M_e}\).

\item {} 
\sphinxAtStartPar
The {\hyperref[\detokenize{quantities/material/mass:id1}]{\sphinxcrossref{\DUrole{std,std-ref}{mass}}}} must be more than \(0.0268 \, {\rm M_e}\) for water worlds and more than \(0.1 \, {\rm M_e}\) for other {\hyperref[\detokenize{quantities/material/composition_type:id1}]{\sphinxcrossref{\DUrole{std,std-ref}{composition types}}}}.

\item {} 
\sphinxAtStartPar
The {\hyperref[\detokenize{quantities/geometric/radius:id1}]{\sphinxcrossref{\DUrole{std,std-ref}{radius}}}} must be between \(0.5 \, {\rm R_e}\) and \(1.5 \, {\rm R_e}\).

\item {} 
\sphinxAtStartPar
The {\hyperref[\detokenize{celestial_bodies/trojan_satellite:id1}]{\sphinxcrossref{\DUrole{std,std-ref}{trojan satellite’s}}}} grandparent must have a viable relaxed {\hyperref[\detokenize{quantities/habitability/habitable_zones/habitable_zones:id1}]{\sphinxcrossref{\DUrole{std,std-ref}{HZ}}}}.

\item {} 
\sphinxAtStartPar
The {\hyperref[\detokenize{quantities/orbital/semi_major_axis:id1}]{\sphinxcrossref{\DUrole{std,std-ref}{semi\sphinxhyphen{}major axis}}}} must be within the relaxed {\hyperref[\detokenize{quantities/habitability/habitable_zones/habitable_zones:id1}]{\sphinxcrossref{\DUrole{std,std-ref}{HZ}}}} of the grandparent.

\item {} 
\sphinxAtStartPar
The orbit of the parent must be {\hyperref[\detokenize{quantities/orbital/orbital_stability:id1}]{\sphinxcrossref{\DUrole{std,std-ref}{stable}}}}.

\item {} 
\sphinxAtStartPar
The {\hyperref[\detokenize{quantities/surface/internal_heating/tectonic_activity:id1}]{\sphinxcrossref{\DUrole{std,std-ref}{tectonic activity}}}} must be labeled as “Medium Low”, “Medium” or “Medium High”.

\end{enumerate}


\chapter{GUI}
\label{\detokenize{gui/gui:gui}}\label{\detokenize{gui/gui::doc}}\phantomsection\label{\detokenize{gui/gui:id1}}
\sphinxAtStartPar
The gui is (a hopefully) easy to navigate tool, once you get the basics. We will start
with an example, and move forward to talking about the individual buttons an functionalities
that are available.


\section{Simple example}
\label{\detokenize{gui/gui:simple-example}}
\sphinxAtStartPar
You can design a new system, by going to Files \sphinxhyphen{}\textgreater{} New Project
and choose between a {\hyperref[\detokenize{celestial_systems/planetary_system:id1}]{\sphinxcrossref{\DUrole{std,std-ref}{planetary}}}}, a
{\hyperref[\detokenize{celestial_systems/stellar_system:id1}]{\sphinxcrossref{\DUrole{std,std-ref}{stellar}}}} or a {\hyperref[\detokenize{celestial_systems/multi_stellar_system:id1}]{\sphinxcrossref{\DUrole{std,std-ref}{multi\sphinxhyphen{}stellar}}}} system.
In the his example, we will choose a {\hyperref[\detokenize{celestial_systems/multi_stellar_system:id1}]{\sphinxcrossref{\DUrole{std,std-ref}{multi\sphinxhyphen{}stellar system}}}},
which includes one or more of the other systems.

\sphinxAtStartPar
When you create a {\hyperref[\detokenize{celestial_systems/multi_stellar_system:id1}]{\sphinxcrossref{\DUrole{std,std-ref}{multi\sphinxhyphen{}stellar system}}}},
you create an {\hyperref[\detokenize{celestial_systems/binary_system:id1}]{\sphinxcrossref{\DUrole{std,std-ref}{S\sphinxhyphen{}type stellar binary system}}}}
(two {\hyperref[\detokenize{celestial_bodies/star:id1}]{\sphinxcrossref{\DUrole{std,std-ref}{stars}}}} orbiting around each other with a distance big enough that other objects
can orbit around each {\hyperref[\detokenize{celestial_bodies/star:id1}]{\sphinxcrossref{\DUrole{std,std-ref}{stars}}}} individually). By default, a {\hyperref[\detokenize{celestial_systems/binary_system:id1}]{\sphinxcrossref{\DUrole{std,std-ref}{binary system}}}}
with two {\hyperref[\detokenize{celestial_bodies/star:id1}]{\sphinxcrossref{\DUrole{std,std-ref}{stars}}}} of the same {\hyperref[\detokenize{quantities/material/mass:id1}]{\sphinxcrossref{\DUrole{std,std-ref}{mass}}}} as our sun are generated,
with {\hyperref[\detokenize{quantities/orbital/semi_major_axis:id1}]{\sphinxcrossref{\DUrole{std,std-ref}{mean distance}}}} of 500 AU
and an eccentric orbit of 0.6 {\hyperref[\detokenize{quantities/orbital/eccentricity:id1}]{\sphinxcrossref{\DUrole{std,std-ref}{eccentricity}}}}.
You can find information about the  {\hyperref[\detokenize{celestial_systems/binary_system:id1}]{\sphinxcrossref{\DUrole{std,std-ref}{binary system}}}} if you right\sphinxhyphen{}click on the binary,
and select Details from the context menu. Similarly, information on the {\hyperref[\detokenize{celestial_bodies/star:id1}]{\sphinxcrossref{\DUrole{std,std-ref}{stars}}}} can be
found in the Details context menu item.

\sphinxAtStartPar
To add elements in a given system, right\sphinxhyphen{}click on the list of the elements you want
to add on, and choose the Add \textless{}element of choice\textgreater{} option. For example, to add new
{\hyperref[\detokenize{celestial_systems/planetary_system:id1}]{\sphinxcrossref{\DUrole{std,std-ref}{planetary system}}}} on the first {\hyperref[\detokenize{celestial_systems/stellar_system:id1}]{\sphinxcrossref{\DUrole{std,std-ref}{stellar system}}}},
right\sphinxhyphen{}click on the Planetary Systems item within the first {\hyperref[\detokenize{celestial_systems/stellar_system:id1}]{\sphinxcrossref{\DUrole{std,std-ref}{stellar system}}}},
and choose add Planetary System. A small prompt
will pop up, where you choose the name of the the {\hyperref[\detokenize{celestial_systems/planetary_system:id1}]{\sphinxcrossref{\DUrole{std,std-ref}{planetary system}}}}
and the {\hyperref[\detokenize{celestial_bodies/planet:id1}]{\sphinxcrossref{\DUrole{std,std-ref}{planet}}}}.
You can then open up the new {\hyperref[\detokenize{celestial_systems/planetary_system:id1}]{\sphinxcrossref{\DUrole{std,std-ref}{planetary system}}}} item,
and find out the new {\hyperref[\detokenize{celestial_bodies/planet:id1}]{\sphinxcrossref{\DUrole{std,std-ref}{planet}}}}, as
well as the empty items {\hyperref[\detokenize{celestial_bodies/satellite:id1}]{\sphinxcrossref{\DUrole{std,std-ref}{satellites}}}} and {\hyperref[\detokenize{celestial_bodies/trojan:id1}]{\sphinxcrossref{\DUrole{std,std-ref}{Trojans}}}}.
You can add {\hyperref[\detokenize{celestial_bodies/satellite:id1}]{\sphinxcrossref{\DUrole{std,std-ref}{satellites}}}} and {\hyperref[\detokenize{celestial_bodies/trojan:id1}]{\sphinxcrossref{\DUrole{std,std-ref}{Trojans}}}} in
a similar way. To modify the {\hyperref[\detokenize{celestial_bodies/planet:id1}]{\sphinxcrossref{\DUrole{std,std-ref}{planet}}}}’s characteristics, open up the details menu of the
{\hyperref[\detokenize{celestial_bodies/planet:id1}]{\sphinxcrossref{\DUrole{std,std-ref}{planet}}}}. To add an {\hyperref[\detokenize{celestial_bodies/asteroid_belt:id1}]{\sphinxcrossref{\DUrole{std,std-ref}{asteroid belt}}}} in the
{\hyperref[\detokenize{celestial_systems/stellar_system:id1}]{\sphinxcrossref{\DUrole{std,std-ref}{stellar system}}}} of your choice, follow the same
procedure as for a {\hyperref[\detokenize{celestial_systems/planetary_system:id1}]{\sphinxcrossref{\DUrole{std,std-ref}{planetary system}}}},
but now do it through the Asteroid Belts item list.

\sphinxAtStartPar
To delete an element, simply right\sphinxhyphen{}click on the undesired element and choose
Delete Permanently.
Some elements (e.g. {\hyperref[\detokenize{celestial_bodies/planet:id1}]{\sphinxcrossref{\DUrole{std,std-ref}{planets}}}} in {\hyperref[\detokenize{celestial_systems/planetary_system:id1}]{\sphinxcrossref{\DUrole{std,std-ref}{planetary systems}}}}
or {\hyperref[\detokenize{celestial_bodies/star:id1}]{\sphinxcrossref{\DUrole{std,std-ref}{stars}}}} in {\hyperref[\detokenize{celestial_systems/stellar_system:id1}]{\sphinxcrossref{\DUrole{std,std-ref}{stellar systems}}}}) are not deletable, only replaceable.

\sphinxAtStartPar
To save your progress, go to Files \sphinxhyphen{}\textgreater{} Save project and choose the name under
which you want to save the file. The files can get quite big due to saving
images for every single element. The average {\hyperref[\detokenize{celestial_systems/stellar_system:id1}]{\sphinxcrossref{\DUrole{std,std-ref}{stellar system}}}}
should be less than 100 MB.

\sphinxAtStartPar
To open an existing project in a new tab, go to files \sphinxhyphen{}\textgreater{} Open Project and
select the project of your choice.

\sphinxAtStartPar
To open the documentation through the GUI, go to Help \sphinxhyphen{}\textgreater{} Documentation.


\section{Details Dialog}
\label{\detokenize{gui/gui:details-dialog}}
\sphinxAtStartPar
Opening a detail dialog, depending on the element opened, there
are multiple tabs and for each one there are many options to modify and explore.
Each {\hyperref[\detokenize{quantities/quantities:id1}]{\sphinxcrossref{\DUrole{std,std-ref}{quantity}}}} you find in the tab that has the information
symbol on the side, can be double clicked to displace the help menu entry on that
{\hyperref[\detokenize{quantities/quantities:id1}]{\sphinxcrossref{\DUrole{std,std-ref}{quantity}}}}.

\sphinxAtStartPar
The main tab is Designations, a tab that contains general information,
such as name and parents (which body they {\hyperref[\detokenize{quantities/orbital/orbital:id1}]{\sphinxcrossref{\DUrole{std,std-ref}{orbit}}}} or are part of),
and other classification and composition characteristics.

\sphinxAtStartPar
The second tab is the physical characteristics tab, which contains
information about the {\hyperref[\detokenize{quantities/material/mass:id1}]{\sphinxcrossref{\DUrole{std,std-ref}{mass}}}}, {\hyperref[\detokenize{quantities/geometric/radius:id1}]{\sphinxcrossref{\DUrole{std,std-ref}{radius}}}}, {\hyperref[\detokenize{quantities/rotational/spin_period:id1}]{\sphinxcrossref{\DUrole{std,std-ref}{rotation}}}}, and {\hyperref[\detokenize{quantities/life/age:id1}]{\sphinxcrossref{\DUrole{std,std-ref}{age}}}}.
For {\hyperref[\detokenize{celestial_bodies/star:id1}]{\sphinxcrossref{\DUrole{std,std-ref}{stars}}}}, it also includes some {\hyperref[\detokenize{quantities/surface/emission/emission:id1}]{\sphinxcrossref{\DUrole{std,std-ref}{spectral/surface}}}} characteristics.

\sphinxAtStartPar
Another tab would be the {\hyperref[\detokenize{quantities/orbital/orbital:id1}]{\sphinxcrossref{\DUrole{std,std-ref}{orbit}}}} characteristics, which includes
{\hyperref[\detokenize{quantities/orbital/eccentricity:id1}]{\sphinxcrossref{\DUrole{std,std-ref}{eccentricity}}}}, {\hyperref[\detokenize{quantities/orbital/semi_major_axis:id1}]{\sphinxcrossref{\DUrole{std,std-ref}{semi\sphinxhyphen{}major axis}}}} etc.

\sphinxAtStartPar
The {\hyperref[\detokenize{quantities/children_orbit_limits/children_orbit_limits:id1}]{\sphinxcrossref{\DUrole{std,std-ref}{children orbit limit}}}} tab contain different types of orbit
limits for the bodies that orbit around the body for which the detail dialog is open.

\sphinxAtStartPar
The {\hyperref[\detokenize{quantities/surface/surface:id1}]{\sphinxcrossref{\DUrole{std,std-ref}{surface}}}} dialog contains all potential surface related
characteristics such as {\hyperref[\detokenize{quantities/surface/emission/temperature:id1}]{\sphinxcrossref{\DUrole{std,std-ref}{temperature}}}}, {\hyperref[\detokenize{quantities/surface/gravity/surface_gravity:id1}]{\sphinxcrossref{\DUrole{std,std-ref}{gravitational acceleration}}}},
{\hyperref[\detokenize{quantities/surface/angular_diameter:id1}]{\sphinxcrossref{\DUrole{std,std-ref}{size of parent in the sky}}}}, and {\hyperref[\detokenize{quantities/surface/internal_heating/tectonic_activity:id1}]{\sphinxcrossref{\DUrole{std,std-ref}{tectonic activity}}}}.

\sphinxAtStartPar
The {\hyperref[\detokenize{quantities/insolation_models/insolation_models:id1}]{\sphinxcrossref{\DUrole{std,std-ref}{insolation}}}} tab contains the two different {\hyperref[\detokenize{quantities/insolation_models/insolation_models:id1}]{\sphinxcrossref{\DUrole{std,std-ref}{insolation models}}}}
that can be used to calculate the {\hyperref[\detokenize{quantities/habitability/habitable_zones/habitable_zones:id1}]{\sphinxcrossref{\DUrole{std,std-ref}{habitable zone}}}} around
{\hyperref[\detokenize{celestial_bodies/star:id1}]{\sphinxcrossref{\DUrole{std,std-ref}{stars}}}} and {\hyperref[\detokenize{celestial_systems/binary_system:id1}]{\sphinxcrossref{\DUrole{std,std-ref}{stellar binaries}}}}.

\sphinxAtStartPar
The {\hyperref[\detokenize{quantities/habitability/habitability:id1}]{\sphinxcrossref{\DUrole{std,std-ref}{habitability}}}} tab contains all the information relevant
to the {\hyperref[\detokenize{quantities/habitability/habitability:id1}]{\sphinxcrossref{\DUrole{std,std-ref}{habitability}}}} of the body. For {\hyperref[\detokenize{celestial_bodies/star:id1}]{\sphinxcrossref{\DUrole{std,std-ref}{stars}}}} that includes
the {\hyperref[\detokenize{quantities/habitability/habitable_zones/habitable_zones:id1}]{\sphinxcrossref{\DUrole{std,std-ref}{habitable zone}}}}.
For {\hyperref[\detokenize{celestial_bodies/planet:id1}]{\sphinxcrossref{\DUrole{std,std-ref}{planets}}}} and {\hyperref[\detokenize{celestial_bodies/satellite:id1}]{\sphinxcrossref{\DUrole{std,std-ref}{satellites}}}},
the {\hyperref[\detokenize{quantities/habitability/habitability:id1}]{\sphinxcrossref{\DUrole{std,std-ref}{habitability}}}} is dependent on multiple factors.
Each one that is violated is portrayed on the habitability violations box.

\sphinxAtStartPar
Finally, the image tab contains the default image, or a option for the user to add their own.


\chapter{Attributions}
\label{\detokenize{attributions/attributions:attributions}}\label{\detokenize{attributions/attributions::doc}}

\section{Images}
\label{\detokenize{attributions/attributions:images}}\label{\detokenize{attributions/attributions:id1}}\begin{itemize}
\item {} 
\sphinxAtStartPar
cold\sphinxhyphen{}ocean\sphinxhyphen{}world.png (Enceladus): This image was originally posted to Flickr by Kevin M. Gill at \sphinxurl{https://flickr.com/photos/53460575@N03/30795220287} (archive). It was reviewed on 2 July 2019 by FlickreviewR 2 and was confirmed to be licensed under the terms of the cc\sphinxhyphen{}by\sphinxhyphen{}2.0.

\end{itemize}



\renewcommand{\indexname}{Index}
\printindex
\end{document}